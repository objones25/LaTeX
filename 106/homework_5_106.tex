\documentclass[10pt]{article}
\usepackage{graphicx}
\usepackage{amssymb}
\usepackage[fleqn]{amsmath}
\usepackage{nccmath}
\usepackage{cases}
\usepackage{hyperref}
\usepackage{multicol}
\usepackage{tikz}
\usepackage{tikz}
\usetikzlibrary{shapes.geometric}
\usepackage{pgfplots}
\usepackage{enumitem}
\pgfplotsset{compat=1.18}
\usepackage{float}

\title{\bf Math 106: Problem Set 5}
\date{2/25/2024}
\author{\bf Owen Jones}
\begin{document}
\maketitle
\textbf{Quadratic Formula}
\begin{itemize}
    \item $x^2+px+c=0\\
    x^2+px+\frac{p^2}{4}=\frac{p^2}{4}-c\\
    {(x+\frac{p}{2})}^2=\frac{p^2}{4}-c\\
    x+\frac{p}{2}=\pm\sqrt{\frac{p^2}{4}-c}\\
    x=-\frac{p}{2}\pm\sqrt{\frac{p^2}{4}-c}\\
    x=\frac{-p\pm\sqrt{p^2-4c}}{2}$\\
    \item $ax^2+bx+c=0\\
    x^2+\frac{b}{a}x+\frac{c}{a}=0$\\
    Let $p=\frac{b}{a},c=\frac{c}{a}$\\
    $x=\frac{-p\pm\sqrt{p^2-4c}}{2}\Rightarrow x=\frac{-\frac{b}{a}\pm\sqrt{\frac{b^2}{a^2}-4\frac{c}{a}}}{2}\\
    x=\frac{-\frac{b}{a}\pm\frac{1}{a}\sqrt{b^2-4ac}}{2}\\
    x=\frac{-b\pm\sqrt{b^2-4ac}}{2a}$
\end{itemize}

\begin{enumerate}
    \item [\textbf{6.4.4}] Suppose $\sqrt[3]{2}=a+b\sqrt{c}$. 
    Cubing both sides, $2=a^3+3a^2b\sqrt{c}+3ab^2c+b^3c\sqrt{c}$.
    $(a^3+3ab^2c-2)+(3a^2b+b^3c)\sqrt{c}=0\Leftrightarrow a^3+3ab^2c-2=3a^2b+b^3c=0$ from \textbf{6.4.3}.
    Thus, $a^3+3ab^2c=2,3a^2b+b^3c=0$.
    \item [\textbf{6.4.5}] Because $3a^2b+b^3c=0\Rightarrow -3a^2b-b^3c=0$. 
    Thus, $2=a^3-3a^2b\sqrt{c}+3ab^2c-b^3c\sqrt{c}$.
    Taking the cube root of both sides $\sqrt[3]{2}=\sqrt[3]{a^3-3a^2b\sqrt{c}+3ab^2c-b^3c\sqrt{c}}=a-b\sqrt{c}$.
    This is a contradiction because $2\sqrt[3]{2}=a+b\sqrt{c}+a-b\sqrt{c}=2a\Rightarrow \sqrt[3]{2}=a$ which is impossible because $\sqrt[3]{2}\notin F_k$ but $a\in F_k$.
    \item [\textbf{6.5.2}] $y^3=2\Rightarrow p=0,q=2\\
    y=\sqrt[3]{\frac{2}{2}+\sqrt{{(\frac{2}{2})}^2-{(\frac{0}{3})}^3}}+\sqrt[3]{\frac{2}{2}-\sqrt{{(\frac{2}{2})}^2-{(\frac{0}{3})}^3}}\\
    =\sqrt[3]{1+\sqrt{1}}+\sqrt[3]{1-\sqrt{1}}=\sqrt[3]{2}$.
    \item [\textbf{6.5.3}] $y=\sqrt[3]{\frac{6}{2}+\sqrt{{(\frac{6}{2})}^2-{(\frac{6}{3})}^3}}+\sqrt[3]{\frac{6}{2}-\sqrt{{(\frac{6}{2})}^2-{(\frac{6}{3})}^3}}\\
    =\sqrt[3]{3+1}+\sqrt[3]{3-1}=\sqrt[3]{4}+\sqrt[3]{2}\\
    \Rightarrow {(\sqrt[3]{4}+\sqrt[3]{2})}^3=4+3\sqrt[3]{32}+3\sqrt[3]{16}+2=6+6(\sqrt[3]{4}+\sqrt[3]{2})$
    \item [\textbf{6.7.1}] 
    $\displaystyle x^n-a^n=x^n-a^n+\sum_{i=1}^{n-1}a^i x^{n-i}-a^i x^{n-i}\\
    =x^n-ax^{n-1}+ax^{n-1}-a^2x^{n-2}+a^2x^{n-2}+\cdots+a^{n-1}x-a^n\\
    =(x-a)(x^{n-1}+ax^{n-2}+\cdots +a^{n-1})$\\
    $\frac{x^n-a^n}{x-a}$ is the sum of a geometric series with $a_0=a^{n-1}$ and common ratio $r=\frac{x}{a}$.
    \item [\textbf{6.7.2}] $\displaystyle p(x)-p(a)=\sum_{i=0}^{k}a_i(x^i-a^i)$. We showed that $(x-a)\mid (x^i-a^i)$ for $i\in\mathbb{N}$ in \textbf{6.7.1}, so $(x-a)$ divides a linear combination of $(x^i-a^i)$.
    Thus, $\displaystyle (x-a)\mid \sum_{i=0}^{k}a_i(x^i-a^i)\Rightarrow (x-a)\mid p(x)-p(a)$.
    \item [\textbf{6.7.3}] Suppose $p(a)=0$. By \textbf{6.7.2} we have $(x-a)\mid p(x)-p(a)$. However, $p(x)-p(a)=p(x)-0=p(x)$, so $(x-a)\mid p(x)$.
\end{enumerate}
\end{document}