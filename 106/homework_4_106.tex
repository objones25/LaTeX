\documentclass[10pt]{article}
\usepackage{graphicx}
\usepackage{amssymb}
\usepackage[fleqn]{amsmath}
\usepackage{nccmath}
\usepackage{cases}
\usepackage{hyperref}
\usepackage{multicol}
\usepackage{tikz}
\usepackage{tikz}
\usetikzlibrary{shapes.geometric}
\usepackage{pgfplots}
\usepackage{enumitem}
\pgfplotsset{compat=1.18}
\usepackage{float}

\title{\bf Math 106: Problem Set 4}
\date{2/18/2024}
\author{\bf Owen Jones}
\begin{document}
\maketitle
\textbf{Prime Divisor Property}\\
Let $p$ be prime, and suppose $p\mid ab$.\\
WLOG assume $p\nmid a$. We want to show $p\mid b$.
Since the only factors of $p$ are $1$ and $p$, if $p\nmid a$, then $\gcd(p,a)=1$.
It follows there exists integers $m,n$ s.t $am+pn=1$. It follows $abm+bpn=b$.
Since $p\mid bpn$ and $p\mid abm$, $p$ must divide a linear combination of $bpn$ and $abm$.
Hence, $p\mid b$.\\  
\textbf{Fundamental Theorem of Arithmetic}\\
Part 1: prime factorization of $n$\\
Pf by induction\\
Base case: $n=2$ is a prime number, so its factors are $1$ and itself. Thus, it's prime factorization is $2$.\\
Induction hypothesis: Assume for some $n>2$ that every integer $k$ s.t $2\le k<n$ can be factored into a product of primes.\\
Induction step: We want to show $n$ can be factored into a product of primes.\\
The case where $n$ is a prime is trivial. Its factorization is just $n$.
Suppose $n$ is not a prime. THus, $n$ has a proper divisor. Let $d\mid n$ where $d\neq n$.
It follows there exists an integer $k$ s.t $dk=n$. 
Since both $d$ and $k$ are less than $n$, the induction hypothesis states that $d$ and $k$ can be written as a product of primes. 
Thus, $n$ can be written as a product of primes.
Hence, by induction, every integer $2$ or greater can be written as a product of primes.\\

Part 2: Uniqueness of the prime factorization\\
Pf by contradiction\\
Assume to the contrary the prime factorization of $n$ is not unique.\\
Let $n=p_1p_2\ldots p_k$ and $n=q_1q_2\ldots q_m$ be prime factorizations for $n$. 
Let $s_1,s_2,\ldots,s_l$ be the shared prime factors between the two factorizations.\\
We relabel and reindex each factorization as $n=s_1s_2\ldots s_l p^*_{l+1}\ldots p^*_k$ and $n=s_1s_2\ldots s_l q^*_{l+1}\ldots q^*_m$.
By assuption, there exists some $p^*_i\notin \{q^*_{l+1},\ldots, q^*_m\}$. 
However, $p_\mid \frac{n}{s_1s_2\ldots s_l}=q^*_{l+1}\ldots q^*_m$, so $p_i=q_j$ for some $j=l+1,\ldots m$.
Thus, we obtain a contradiction because $p^*_i\notin \{q^*_{l+1},\ldots, q^*_m\}$.
Hence, $n$ has a unique prime factorization.
\begin{enumerate}
    \item [\textbf{5.2.1}] Suppose $mp\equiv 1\pmod{a}$. 
    It follows there exists some integer $k$ s.t $mp-ak=1$. 
    Because $mp-ak$ is a linear combination of $p$ and $a$, $\gcd(p,a)\mid mp-ak$. 
    Thus, $\gcd(p,a)\mid 1$. However, the only divisor of $1$ is itself, so $\gcd(p,a)=1$.
    \item [\textbf{5.2.2}] Suppose $m_1,\ldots m_k$ be pairwise relatively prime integers and let $x$ be an integer that satisfies the following system of congruence relations:\\
    \begin{align*}
        x\equiv a_1\pmod{m_1}\\
        x\equiv a_2\pmod{m_2}\\
        \ldots\\
        x\equiv a_k\pmod{m_k}
    \end{align*}\\
    Let $M=\prod m_i$ and let $z_i=\frac{M}{m_i}$. 
    Because $z_i$ and $m_i$ are relatively prime, Bezout's identity says there exists an integer $y_i$ s.t $y_i z_i\equiv 1\pmod{m_i}$.
    It follows $a_i y_i z_i\equiv a_i \pmod{m_i}$.
    For any $i,j$ s.t $i\neq j$ $m_j\mid z_i$, so $a_i y_i z_i\equiv 0\pmod{m_j}$.
    Thus, $\displaystyle x=\sum_{i=1}^{k}a_i y_i z_i$ is a solution to the system of congruence relations.\\
    Let $x_1,x_2$ both be solutions to the system of congruence relations.
    It follows 
    \begin{align*}
        x_1\equiv x_2\pmod{m_1}\\
        x_1\equiv x_2\pmod{m_2}\\
        \ldots\\
        x_1\equiv x_2\pmod{m_k}
    \end{align*}\\
    Because $m_i\mid x_1-x_2$ for all $i=1\ldots k$ and the $m_i$'s are relatively prime, $M\mid x_1-x_2$. Thus, the solution $\displaystyle x=\sum_{i=1}^{k}a_i y_i z_i$ is unique $\pmod{M}$.
    \item [\textbf{5.3.1}] $\begin{array}{c c c}
        & x & y\\
        21 & 1 & 0\\
        17 & 0 & 1\\
        4 & 1 & -1\\
        1 & -4 & 5
    \end{array}\\
    \Rightarrow 17\cdot 5-21\cdot 4=1$
    \item [\textbf{5.3.2}] $17\cdot 15-21\cdot 12=3$
    \item [\textbf{5.4.2}] Suppose $(x_1,y_1,k_1)$ and $(x_2,y_2,k_2)$ are solutions to $x^2-Ny^2=k$.\\ 
    We are given $k_1k_2=(x_1^2-Ny_1^2)(x_2^2-Ny_2^2)\\
    =(x_1-\sqrt{N}y_1)(x_1+\sqrt{N}y_1)(x_2-\sqrt{N}y_2)(x_2+\sqrt{N}y_2)$.\\
    \begin{align*}
        (x_1-\sqrt{N}y_1)(x_2-\sqrt{N}y_2)=x_1x_2-\sqrt{N}(x_1y_2+x_2y_1)+Ny_1y_2\\
        =(x_1x_2+Ny_1y_2)-\sqrt{N}(x_1y_2+x_2y_1)\\
        (x_1+\sqrt{N}y_1)(x_2+\sqrt{N}y_2)=x_1x_2+\sqrt{N}(x_1y_2+x_2y_1)+Ny_1y_2\\
        =(x_1x_2+Ny_1y_2)-\sqrt{N}(x_1y_2+x_2y_1)\\
        \Rightarrow k_1k_2=(x_1^2-Ny_1^2)(x_2^2-Ny_2^2)={(x_1x_2+Ny_1y_2)}^2-N{(x_1y_2+x_2y_1)}^2
    \end{align*}
    \item [\textbf{5.4.3}] A positive integer is a perfect square if and only if every prime in its factorization occurs an even number of times.
    Let $N$ be a nonsquare integer.
    By the Fundamental Theorem of Arithmetic, $N$ can be written as a product of primes. 
    It follows there exists some prime $p$ that occurs an odd number of times.
    Assume to the contrary that $\sqrt{N}=\frac{a}{b}$ is rational, where $a,b\in\mathbb{Z}$ are relatively prime and $b>0$.
    Squaring both sides and multiplying by $b^2$, we obtain $b^2N=a^2$.
    Because $b^2$ is a perfect square, $p$ occurs an even number of times in its prime factorization. 
    Thus, $p$ occurs an odd number of times in the prime factorization of $b^2N$.
    However, $p$ must occur an even number of times in the prime factorization of $a^2$ because $a^2$ is also a perfect square, so by the uniqueness of a number's prime factorization, we obtain a contradiction.
    Hence, $\sqrt{N}$ cannot be rational.\\
    Assume to the contrary $a_1-\sqrt{N}b_1=a_2-\sqrt{N}b_2$, but $a_1\neq a_2$ or $b_1\neq b_2$.
    Suppose WLOG $b_1\neq b_2$. It follows $\sqrt{N}=\frac{a_1-a_2}{b_1-b_2}$.
    However, $N$ is not a perfect square, so $\sqrt{N}$ is irrational.
    Because $a_1,a_2,b_1,b_2\in\mathbf{Z}\Rightarrow \frac{a_1-a_2}{b_1-b_2}\in\mathbb{Q}$ which is a contradiction, so $b_1=b_2$.
    Suppose $a_1\neq a_2$. 
    Since $b_1=b_2\Rightarrow \sqrt{N}b_1=\sqrt{N}b_2$. 
    It follows $a_1-\sqrt{N}b_1\neq a_2-\sqrt{N}b_2$ which is a contradiction, so $a_1=a_2$.
    \item [\textbf{5.4.4}] $(x_1-\sqrt{N}y_1)(x_2\sqrt{N}y_2)=(x_1x_2+Ny_1y_2)-\sqrt{N}(x_1y_2+x_2y_1)$. By \textbf{5.4.3} $(x_1x_2+Ny_1y_2)-\sqrt{N}(x_1y_2+x_2y_1)=x_3-\sqrt{N}y_3\Rightarrow$ $x_1x_2+Ny_1y_2=x_3$ and $x_1y_2+x_2y_1=y_3$.
    \item [\textbf{6.3.1}] Let $L$ be the line through rational points $p_1=(x_1,y_1)$ and $p_2=(x_2,y_2)$. 
    We can define $L$ by the equation $(y-y_1)(x_2-x_1)=(y_2-y_1)(x-x_1)$.
    Because $x_1,x_2,y_1,y_2\in\mathbb{Q}$, the addition, subtraction, multiplication, and nonzero division of $x_1,x_2,y_1,y_2$ are rational. 
    Moreover, the coefficients of $(x_2-x_1)y+(y_1-y_2)x=y_1x_2-y_2x_1$ are all rational.
    \item [\textbf{6.3.2}] Let the center of the circle be $c=(x_1,y_1)$ with point on the radius $r=(x_2,y_2)$.
    Define the equation for the circle, ${(x-x_1)}^2+{(y-y_1)}^2={(x_2-x_1)}^2+{(y_2-y_1)}^2$.
    Because $x_1,x_2,y_1,y_2\in\mathbb{Q}$, the addition, subtraction, multiplication, and nonzero division of $x_1,x_2,y_1,y_2$ are rational. 
    Thus, $x^2-2x_1x+x_1^2+y^2-2y_1y+y_1^2=x_2^2-2x_1x_2+x_1^2+y_2^2-2y_1y_2+y_1^2$ are all rational.
    \item [\textbf{6.3.3}] Define the lines $\ell_1: a_1x+b_1y=c_1, \ell_2: a_2x+b_2y=c_2$. Suppose they intersect at some point $(x^*,y^*)$.\\
    By elimination, we obtain $(a_2b_1-a_1b_2)y=a_2c_1-a_1c_2\Rightarrow y^*=\frac{a_2c_1-a_1c_2}{a_2b_1-a_1b_2}$. $\ell_1\parallel\ell_2$ if $a_1b_2=a_2b_1$. 
    Plugging in $y^*$ into one of the two equations, we can solve for $x^*=\frac{b_2c_1-b_1c_2}{a_1b_2-a_2b_1}$.
    \item [\textbf{6.3.4}] The case of the vertical line $x=c$ is a simpler case where the line and circle intersect at $(c,k+\sqrt{r^2-{(c-h)}^2}),(c,k-\sqrt{r^2-{(c-h)}^2})$.\\
    Suppose a line $y=mx+b$ and circle ${(x-h)}^2+{(y-k)}^2=r^2$ intersect at some point(s).\\
    Substituting $y$ with $mx+b$ we get ${(x-h)}^2+{(mx+b-k)}^2=r^2$.\\
    \begin{align*}
       \text{let }c=b-k\\
       x^2-2hx+h^2+m^2x^2+2cmx+c^2-r^2=0\\
       (m^2+1)x^2+(2cm-2h)x+h^2+c^2-r^2=0\\
       x^*=\frac{h-cm\pm\sqrt{-2cmh-m^2h^2-c^2+m^2r^2+r^2}}{m^2+1}\\
       x^*=\frac{h-(b-k)m\pm\sqrt{r^2(m^2+1)-{(b-k+mh)}^2}}{m^2+1}
    \end{align*}
    so we can find solutions for $x^*$ using only rational equations and square roots.
    Plugging the solutions for $x^*$ into $y=mx+b$ we can find the corresponding $y^*$ values.
    \item [\textbf{6.4.1}] Assume to the contrary $\sqrt[3]{2}=\frac{a}{b}$ is rational where $a$ and $b$ are coprime. 
    Cubing both sides and multiplying by $b^3$ we obtain $2b^3=a^3$. 
    It follows $2\mid a^3$ and by the FTA $2\mid a$. Thus, there exists some integer $a'$ s.t $a=2a'$.
    This implies $b^3=4{a'}^3\Rightarrow 2\mid b^3$, and once again, by the FTA $2\mid b$.
    However, $2\mid a$ and $2\mid b$, so we obtain a contradiction because we originally stated $a$ and $b$ are coprime.
    \item [\textbf{6.4.2}] Proof by induction\\
    Base case: The set of rational numbers is trivially a field.\\
    Induction hypothesis: Assume for some $k$ that $F_k$ is a field.\\
    Induction step: Let $x=a_1+b_1\sqrt{{c_k}_1},y=a_2+b_2\sqrt{c_k}$\\
    $x+y\in F_{k+1}$ because $a_1+a_2,b_1+b_2,c_k\in F_k$\\
    $x-y\in F_{k+1}$ because $a_1-a_2,b_1-b_2,c_k\in F_k$\\
    $xy\in F_{k+1}$ because $a_1a_2+b_1b_2c_k,a_1b_2+a_2b_1,c_k\in F_k$\\
    $\frac{x}{y}\in F_{k+1}$ because $\frac{a_1a_2-b_1b_2c_k}{a_2^2-b_2^2c_k},\frac{a_2b_1-a_1b_2}{a_2^2-b_2^2c_k},c_k\in F_k$\\
    Hence, by induction, the claim holds for all $k$.
\end{enumerate}
\end{document}