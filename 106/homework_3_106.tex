\documentclass[10pt]{article}
\usepackage{graphicx}
\usepackage{amssymb}
\usepackage[fleqn]{amsmath}
\usepackage{nccmath}
\usepackage{cases}
\usepackage{hyperref}
\usepackage{multicol}
\usepackage{tikz}
\usepackage{tikz}
\usetikzlibrary{shapes.geometric}
\usepackage{pgfplots}
\usepackage{enumitem}
\pgfplotsset{compat=1.18}
\usepackage{float}

\title{\bf Math 106: Problem Set 3}
\date{2/4/2024}
\author{\bf Owen Jones}
\begin{document}
\maketitle
\begin{enumerate}
    \item [\textbf{3.2.1}] Let $s$ be a square number.
    It follows there exists some integer $q$ s.t $s=q^2$.
    \begin{itemize}
        \item [$q\equiv 0\pmod{4}$] There exists some integer $k$ s.t $q=4k$. It follows $s=16k^2$. $8\mid 16k^2\Rightarrow s\equiv 0\pmod{8}$.
        \item [$q\equiv 2\pmod{4}$] There exists some integer $k$ s.t $q=2+4k$. It follows $s=16k^2+16k+4$. $8\mid 16k^2+16k\Rightarrow s\equiv 4\pmod{8}$.
        \item [$q$ is odd] There exists some integer $k$ s.t $q=\pm1+4k$. It follows $s=16k^2\pm8k+1$. $8\mid 16k^2+16k\Rightarrow s\equiv 1\pmod{8}$.
    \end{itemize}
    \item [\textbf{3.2.2}] From \textbf{3.2.1}, we know any square leaves remainder $0,1$, or $4$ on division by $8$. 
    Thus, if we take $3$ square numbers, their remainders added together $\pmod{8}$ will be some value. Some quick examples:\\
    $0+0+0=0,
    0+0+1=1,
    0+1+1=2,
    1+1+1=3,
    4+0+0=4,
    4+1+0=5,
    4+1+1=6$\\
    It remains to show it is impossible for the sum of $3$ squares to leave $7$ on division by $8$.\\
    Since $7$ is odd, we must have either $1$ or $3$ odd squares. 
    $3$ odd squares added together leaves remainder $3$ on division by $8$, so we must have $1$ odd square.
    Since the other two squares are even, they must leave remainder $0$ or $4$ on division by $8$.
    Thus, the sum of the even squares leave remainder $0$ or $4$ on division by $8$\\
    Since neither $1$ nor $5$ is $7$, it is impossible to leave remainder $7$ on division by $8$.
    \item [\textbf{3.2.3}] Let $x_k$ by the $k^{th}$ pentagonal number. From figure $3.1$, we deduce $x_{k+1}=x_k+3k+1$.\\
    Pf by induction:\\
    Base case: $\frac{3\cdot 1^2-1}{2}=1$ which is the $1^{st}$ pentagonal number.\\
    Induction hypothesis: Assume for some $k\ge 1$ $x_k=\frac{3k^2-k}{2}$.\\
    Induction step: $x_{k+1}=\frac{3k^2-k}{2}+3k+1=\frac{3k^2+5k+2}{2}=\frac{3{(k+1)}^2-(k+1)}{2}$.\\
    Hence, by induction, the claim holds for all $k$.
    \item [\textbf{3.2.4}] Let $t_k$ be the $k^{th}$ triangular number. 
    Thus $t_k=\sum_{i=0}^{k}i=\frac{k(k+1)}{2}$.
    We show $k^2=t_{k-1}+t_k$.\\
    $t_{k-1}+t_k=\frac{(k-1)k}{2}+\frac{k(k+1)}{2}=\frac{2k^2}{2}=k^2$
    \item [\textbf{3.3.1}] Let $q$ be a prime divisor of $2^{n-1}p$.\\ 
    Thus, either $\begin{cases}
        q=p & \text{if } q\mid p\\
        q=2 & \text{if } q\mid 2^{n-1}
    \end{cases}$ 
    If $q=p$, then we can iterate through $2^{n-1}$ using the prime divisor property to show $1,2,2^2,\cdots 2^{n-1}$ are all proper divisors of $2^{n-1}p$.
    If $q=2$, then we can iterate through $2^{n-2}p$ using the prime divisor property.\\
    Thus, we obtain $\begin{cases}
        q=p \text{ and proceed to }2^{n-2}\text{ case} & \text{if } q\mid p\\
        q=2 & \text{if } q\mid 2^{n-2}
    \end{cases}$
    We can iterate through this case to show $p,2p,2^2p,\cdots 2^{n-2}p$ are all proper divisors of $2^{n-1}p$.\\
    Thus, we only need to show that there are no other proper divisors of $2^{n-1}p$. 
    Every number has a unique prime factorization. 
    It follows that any proper divisor of $2^{n-1}p$ must must constructed from $2$s and $p$.
    Moreover, any number $2^j p^k$ where $j>n-1$ or $p>1$ can't be a divisor because $2^{j-n+1}p^{k-1}\nmid 1$.
    \item [\textbf{3.3.2}] If we divide $a$ by $b$, we obtain a quotient $q_1$ and remainder $r_2$.
    It follows we can write $r_2$ as a linear combination of $a$ and $b$ i.e $r_2=a-q_1b$.
    % If we divide $b$ by $r_2$, we obtain a quotient $q_2$ and remainder $r_3$.
    % Once again, we can write $r_3$ as a linear combination of $b$ and $r_2$ i.e $r_3=b-q_2r_2$.
    % Moreover, we can write $r_3$ as a linear combination of $a$ and $b$ i.e $r_3=b-q_2(a-q_1b)=(1+q_1q_2)b-q_2a$.
    Assume for some $i$, we can write $r_i$ and $r_{i+1}$ as a linear combination of $a$ and $b$.
    We have $r_{i+2}=r_i-q_{i+1}r_i$ by division with remainder. 
    Thus, $r_{i+2}=(am_i+bn_i)-q_{i+1}(am_{i+1}+bn_{i+1})=a(m_i-q_{i+1}m_{i+1})+b(n_i-q_{i+1}n_{i+1})$ which is a linear combination of $a$ and $b$.
    Hence, $m_{i+2}=m_i-q_{i+1}m_{i+1}$ and $n_{i+2}=n_i-q_{i+1}n_{i+1}$ where $m_0=1,m_1=0,n_0=0,n_1=1$. 
    When we terminate the Euclidean Algorithm after some $k$ steps, we obtain $am_k+bn_k=\gcd(a,b)$.
    \item [\textbf{3.3.3}] If $\gcd(a,b)\mid c$ there exists an integer $k$ s.t $\gcd(a,b)k=c$. 
    It follows from \textbf{3.3.2} there exists $m,n$ s.t $am+bn=\gcd(a,b)$. 
    Thus, $a(mk)+b(nk)=c$ has an integer solution because $mk,nk$ are both integers.
    Suppose $am+bn=c$ has an integer solution. 
    $\gcd(a,b)\mid a$ and $\gcd(a,b)\mid b$, so $\gcd(a,b)\mid am+bn$. 
    Thus, $\gcd(a,b)\mid c$ must also be true.
    \item [\textbf{3.3.4}] $\gcd(12,15)=3$. Thus, by \textbf{3.3.3}, if there exists a solution to $12x+15y=1$ then $3\mid 1$. This is clearly false, so $12x+15y=1$ has no integer solutions.
\end{enumerate}
\end{document}