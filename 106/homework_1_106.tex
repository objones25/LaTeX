\documentclass[10pt]{article}
\usepackage{graphicx}
\usepackage{amssymb}
\usepackage[fleqn]{amsmath}
\usepackage{nccmath}
\usepackage{cases}
\usepackage{hyperref}
\usepackage{multicol}
\usepackage{tikz}
\usepackage{pgfplots}
\usepackage{enumitem}
\pgfplotsset{compat=1.18}
\usepackage{float}

\title{\bf Math 106: Problem Set 1}
\date{1/21/2024}
\author{\bf Owen Jones}
\begin{document}
\maketitle
\begin{enumerate}
    \item [\bf{1.2.3}] WLOG let $m\in\mathbb{Z}$. 
    If $m$ is even, then $m\equiv 0\mod(2)\Rightarrow m=2k$ for some $k\in\mathbb{Z}$.
    Otherwise, $m$ is odd, so $m\equiv 1\mod(2)\Rightarrow m=2k+1$ for some $k\in\mathbb{Z}$.
    Thus, because $m$ is arbitrary, $m^2$ can be any perfect square.
    Either $m^2={(2k+1)}^2=4k^2+4k+1\equiv 1\mod(4)$ because $4k^2$ and $4k$ are clearly divisible by $4$, or $m^2=4k^2\equiv 0\mod(4)$ because $4k^2$ is clearly divisible by $4$. 
    Hence, every perfect square leaves remainder $0$ or $1$ on division by $4$.
    \item [\bf{1.2.4}] WLOG let $a$ be odd. 
    Suppose for the sake of contradiction $b$ is also odd. 
    Since $a^2$ and $b^2$ are odd, then $c^2$ must be even. 
    It follows $c^2\equiv 0\mod(4)$ and $b^2\equiv 1\mod(4)$ by $\mathbf{1.2.3}$.
    Thus, $c^2-b^2\equiv 3\mod(4)$.
    However, as we showed in $\mathbf{1.2.3}$, $a^2\equiv 1\mod(4)$, so we obtain a contradiction.
    Thus, $b$ cannot be odd.
    Hence, both $a$ and $b$ cannot both be odd.
    \item [\bf{1.3.1}] If $(a,b,c)$ is a pythagorean triple, then $\frac{a}{c}=\frac{1-t^2}{1+t^2},\frac{b}{c}=\frac{2t}{1+t^2}$ where $t=\frac{q}{p}$ for some integers $p,q$. 
    Substituting $\frac{q}{p}$ we obtain $\frac{a}{c}=\frac{1-\frac{q^2}{p^2}}{1+\frac{q^2}{p^2}},\frac{b}{c}=\frac{2\frac{q}{p}}{1+\frac{q^2}{p^2}}$. 
    Multiplying by $\frac{p^2}{p^2}$, $\frac{a}{c}=\frac{p^2-q^2}{p^2+q^2},\frac{b}{c}=\frac{2pq}{p^2+q^2}$.
    \item [\bf{1.3.2}] If $\frac{a}{c}=\frac{p^2-q^2}{p^2+q^2},\frac{b}{c}=\frac{2pq}{p^2+q^2}$ from $\mathbf{1.3.1}$, then if we set $c:=r(p^2+q^2)$ then $a=r(p^2-q^2)$ and $b=2rpq$
    \item [\bf{1.3.4}] $\cos(\theta)=\frac{x}{1}=\frac{1-t^2}{1+t^2},\sin(\theta)=\frac{y}{1}=\frac{2t}{1+t^2}$ by the solution pair $(\frac{1-t^2}{1+t^2},\frac{2t}{1+t^2})$. $\Rightarrow \tan\frac{\theta}{2}=\frac{\sin\theta}{1+\cos\theta}=\frac{\frac{2t}{1+t^2}}{\frac{1+t^2}{1+t^2}+\frac{1-t^2}{1+t^2}}=\frac{\frac{2t}{1+t^2}}{\frac{2}{1+t^2}}=t$
    \item [\bf{1.4.2}] Will do later
    \item [\bf{1.5.1}] For some arbitrary odd integer $m$, there exists some integer $q$ such that $m=2q+1$. 
    Squaring both sides we obtain $m^2={(2q+1)}^2=4q^2+4q+1=2r+1$ where $r=2q^2+2q$. 
    Since $m^2$ can be written in the form $m^2=2r+1$ for some integer $r$, $m^2$ is also odd.
    \item [\bf{1.5.2}] Squaring $2q+1$ we obtain $4q^2+4q+1=4s+1$ where $s=q^2+q$. 
    It follows $4\mid ({(2q+1)}^2-1)\Rightarrow {(2q+1)}^2\equiv 1\pmod{4}$. 
    Any odd integer $m\equiv 1\pmod4$ or $m\equiv 3\pmod4$. 
    Any even integer $m\equiv 0\pmod4$ or $m\equiv 2\pmod4$.
    Thus, $m^2\equiv1^2\pmod4,3^2\pmod4\Rightarrow m^2\equiv 1\pmod4$ if $m$ is odd, and $m^2\equiv0^2\pmod4,2^2\pmod4\Rightarrow m^2\equiv 0\pmod4$ if $m$ is even.
    \item [\bf{1.6.1}] Let $x_1,x_2$ be real numbers with the same sign or $0$, and let $y_1,y_2$ be real numbers that satisfy $x_1y_2=x_2y_1$. 
    WLOG by translation let $A:=(-x_1,-y_1),B:=(0,0),C:=(x_2,y_2)$.
    Let $t\in[0,1]$ and $((x_2+x_1)t-x_1,(y_2+y_1)t-y_1)$ be the set of points between $A$ and $C$.
    If $x_1=0$ or $x_2=0$ then either $A$ or $C$ must be at the origin or both $A$ and $C$ must be on the line $x=0$.
    In either case, $A,B,C$ are colinear and $AB+BC=AC\Leftrightarrow \sqrt{x_1^2+y_1^2}+\sqrt{x_2^2+y_2^2}=\sqrt{{(x_1+x_2)}^2+{(y_1+y_2)}^2}$.
    Otherwise, set $t=\frac{x_1}{x_1+x_2}$.\\ $(y_2+y_1)t-y_1=\frac{(y_2+y_1)x_1}{x_1+x_2}-\frac{y_1(x_1+x_2)}{x_1+x_2}=\frac{y_2x_1-y_1x_2}{x_1+x_2}=0$.
    Thus, $A,B,C$ are all colinear.
    Moreover, $AB+BC=AC\Leftrightarrow \sqrt{x_1^2+y_1^2}+\sqrt{x_2^2+y_2^2}=\sqrt{{(x_1+x_2)}^2+{(y_1+y_2)}^2}$.
    \item [\bf{1.6.2}] Let $L=\sqrt{x_1^2+y_1^2}+\sqrt{x_2^2+y_2^2}\\
    \Rightarrow x_1^2+y_1^2=L^2-2L\sqrt{x_2^2+y_2^2}+x_2^2+y_2^2\\
    \Rightarrow 2L\sqrt{x_2^2+y_2^2}=L^2+x_2^2+y_2^2-x_1^2-y_1^2\\
    \Rightarrow 4L^2(x_2^2+y_2^2)=L^4+2L^2(x_2^2+y_2^2-x_1^2-y_1^2)+{(x_2^2+y_2^2-x_1^2-y_1^2)}^2\\
    \Rightarrow 0=L^4-2L^2(x_1^2+x_2^2+y_1^2+y_2^2)+{(x_2^2+y_2^2-x_1^2-y_1^2)}^2$
\end{enumerate}
\end{document}