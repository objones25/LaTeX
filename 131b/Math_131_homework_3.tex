\documentclass[10pt]{article}
\usepackage{amsmath,amssymb}
\setlength{\oddsidemargin}{0in}
\setlength{\evensidemargin}{0in}
\setlength{\textheight}{9in}
\setlength{\textwidth}{6.5in}
\setlength{\topmargin}{-0.5in}
\usepackage{enumitem}
\usepackage{graphicx}

\title{\bf Math 131B: Homework 3}
\date{4/28/2023}
\author{\bf Owen Jones}

\begin{document}
\maketitle
\begin{enumerate}[label=Problem \arabic*.]
    \item \textbf{Exercise 1.5.4}\par 
    Let $f:\mathbb{R}\rightarrow\mathbb{R}$ $f(x)=0$. $\mathbb{R}$ is open, but $\{0\}$ is not open because $0$ is a boundary point.\par 
    QED
    \item \textbf{Exercise 1.5.5}\par 
    Let $f:[1,\infty)\rightarrow(0,1]$ $f(x)=\frac{1}{x}$. $[1,\infty)$ is closed because it contains all of its boundary points, $1$, but $(0,1]$ is not closed because $0\notin(0,1]$.\par 
    QED
    
    \item \textbf{Exercise 1.5.10 (b)}\par 
    First we show that if $(X,d)$ is a compact metric space, then $(X,d)$ is complete. 
    Let $(x^{(n)})_{n=m}^{\infty}$ be an arbitrary Cauchy sequence in $(X,d)$. Since $(X,d)$ is compact, there exists a subsequence $(x^{(n_j)})_{j=1}^{\infty}$ that converges to some value $x_0\in X$. 
    It follows by Lemma $1.4.9$ that $(x^{(n)})_{n=m}^{\infty}$ also converges to $x_0$. 
    Because any arbitrary Cauchy sequence in $(X,d)$ is convergent in $(X,d)$, $(X,d)$ is a complete metric space.\par 
    Next we show that $(X,d)$ must also be totally bounded. 
    Assume for the sake of contradiction that $(X,d)$ is compact but not totally bounded. 
    Therefore, $\exists\epsilon>0$ s.t no finite number of balls of radius-$\epsilon$ will cover $X$.
    Because $X$ requires infinitely many $\epsilon$-balls to be covered entirely, $X\setminus(\displaystyle{\bigcup_{i=1}^{n}}B(x^{(i)},\epsilon))\neq \emptyset$ for all $n$. 
    Thus, we can construct a sequence $(x^{(n)})_{n=1}^{\infty}$ in $(X,d)$ where $x^{(n)} \notin \displaystyle{\bigcup_{i=1}^{n-1}}B(x^{(i)},\epsilon)$.
    Since each term of the sequence is at least distance-$\epsilon$ from every other term of the sequence, $(x^{(n)})_{n=1}^{\infty}$ has no convergent subsequences. 
    We obtain a contradiction because $(x^{(n)})_{n=1}^{\infty}$ is a sequence in $(X,d)$ and $(X,d)$ is compact, so $(x^{(n)})_{n=1}^{\infty}$ must have a convergent subsequence.
    Hence, $(X,d)$ must be totally bounded.\par 
    QED
    \item \textbf{Exercise 2.1.1}
    We will show a, b, and c are logically equivalent by showing $a\Rightarrow b$, $b \Rightarrow a$, $a\Rightarrow c$, and $c\Rightarrow a$.
    \begin{itemize}
        \item [$(a\Rightarrow b)$] Suppose $f$ is continuous at $x_0$, and let $(x^{(n)})_{n=1}^{\infty}$ be a sequence in $X$ that converges to $x_0$. 
        Because $f$ is continuous at $x_0$, there exists $\delta>0$ for every $\epsilon>0$ s.t $d_Y(f(x^{(n)}),f(x_0))<\epsilon$ whenever $d_X(x^{(n)},x_0)<\delta$.
        Given $\epsilon>0$ choose $N$ to be sufficiently large s.t $n\ge N\Rightarrow d_X(x^{(n)},x_0)<\delta$.
        Thus, $d_Y(f(x^{(n)}),f(x_0))<\epsilon$ by the continuity of $f$.
        Hence, $(f(x^{(n)}))_{n=1}^{\infty}$ converges to $f(x_0)$.
       
        \item [$(b\Rightarrow a)$] Suppose the sequence $(f(x^{(n)}))_{n=1}^{\infty}$ converges to $f(x_0)$ with respect to the metric $d_Y$ whenever a sequence $(x^{(n)})_{n=1}^{\infty}$ in $X$ converges to $x_0$ with respect to the metric $d_X$, and assume for the sake of contradiction that $f$ is not continuous at $x_0$.
        Because $f$ is not continuous at $x_0$, $\exists\epsilon>0$ s.t $\forall\delta>0$ $d(x,x_0)<\delta$ yet $d(f(x),f(x_0))\ge\epsilon$.
        Let $(x^{(n)})_{n=1}^{\infty}$ be a sequence in $X$ s.t $d_X(x^{(n)},x_0)<\frac{1}{n}$ while $d_Y(f(x^{(n)}),f(x_0))\ge\epsilon$.
        We obtain a contradiction because we have a convergent sequence $(x^{(n)})_{n=1}^{\infty}$ where $(f(x^{(n)}))_{n=1}^{\infty}$ doesn't converge.
        Hence, $f$ must be continuous at $x_0$.
       
        \item [$(a\Rightarrow c)$] Suppose $f$ is continuous at $x_0$, and let $V\subset Y$ be an open set that contains $f(x_0)$. 
        Because $f$ is continuous at $x_0$, there exists $\delta>0$ for every $\epsilon>0$ s.t $d_Y(f(x),f(x_0))<\epsilon$ whenever $d_X(x,x_0)<\delta$. 
        By Proposition 1.2.15 (a), $\exists r_y>0$ s.t $B_{(Y,d_Y)}(f(x_0),r_y)\subseteq V$.
        It follows $\exists r_x>0$ s.t $f(x)\in B_{(Y,d_Y)}(f(x_0),r_y)$ whenever $x\in U=B_{(X,d_X)}(x_0,r_x)$.
        Hence, we have $U\subseteq X$ s.t $f(U)\subseteq V$.

        \item [$(c\Rightarrow a)$] Suppose for every open set $V\subset Y$ that contains $f(x_0)$, there exists an open set $U\subset X$ containing $x_0$ s.t $f(U)\subseteq V$, and assume for the sake of contradiction that $f$ is not continuous at $x_0$.
        Because $f$ is not continuous at $x_0$, $\exists\epsilon>0$ s.t $\forall\delta>0$ $d(x,x_0)<\delta$ yet $d(f(x),f(x_0))\ge\epsilon$.
        Let $V=B_{(Y,d_Y)}(y_0,\epsilon)$. It follows there exists an open set $U\subset X$ s.t $f(U)\subseteq V$.
        By Proposition 1.2.15 (a), $\exists r>0$ s.t $B_{(X,d_X)}(x_0,r)\subseteq U$. 
        Because $f$ is not continuous at $x_0$, there exists $x\in B_{(X,d_X)}(x_0,r)$ s.t $f(x)\notin B_{(Y,d_Y)}(y_0,\epsilon)$ which implies $f(U)\not\subset V$. Hence, we obtain a contradiction, so $f$ must be continuous at $x_0$.
    \end{itemize}
    Thus, by transitivity, a,b, and c are logically equivalent.

    \item \textbf{Exercise 2.1.4}
    \begin{itemize}
        \item [a)] $f: \mathbb{R}\rightarrow\mathbb{R}$ $f(x)= \begin{cases}
           x & \text{if } x\in \mathbb{Q}\\
           x+1 & \text{if } x\in \mathbb{R}\setminus\mathbb{Q}
       \end{cases}$, 
        $g: \mathbb{R}\rightarrow\mathbb{R}$ $g(x)=0$, $g\circ f(x)=0$
       \item [b)] $f: \mathbb{R}\rightarrow\mathbb{R}$ $f(x)=0$, $g: \mathbb{R}\rightarrow\mathbb{R}$ $g(x)= \begin{cases}
        x & \text{if } x\in \mathbb{Q}\\
        x+1 & \text{if } x\in \mathbb{R}\setminus\mathbb{Q}
    \end{cases}$, $g\circ f(x)=0$
        \item [c)] $f: \mathbb{R}\rightarrow\mathbb{R}$ $f(x)= \begin{cases}
            x & \text{if } x\in \mathbb{Q}\\
            x+1 & \text{if } x\in \mathbb{R}\setminus\mathbb{Q}
        \end{cases}$, $g: \mathbb{R}\rightarrow\mathbb{R}$ $g(x)= \begin{cases}
            x+1 & \text{if } x\in \mathbb{Q}\\
            x & \text{if } x\in \mathbb{R}\setminus\mathbb{Q}
        \end{cases}$, $g\circ f(x)=x+1$
    \end{itemize}\par 
    Corrolary 2.1.7 is not an iff statement. Neither a) nor b) discuss when either $f(x)$ or $g(x)$ are discontinuous. Each of parts a),b), and c) of Exercise 2.1.7 include at least 1 discontinuous function for $f(x)$ or $g(x)$.
\end{enumerate}
\end{document}