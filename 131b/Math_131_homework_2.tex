\documentclass[10pt]{article}
\usepackage{amsmath,amssymb}
\setlength{\oddsidemargin}{0in}
\setlength{\evensidemargin}{0in}
\setlength{\textheight}{9in}
\setlength{\textwidth}{6.5in}
\setlength{\topmargin}{-0.5in}
\usepackage{enumitem}
\usepackage{graphicx}

\title{\bf Math 131B: Homework 2}
\date{4/21/2023}
\author{\bf Owen Jones}

\begin{document}
\maketitle

\begin{enumerate}[label=Problem \arabic*.]
    \item \textbf{Exercise 1.3.1} \par
    First, suppose $E$ is relatively closed with respect to $Y$. Then $Y\setminus E$ is relatively open with respect to $Y$. By Proposition 1.3.4 a), $\exists V\subseteq X$ which is open in $X$ s.t $Y\cap V=Y\setminus E$. Let $K=X\setminus V$ which is closed in $X$ because $V$ is open in $X$.
    It follows $Y\cap K=Y\cap (X\setminus V)$. \textit{note: If $x\in Y\cap (X\setminus V)$, then $x\in Y$ and $x\in X\setminus V$. Because $x\in Y$ and $Y\subseteq X, x\in Y\setminus V$. Thus, $Y\cap (X\setminus V)\subseteq Y\setminus V$. 
    If $x\in Y\setminus V$ then $x\in Y$ and $x \notin V$. Because $Y\subseteq X$ and $x\notin V$, $x\in X\setminus V$. Because $x\in Y$ and $x\in X\setminus V$, $x\in Y\cap (X\setminus V)$. Thus, $Y\setminus V\subseteq Y\cap (X\setminus V)$. Because the two sets are subsets of each other, they must be equal.}
    Because $Y\setminus V=Y\cap (X\setminus V)$ and $K=(X\setminus V)$, it follows $Y\cap K=Y\setminus V$, which is equivalent to $Y\setminus (Y\cap V)$ (set minus operator is the subset of $Y$ that doesn't have any shared elements with $V$). Because $Y\cap V=Y\setminus E$, it follows $Y\setminus (Y\cap V)=Y\setminus (Y\setminus E)$. Because $E\subseteq Y$, $Y\setminus (Y\setminus E)=E$.
    Thus, if $E$ is relatively closed with respect to $Y$, $\exists K\subseteq X$ which is closed in $X$ s.t $Y\cap K=E$.\par 
    Next, suppose $E=Y\cap K$ for some $K\subseteq X$ which is closed in $X$. Then, $X\setminus K$ is open in $X$. In the previous part of the proof, we noted that $Y\cap (X\setminus K)=Y\setminus (Y\cap K)$, and because $(X\setminus K)$ is open, $Y\setminus (Y\cap K)$ is reletively open with respect to $Y$ by proposition 1.3.4 a). As we've shown previously $Y\setminus (Y\cap K)=Y\setminus E$, so $E$ must be relatively closed with respect to $Y$.\par
    Hence, $E$ is relatively closed with respect to $Y$ if and only if $E=K\cap Y$ for some set $K\subseteq X$ which is closed in $X$.\par 
    QED
    \item \textbf{Exercise 1.4.4} \par
    Let $\epsilon>0$. Because the sequence $(x^{(n)})^{\infty}_{n=m}$ is a Cauchy sequence in $(X,d)$, there exists an $N\ge m$ such that $d(x^{(j)},x^{(k)})<\frac{\epsilon}{2}$ for all $j,k\ge N$. 
    Because the subsequence $(x^{(n_i)})^{\infty}_{i=1}$ converges to $x_0$, there exists $n_l\ge N$ where $l\ge 1$ s.t $d(x^{(n_l)},x_0)<\frac{\epsilon}{2}$. 
    It follows that because $n_l\ge N$, $d(x^{(n_l)},x^{(k)})<\frac{\epsilon}{2}$ for any $k\ge N$. 
    By the triangle innequality, if $n\ge N$, then $d(x^{(n)},x_0)\le d(x^{(n)},x^{(n_l)})+d(x_0,x^{(n_l)})< 2\cdot \frac{\epsilon}{2}=\epsilon$. 
    Hence, the sequence $(x^{(n)})^{\infty}_{n=m}$ converges to $x_0$.\par 
    QED
    \item \textbf{Exercise 1.4.7} \par
    \begin{itemize}
        \item [a)] Let $x_0\in \overline{Y}$. 
        It follows there exists a sequence $(x^{(n)})_{n=m}^{\infty}$ in $Y$ s.t $\displaystyle{\lim_{n\rightarrow \infty}}d_{|YxY}(x^{(n)},x_0)=0$. 
        Because $(x^{(n)})_{n=m}^{\infty}$ is convergent, it must be Cauchy.
        \textit{note: To prove Lemma 1.4.7, suppose $(x^{(n)})_{n=m}^{\infty}$ converges to $x_0$. 
        For $\epsilon>0$, choose $N$ large enough s.t for any $j,k\ge N$, $d(x^{(j)},x_0)<\frac{\epsilon}{2}$ and $d(x^{(k)},x_0)<\frac{\epsilon}{2}$. 
        By the triangle innequality, $d(x^{(k)},x^{(j)})\le d(x^{(k)},x_0)+d(x^{(j)},x_0)<2\cdot \frac{\epsilon}{2}=\epsilon$.
        Hence, $(x^{(n)})_{n=m}^{\infty}$ is Cauchy.} 
        Because $Y$ is complete, any Cauchy sequence in $Y$ must be convergent in $Y$. 
        Thus, $x_0\in Y\Rightarrow \overline{Y}\subseteq Y$.
        Because $Y\subseteq X\Rightarrow Y\subseteq \overline{Y}$. 
        Thus, $Y=\overline{Y}$, so $Y$ is closed in $X$.\par 
        QED
        \item [b)] Let $(x^{(n)})_{n=m}^{\infty}$ be a Cauchy sequence in $Y$. 
        Because $(X,d)$ is a complete metric space and $Y\subseteq X$, the Cauchy sequence $(x^{(n)})_{n=m}^{\infty}$ must converge to some point we will call $x_0$ in $X$. 
        Since there exists a sequence $(x^{(n)})_{n=m}^{\infty}$ in $Y$ that converges to $x_0$, $x_0$ must be adherent to $Y$. 
        Because $Y$ is closed in $X$, $x_0$ is an element of $Y$. 
        Since $(x^{(n)})_{n=m}^{\infty}$ is an arbitrary Cauchy sequence in $Y$ that is convergent in $Y$, the subspace $(Y,d|YxY)$ is complete.\par
        QED        
    \end{itemize}
    \item \textbf{Exercise 1.5.12} \par
    \begin{itemize}
        \item [a)] Let $(x^{(n)})_{n=m}^{\infty}$ be a Cauchy sequence in $X$. 
        Thus, there exists $N\ge m$ s.t for all $j,k\ge N$, $d_{disc}(x^{(k)},x^{(j)})<1$. 
        Because $d_{disc}(x^{(k)},x^{(j)})<1$, $x^{(k)}$ must be equal to $x^{(j)}$. 
        Let $\epsilon>0$. If $n\ge N$, $d_{disc}(x^{(n)},x^{(N)})=0<\epsilon$. 
        Because $x^{(N)}$ is an element of $X$, $(x^{(n)})_{n=m}^{\infty}$ is a convergent sequence in $X$.\par 
        QED
        \item [b)] $X$ is compact iff $X$ is finite. \par 
        $(\Rightarrow)$ $X$ is compact if any open cover $X\subseteq\displaystyle{\bigcup_{\alpha\in I}}V_\alpha$ can be reduced to a finite collection of subsets that still covers $X$.
        Suppose $\displaystyle{\bigcup_{\alpha\in I}}V_\alpha$ is a collection of open balls of radius less than 1. Thus, $\displaystyle{\bigcup_{\alpha\in I}}V_\alpha$ is a collection of singleton subsets.
        It follows $\displaystyle{\bigcup_{\alpha\in I}}V_\alpha$ can only be reduced to a finite $F\subseteq I$ if there are finitely many $x\in X$.
        If there are infinitely many $x\in X$, you'd need infinitely many singleton subsets to cover $X$.\par 
        $(\Leftarrow)$ Suppose $X$ is finite. If $X\subseteq\displaystyle{\bigcup_{\alpha\in I}}V_\alpha$ then for each $x\in X$ $\exists\alpha\in I$ s.t $x\in V_\alpha$.
        Labeling each element of $X$ $\{x_1,x_2,...,x_n\}$ with corresponding $\{\alpha_1,\alpha_2,...,\alpha_n\}$, we can create a finite $F\subseteq I$ where $F=\{\alpha_1,\alpha_2,...,\alpha_n\}$ and $\displaystyle{\bigcup_{\alpha\in F}}V_\alpha$ still covers $X$.
        Thus, $X$ is compact.
        QED\par 
       
    \end{itemize}
    \item \textbf{Exercise 1.5.15} \par
    Assume to the contrary $\displaystyle{\cap_{\alpha\in I}K_{\alpha}}=\emptyset$. 
    Because $(K_\alpha)\alpha\in I$ is a collection of closed sets in $X$, $(K^{c}_{\alpha})\alpha\in I$ is a collection of open sets in $X$. 
    Because $\displaystyle{\cap_{\alpha\in I}K_{\alpha}}=\emptyset$, it follows that $\displaystyle{\cup_{\alpha\in I}K^{c}_{\alpha}}=X$. 
    Thus, $\displaystyle{\cup_{\alpha\in I}K^{c}_{\alpha}}$ is an open cover for $X$. 
    By the compactness of $X$, there exists a finite subset $F\subseteq I$ s.t $\displaystyle{\cup_{\alpha\in F}K^{c}_{\alpha}}=X$.
    Therefore $\displaystyle{\cap_{\alpha\in F}K_{\alpha}}=\emptyset$ which is a contradiction by the property that any finite subcollection of $I$ has a non-empty intersection.
    Thus, $\displaystyle{\cap_{\alpha\in I}K_{\alpha}}\neq\emptyset$.\par 
    QED
    
    \item \textbf{Additional problem} \par 
    \begin{itemize}
        \item [a)] Let $(x^{(n)})_{n=1}^{\infty}$ denote the sequence $x^{(n)}=\frac{1}{n}$ in $E$. 
        Let $k\ge 1$ be arbitrary and $\epsilon>0$. 
        If $\epsilon$ is small enough s.t $\epsilon<\frac{1}{k}$, there exists $N>\frac{k}{1-\epsilon k}$ s.t if $n\ge N$, $d(x^{(n)},\frac{1}{k})>\frac{1}{k}-\frac{1-\epsilon k}{k}=\epsilon$. 
        Thus, for any $k$ there exists $\epsilon$ sufficiently small s.t the set $\{x^{(n)}:d(x^{(n)},\frac{1}{k})<\epsilon\}$ is finite. 
        Hence, $(x^{(n)})_{n=1}^{\infty}$ doesn't have a subsequential limit. \par 
        QED
        \item [b)] Let $(x^{(n)})_{n=m}^{\infty}$ be a sequence in $E\cup \{0\}$. If $(x^{(n)})_{n=m}^{\infty}$ converges to $0$, then $(x^{(n)})_{n=m}^{\infty}$ has a subsequential limit. If not, for some $\epsilon>0$, there exists $n\ge N$ for every $N\ge m$ s.t $d(x^{(n)},0)\ge \epsilon$.
        It follows there exist infinitely many $n$ where $d(x^{(n)},0)\ge \epsilon$. 
        However there exist only finitely many $k\in E\cup\{0\}$ s.t $\epsilon\le k$, so for all but finitely many points in $E\cup\{0\}$ are contained in $B(0,\epsilon)$. 
        Thus, we are distributing infinitely many terms of the sequence $(x^{(n)})_{n=m}^{\infty}$ among finitely many points in ($E\cup\{0\})\setminus B(0,\epsilon)$.
        By the pigeonhole principle, there exists at least one $k\in (E\cup\{0\})\setminus B(0,\epsilon)$ that is a subsequential limit.
        Moreover, for each $\epsilon>0$ and $N\ge m$, there exists $n\ge N$ s.t $d(x^{(n)},k)<\epsilon$.
        Hence, $(x^{(n)})_{n=m}^{\infty}$ must have a subsequential limit.
        \par QED
    \end{itemize}

\end{enumerate}


\end{document}