\documentclass[10pt]{article}
\usepackage{amsmath,amssymb}
\setlength{\oddsidemargin}{0in}
\setlength{\evensidemargin}{0in}
\setlength{\textheight}{9in}
\setlength{\textwidth}{6.5in}
\setlength{\topmargin}{-0.1in}
\usepackage{enumitem}
\usepackage{graphicx}

\begin{document}
\begin{small}
Open and Closed Set Basic Properties Let $(X,d)$ be a metric space with $E\subseteq X$.\\
$E$ is open iff E=int(E). $E$ is closed if it contain all of its boundary points. 
Open and closed balls are open and closed respectively. 
Singleton sets are closed.
E is open iff the complement of E is closed.
The finite intersection of open sets and union of closed sets are open and closed respectively.
The infinite/finite union of open sets and intersection of closed sets are open and closed respectively.
int(E) is the largest open set in E and $\overline{E}$ is the smallest closed set that contains E.
$E\subset Y$ is relatively open/closed with respect to a metric space $Y\subset X$ iff there exists V, a superset of E, which is open/closed in X.\\

Complete Metric Spaces: A metric space $(X,d)$ is said to be complete iff every Cauchy sequence in $(X,d)$ is in fact convergent in $(X, d)$.
Complete subspaces are closed and any closed subset of a complete metric space is complete.\\

Bounded: A subset is bounded iff there exists a ball centered at a point which contains the subset.\\

(Compactness) Limit definition: every sequence has a convergent subsequence. Open cover definition: Every open cover can be reduced to a finite subcover.\\
Compact sets are complete (closed) and bounded. If in a Euclidean space then closed and bounded are sufficient conditions. A subset of a compact metric space is compact iff it is closed. 
Finite collections of compact metric spaces are compact. Finite subsets are compact.\\

Continuity: Continuity preserves convergence. 
Epsilon delta, sequences, and open sets definitions are all equivalent. 
The preimage of closed and open sets are open and closed respectively.
Continuity preserved by composition.

(Continuous maps preserve compactness). 
Let $f : X \rightarrow Y$ be a continuous map from one metric space $(X,dX)$ to an- other $(Y, dY )$. 
Let $K \subseteq X$ be any compact subset of $X$. Then the image $f(K) := {f(x) : x \in K}$ of K is also compact.\\

(Maximum principle). 
Let $(X, d)$ be a compact metric space, and let $f : X \rightarrow R$ be a continuous function. 
Then $f$ is bounded. 
Furthermore, $f$ attains its maximum at some point $x_{max} \in X$, and also attains its minimum at some point $x_{min} \in X$.\\
Continuous functions on compact sets are automatically uniformly continuous.\\

(Connected spaces). 
Let $(X, d)$ be a metric space. 
We say that $X$ is disconnected iff there exist disjoint non-empty open sets $V$ and $W$ in $X$ such that $V \cup W = X$. 
We say that $X$ is connected iff it is non-empty and not disconnected. (The connectedness of $X$ is equivalent to for any $x,y\in X$ where $x<y$ $[x,y]\subseteq X$)\\
Continuity preserves connectedness.\\

Let $(X, dX )$ and $(Y, dY )$ be metric spaces, let $E$ be a subset of $X$, and let $f:X\rightarrow Y$ be a function. Let $x_0 \in X$ be an adherent point of $E$ and $L \in Y$ . Then the following four statements are logically equivalent:
(a) $\lim_{x\rightarrow x_0;x\in E}f(x)=L$, 
(b) For every sequence $(x^{(n)})^\infty_{n=1}$ in E which converges to $x_0$ with respect to the metric $dX$, the sequence $(f(x^{(n)}))^\infty_{n=1}$ converges to $L$ with respect to the metric $dY$,
(c) For every open set $V \subset Y$ which contains $L$, there exists an open set $U\subset X$ containing $x_0$ such that $f(U\cap E)\subseteq V$, 
(d) If one defines the function $g : E\cup{x_0} \rightarrow Y$ by defining $g(x_0) := L$, and $g(x) := f(x)$ for $x \in E\setminus{x_0}$, then $g$ is continuous at $x_0$. Furthermore, if $x_0 \in E$, then $f(x_0) = L$.\\

(Pointwise convergence). For every $x$ and every $\epsilon > 0$ there exists $N > 0$ such that $dY$ $(f(n)(x),f(x)) < \epsilon$ for every $n > N$. We call the function $f$ the pointwise limit of the functions $f^{(n)}$.\\

(Uniform convergence). For every $\epsilon > 0$ there exists $N > 0$ such that $dY$ $(f(n)(x),f(x)) < \epsilon$ for every $n > N$ and every $x$. We call the function $f$ the uniform limit of the functions $f^{(n)}$.\\

Uniform limits preserve Continuity. We can exchange the order of limits and uniform convergence in complete metric spaces.\\

(Bounded functions). A function $f : X \rightarrow Y$ from one metric space $(X,dX)$ to another $(Y,dY)$ is bounded if $f(X)$ is a bounded set, i.e., there exists a ball $B(Y,dY )(y_0,R)$ in Y such that $f(x) \in B(Y,dY )(y_0, R)$ for all $x \in X$. (Uniform limits preserve boundedness)\\

(Metric space of bounded functions). $B(X\rightarrow Y):=\{f|f:X\rightarrow Y \text{is a bounded function}\}$ with notion of distance $d_\infty(f,g) := \sup\{dY (f(x),g(x)) : x \in X\}$ (The space of continuous functions is complete.)\\

(Sup norm). $||f||_\infty=\sup\{|f(x)|:x\in X\}$ (Weierstrauss M-test). $\displaystyle\sum_{n=1}^{\infty}f^{(n)}$ converges uniformly if $\displaystyle\sum_{n=1}^{\infty}||f^{(n)}||_\infty$ converges.\\

Let $[a, b]$ be an interval, and for each integer $n \ge 1$, let $f (n) : [a, b] \rightarrow R$ be a Riemann-integrable function. Suppose $f (n)$ converges uniformly on $[a, b]$ to a function $f : [a, b] \rightarrow R$. Then $f$ is also Riemann integrable, and $\displaystyle\lim_{n\rightarrow\infty}\int_{[a,b]}f^{(n)}=\int_{[a,b]}f$.\\

If $f_n'$ converges uniformly, and $f_n(x_0)$ converges for some $x_0$, then $f_n$ also converges uniformly, and $\frac{d}{dx}\lim_{n\rightarrow\infty}f^{(n)}(x)=\lim_{n\rightarrow\infty}\frac{d}{dx}f^{(n)}(x)$.\\

Exchanging the order of series and integration/differentiation uses the same logic as exchanging limits with integration/differentiation.\\

Radius of convergence $R=\frac{1}{\limsup_{n\rightarrow\infty}}|c_n|^\frac{1}{n}$\\

(Real Analytic Functions): Let E be a subset of R, and let $f:E\rightarrow R$ be a function. 
If a is an interior point of E, we say that f is real analytic at a if there exists an open interval $(a - r, a + r)$ in E for some $r > 0$ such that there exists a power series $\displaystyle\sum_{n=0}^{\infty} c_n(x-a)^n$ centered at a which has a radius of convergence greater than or equal to r, and which converges to f on $(a-r,a+r)$.
If E is an open set, and f is real analytic at every point a of E, we say that f is real analytic on E.\\
Real analytic functions are infinitely differentiable.\\
$\displaystyle \exp(x)=\sum_{n=0}^{\infty}\frac{x^n}{n!}$ is absolutely convergent, with infinite radius of convergence, and is analytic for all R.\\
$\displaystyle \log(1-x)=-\sum_{n=1}^{\infty}\frac{x^n}{n}$, $\displaystyle \log(x)=\sum_{n=1}^{\infty}\frac{(-1)^{n+1}}{n}(x-1)^n$ with radius of convergence 1.\\ 
(Trigonometric functions)\\ 
$\displaystyle\sin(x)=\sum_{n=0}^{\infty}\frac{(-1)^nx^{2n}}{(2n)!}$ and $\displaystyle\cos(x)=\sum_{n=0}^{\infty}\frac{(-1)^nx^{2n+1}}{(2n+1)!}$.\\
$\sin(x)^2+\cos(x)^2=1$ where $\sin(x)\in[-1,1]$ and $\cos(x)\in[-1,1]$. $\sin'(x)=\cos(x)$ and $\cos'(x)=-\sin(x)$. $\sin(-x)=-\sin(x)$ and $\cos(-x)=\cos(x)$. $\cos(x+y)=\cos(x)\cos(y)-\sin(x)\sin(y)$ and $\sin(x+y)=\sin(x)\cos(y)+\sin(y)\cos(x)$. $\sin(0)=0$ and $\cos(0)=1$. Both functions are $2\pi$ periodic.


\end{small}
\end{document}

