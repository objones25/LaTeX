\documentclass[10pt]{article}
\usepackage{amsmath,amssymb}
\setlength{\oddsidemargin}{0in}
\setlength{\evensidemargin}{0in}
\setlength{\textheight}{10in}
\setlength{\textwidth}{6.5in}
\setlength{\topmargin}{-0.1in}
\usepackage{enumitem}
\usepackage{graphicx}
\usepackage{multicol}


\begin{document}
    \begin{tiny}
    Identity: $(\displaystyle{\sum_{i=1}^{n}}a_ib_i)^2+\frac{1}{2}\displaystyle{\sum_{i=1}^{n}\sum_{j=1}^{n}}(a_ib_j-a_jb_i)^2=(\displaystyle{\sum_{i=1}^{n}}a_{i}^{2})(\displaystyle{\sum_{i=1}^{n}}b_{i}^{2})$\hspace*{1cm}
    Triangle Innequality: $(\displaystyle{\sum_{i=1}^{n}}(a_{i}+b_{i})^2)^\frac{1}{2}\le(\displaystyle{\sum_{i=1}^{n}}a_{i}^{2})^\frac{1}{2}+(\displaystyle{\sum_{i=1}^{n}}b_{i}^{2})^\frac{1}{2}$\\
    \end{tiny}
    \begin{small}
    Open and Closed Sets: Open contains none of the boundary points. Closed contains all of the boundary points.\\
    A subset E of X is only open if there exists a ball centered at each point in E that fits entirely inside E. A subset E is only closed if every convergent sequence converges in E.\\
    Open balls are open sets and closed balls are closed sets. Any singleton set is closed. A set is open iff the complement is closed.\\ 
    Finite collections of open (intersection) and closed (union) sets are open and closed respectively.\\
    Infinite/Finite collections of open (union) and closed (intersection) sets are open and closed respectively.\\
    $Int(E)$ is the largest open set in $E$, $\overline{E}$ is the smallest closed set which contains $E$.\\

    Relative topology: Let $(X,d)$be a metric space, let Y be a subset of $X$, and let E be a subset of Y. 
    We say that E is relatively open with respect to Y if it is open in the metric subspace $(Y, d_{|Y\times Y})$. 
    Similarly, we say that E is relatively closed with respect to Y if it is closed in the metric space $(Y, d_{|Y\times Y})$. 
    
    \begin{itemize}
        \item [a)] $E$ is relatively open with respect to $Y$ iff $E=V\cap Y$ for some set $V\subset X$ which is open in $X$.
        \item [b)] $E$ is relatively closed with respect to $Y$ iff $E=K\cap Y$ for some set $K\subset X$ which is closed in $X$.
    \end{itemize}

    Complete metric spaces: A metric space $(X,d)$ is said to be complete iff every Cauchy sequence in $(X,d)$ is in fact convergent in $(X,d)$. If a subspace is complete, it must be closed in the metric space. If a metric space $(X,d)$ is complete, every subspace must be closed in $(X,d)$.\\

    Compact metric spaces: A metric space $(X,d)$ is said to be compact iff every sequence in $(X, d)$ has at least one convergent subsequence. A subset $Y$ of a metric space $X$ is said to be compact if the subspace $(Y, d_{Y\times Y})$ is compact.\\
    Alternatively: Any collection of open sets that cover $X$ can be reduced to a finite subcover.\\

    Bounded sets: Let $(X, d)$ be a metric space $Y$ is bounded iff it can fit entirely inside a ball. Compact sets are complete and bounded.\\

    Heine-Borel: Compact subsets are closed and bounded. For Euclidean spaces with Euclidean metric, a metric space is compact iff it is closed and bounded.\\

    Every nested sequence of compact subsets of $X$ is non-empty.\\

    Let $(X,d)$ be a metric space.
    \begin{itemize}
        \item [a)] If $Y$ is a compact subset of $X$, and $Z\subset Y$, then $Z$ is compact iff $Z$ is closed.
        \item [b)] If $Y_1, Y_2,...,Y_n$ are a finite collection of compact subsets of $X$, then their union is also compact.
        \item [c)] Every finite subset of $X$ (including the empty set) is compact.
    \end{itemize}

    Continuity preserves convergence
    \begin{itemize}
        \item [a)] $f$ is continuous at $x_0$.
        \item [b)] Whenever $(x^{(n)})_{n=1}^{\infty}$ is a sequence in $X$ which converges to $x_0$ with respect to the metric $d_X$, the sequence $(f(x^{(n)}))_{n=1}^{\infty}$ converges to $f(x_0)$ with respect to the metric $d_Y$. 
        \item [c)] For every open set $V\subset Y$ that contains $f(x_0)$, there exists an open set $U\subset X$ containing $x_0$ such that $f(U)\subset V$.
    \end{itemize}
    In addition
    \begin{itemize}
        \item [c)] Whenever $V$ is an open set in $Y$, the set $f^{-1}(V)=\{x\in X:f(x)\in V\}$ is an open set in $X$.
        \item [d)] Whenever $F$ is an closed set in $Y$, the set $f^{-1}(F)=\{x\in X:f(x)\in F\}$ is a closed set in $X$.
    \end{itemize}
    Continuity and product spaces: Let $f:X\rightarrow\mathbb{R}$ and $g:X\rightarrow\mathbb{R}$ be functions, and let $f\oplus g:X\rightarrow\mathbb{R}^2$ be their direct sum. We give $\mathbb{R}^2$ the Euclidean metric. 
    \begin{itemize}
        \item [a)] If $x_0\in X$, then $f$ and $g$ are both continuous at $x_0$ iff $f\oplus g$ is continuous at $x_0$.
        \item [b)] $f$ and $g$ are both continuous iff $f\oplus g$ is continuous.
    \end{itemize} 

    \end{small}  
\end{document}