\documentclass[10pt]{article}
\usepackage{graphicx}
\usepackage{amssymb}
\usepackage[fleqn]{amsmath}
\usepackage{nccmath}
\usepackage{cases}
\usepackage{hyperref}
\usepackage{multicol}
\usepackage{tikz}
\usepackage{pgfplots}
\usepackage{enumitem}
\pgfplotsset{compat=1.18}
\usepackage{float}

\title{\bf Math 100: Problem Set 4}
\date{11/1/2023}
\author{\bf Owen Jones}
\begin{document}
\maketitle
\begin{enumerate}[label= (Q-\arabic*)]
\item We can express the number $1$ written $p-1$ times as $\displaystyle \sum_{k=0}^{p-2}10^k$. 
It follows $\displaystyle \sum_{k=0}^{p-2}10^k=\frac{10^{p-1}-1}{10-1}$. 
By Fermat's Little Theorem $10^{p-1}\equiv1\mod(p)$, so $p$ divides $10^{p-1}-1$.
Therefore, $p$ must divide either $\displaystyle \sum_{k=0}^{p-2}10^k$ or $9$. 
$p$ is a prime greater than $3$, so $p$ can't divide $9$. Thus, $p$ must divide $\displaystyle \sum_{k=0}^{p-2}10^k$.
\item Since $\gcd(2,121)=1$, we can use Euler's Theorem to show $2^{\phi(121)}\equiv1\mod(121)$. 
$\phi(121)=121(1-\frac{1}{11})=110$ because $121$ is relatively prime to every number up to itself except for multiples of $11$.
Thus $2^{998}=2^{8}2^{9\phi(121)}\equiv 2^{8}\mod (121)$. $2^8=256=2\cdot121+14\Rightarrow2^{998}=14\mod(121)$.
\item $p^{5-1}\equiv1\mod(5),p^{3-1}\equiv1\mod(3)$, and $p^{\phi(16)}\equiv1\mod(16)$ by Euler's Theorem. 
It follows $p^{2(5-1)}\equiv1\mod(5),p^{4(3-1)}\equiv1\mod(3)$, and $\phi(16)=16(1-\frac{1}{2})=8$.
Therefore $p^8-1$ is divible by $3,5$, and $16$. Since $3,5$, and $16$ are all relatively prime, $p^8-1$ must be divisible by their product, $240$.
\item We want to use the Euclidean Algorithm. 
\begin{align*}
    x^8-1=x^3(x^5-1)+x^3-1\\
    x^5-1=x^2(x^3-1)+x^2-1\\
    x^3-1=x(x^2-1)+x-1\\
    x^2-1=(x+1)(x-1)\\
\end{align*}
This implies
\begin{align*}
    (x^3-1)-x(x^2-1)=x-1\\
    (x^8-1-x^3(x^5-1))-x(x^5-1-x^2(x^3-1))=x-1\\
    x^8-1-(x^3+x)(x^5-1)+x^3(x^3-1)=x-1\\
    x^8-1-(x^3+x)(x^5-1)+x^3(x^8-1-x^3(x^5-1))=x-1\\
    (x^3+1)(x^8-1)-(x^6+x^3+x)(x^5-1)=x-1
\end{align*}
Thus, we choose $F(x)=x^3+1$ and $G(x)=-x^6-x^3-x$
\item Let $P(x)=x^{4a}+x^{4b+1}+x^{4c+2}+x^{4d+c}$ and $Q(x)=x^3+x^2+x+1$. 
We can express $P (x)=A (x) Q (x)+R (x)$ where $A(x)$ and $R(x)$ are polynomials and $R (x)$ has degree less than $Q(x)$.\\ 
$P(i)=i^{4a}+i^{4b}i+i^{4c}i^2+i^{4d}i^3=1+i+i^2+i^3=1+i-1-i=0\\
P(-i)={(-i)}^{4a}+{(-i)}^{4b}{(-i)}+{(-i)}^{4c}{(-i)}^2+{(-i)}^{4d}{(-i)}^3=1-i-1+i=0\\
P(-1)={(-1)}^{4a}+{(-1)}^{4b}{(-1)}+{(-1)}^{4c}{(-1)}^2+{(-1)}^{4d}{(-1)}^3=1-1+1-1=0\\$
$P(x)=Q(x)$ at three different points, so $R(x)=0$ at those three points. 
However, the only polynomial of degree less than $3$ that has more than $2$ roots is the zero function.
Thus $Q(x)$ is a factor of $P(x)$
\item \begin{itemize}
    \item [(a)] $x^8+x^4+1=x^8+2x^4+1-x^4\\
    ={(x^4+1)}^2-{(x^2)}^2\\
    = (x^4-x^2+1) (x^4+x^2+1)$
    \item [(b)] $(x^4-x^2+1) (x^4+x^2+1)=(x^2+\sqrt{3}x+1)(x^2-\sqrt{3}x+1)(x^2+x+1)(x^2-x+1)$
    \item [(c)] $ (x^2+\sqrt{3}x+1) (x^2-\sqrt{3}x+1) (x^2+x+1) (x^2-x+1)= (x+\frac{\sqrt{3}-i}{2}) (x+\frac{\sqrt{3}+i}{2})(x-\frac{\sqrt{3}-i}{2}) (x-\frac{\sqrt{3}+i}{2})(x-\frac{1+\sqrt{3}i}{2})(x-\frac{1-\sqrt{3}i}{2})(x+\frac{1+\sqrt{3}i}{2})(x+\frac{1-\sqrt{3}i}{2})$
\end{itemize}
\item \begin{itemize}
    \item [(a)] Let $f(x)$ have the rational root $\frac{p}{q}$. 
    It follows by Gauss' Lemma that $f(x)=(qx-p) a(x)$ for some polynomial $a(x)$ with integer coefficients.
    We are given that $f(1)$ is odd, so $(q-p)$ must be odd and $a(1)$ must be odd. 
    It follows that $p$ and $q$ have different parity. 
    However, this means that $p\nmid a_0$ or $q\nmid a_n$ because an odd number can't be divisible by an even number.
    Thus, we obtain a contradiction, so $f(x)$ has no rational roots.
    \item [(b)] $x^{13}+x+90=f(x)(x^2-x+a)$ for some polynomial with integer coefficients.
    Observe $1^{13}+1+90-0^{13}-0-90=2=a(f(1)-f(0))$. 
    Thus, $a$ divides $2$.
    It follows $a=\pm1,\pm2$. 
    WTS $x^2-x+2|x^{13}+x+90$
    By long division $x^{13}+x+90=(x^11+x^10-x^9-3x^8-x^7+5x^6+7x^5-3x^4-17x^3-11x^2+23x+45)$
\end{itemize}
\item \begin{itemize}
    \item [(a)] We will show both directions by induction. Let $F(x)$ be a polynomial. For the forward direction, we assume $a$ is a zero multiplicity of $m+1$.\\
    $P(0):$ If $a$ is a zero multiplicity of $1$ then let $F(x):=(x-a)B(x)$ where $B(x)$ is a polynomial that is nonzero at $x=a$. 
    Because $a$ is a root $F(a)=0$, but $F'(a)=B(a)+(a-a)B'(a)=B(a)\neq0$. Thus, $P(0)$ holds.\\
    $P(m):$ Assume $F^{(i)}(a)=0$ for $0\le i\le m$ and $F^{(m+1)}(a)\neq0$ for some $m$.\\
    $P(m+1):$ Let $F(x):={(x-a)}^{m+2}B(x)$ where $B(x)$ is a polynomial that is nonzero at $x=a$.\\ 
    $F(a)=0$ because $a$ is a root.
    $F'(x)=(m+2){(x-a)}^{m+1}B(x)+{(x-a)}^{m+2}B'(x)\\
    =((m+2)B(x)+B'(x)(x-a)){(x-a)}^{m+1}$\\
    Because $(m+2)B(a)+B'(a)(a-a)\neq0$, $a$ is a zero multiplicity of $m+1$ for $F'(x)$, so by the induction hypothesis $F^{(i+1)}(a)=0$ for $0\le i\le m$ and $F^{(m+2)}(a)\neq0$.
    Hence, by induction, the claim holds for all $m$\\
    For the reverse direction, we assume $F^{(i)}(a)=0$ for all $0\le i\le m$.\\
    $P(0):$ $F(a)=0\Leftrightarrow$ $a$ is a root of $F(x)$. Thus, we can define $F(x):=(x-a)B(x)$. $F(a)=(a-a)B(a)=0$. $F'(x)=B(x)+(x-a)B'(x)$ and $F'(a)\neq0$, so $a$ is not a root of $B(x)\Rightarrow$ $a$ is a zero multiplicity of $1$.\\
    $P(m):$ If $F^{(i)}(a)=0$ for all $0\le i\le m$ assume $a$ is a zero multiplicity of $m+1$.\\
    $P(m+1):$ Because $F(a)=0$ we can write $F(x):=(x-a)B(x)\Rightarrow F'(x)=B(x)+(x-a)B'(x)$. 
    Because $F'(a)=0$, we can write $F'(x):=(x-a)(B_1(x)+B'(x))$ where $B(x)=(x-a)B_1(x)$. 
    Now we have a function $F'(x)$ s.t $F^{(i+1)}(a)=0$ for all $0\le i\le m$, so $a$ has zero multiplicity $m+1$.
    Therefore, for the function $F(x)$, $a$ is a zero multiplicity of $m+2$.
    Hence, by induction, the claim holds for all $m$.
    \item [(b)] $f(1)=0$, $f'(1)=n1^{n-1}-n=0$, $f''(1)=n(n-1)1^{n-1}=n(n-1)\neq=0$, so $1$ is a zero multiplicity of $2$.

\end{itemize}
\item \begin{itemize}
    \item [(a)] $x^3+a^2+bx+c=0\Rightarrow x+a+\frac{b}{x}+\frac{c}{x^2}=0$ if $x\neq0$. None of $r,s,t=0$ because $c\neq0$ so\\
    \begin{align*}
        r+a+\frac{b}{r}+\frac{c}{r^2}=0\\
        s+a+\frac{b}{s}+\frac{c}{s^2}=0\\
        t+a+\frac{b}{t}+\frac{c}{t^2}=0\\
        r+s+t+3a+b(\frac{rs+st+rt}{rst})=c(\frac{1}{r^2}+\frac{1}{s^2}+\frac{1}{t^2})=0\\
        \frac{b^2}{c^2}-\frac{2a}{c}=\frac{1}{r^2}+\frac{1}{s^2}+\frac{1}{t^2}
    \end{align*}
    \item [(b)] $r+s+t=-a\Rightarrow r^2+s^2+t^2=a^2-2(rs+st+rt)=a^2-2b\\
    rs+st+rt=b\Rightarrow r^2s^2+s^2t^2+r^2t^2=b^2-2rst(r+s+t)=b^2-2ac\\
    rst=-c\Rightarrow r^2s^2t^2=c^2$\\
    $x^3+(2b-a^2)x^2+(b^2-2ac)x-c^2$ will have root of $r^2,s^2,t^2$.
\end{itemize}
\item $ab^p-ba^p=ab(b^{p-1}-a^{p-1})$. It follows by Fermat's Little Theorem that $a^{p-1}\equiv1\mod(p)$ and $b^{p-1}\equiv1\mod(p)$. Thus, $p|(b^{p-1}-a^{p-1})\Rightarrow p|ab^p-ba^p$.
\end{enumerate}
\end{document}