\documentclass[12pt]{article}
\usepackage{amsmath,amssymb}
\setlength{\oddsidemargin}{0in}
\setlength{\evensidemargin}{0in}
\setlength{\textheight}{9in}
\setlength{\textwidth}{6.5in}
\setlength{\topmargin}{-0.5in}
\usepackage{enumitem}
\usepackage[table]{xcolor}
\usepackage{graphicx}
\usepackage{listings}
\newcommand{\Adv}{{\mathbf{Adv}}}       
\newcommand{\prp}{{\mathrm{prp}}}             
\newcommand{\calK}{{\cal K}}
\newcommand{\outputs}{{\Rightarrow}}                



% Use \textbf{Solution:}\\ to begin your solution.

%%%%%%%%%%%%%%%%%%%%%%%%%%%%%%%%%%%%%%%%%%%%%%%%%%%%%%%%%%%%%%%%%%%%%%%%%%%
\title{\bf Math 151A: Problem Set 1}
\date{ }
\author{\bf Prof. Schaeffer}

\begin{document}
\maketitle


{\small \textbf{Instructions}:
\begin{itemize}
\item Due on Friday, April 14th by 1:00pm.
\item Late HW will not be accepted.
\item Write down all of the details and attach your code to the end of the assignment for full credit. 
\item If you LaTeX your solutions, you will get $5\%$ extra credit. 
\item (T) are ``pencil-and-paper'' problems and (C) include a computational/programming component. 
\end{itemize}}

%Add \newpage  between problelms if you are using this to LaTex your solutions




%%%%%%%%%%%%%%%%%%%%%%%%%%%%%%%%%%%%%%%%%%%%%%%%%%%%%%%%
\begin{enumerate}[label=\bfseries Problem \arabic*:]


%%%%%%%%%%%%%%%%%%%%%%%%%%%%%%

\item \textbf{(T) Taylor's Theorem}\\
Let $f(x)=e^{2x}$ for $x\in[0,2]$.

\begin{itemize} 
\item[a)] Find Taylor's polynomial of degree-$2$, i.e. $P_2(x)$, around the point $x_0=0$ and use it to approximate the value of $f(1.5)$, i.e. $f(1.5)\approx P_2(1.5)$. 


\item[b)]  What is the error as a function of $\xi(x)$ when $x=1.5$ (specify the domain for $\xi(1.5)$)?


\item[c)]  What is the actual error (in magnitude)?


\end{itemize}




\vspace{1em}
%%%%%%%%%%%%%%%%%%%%%%%%%%%%%%
\item \textbf{(T) Bisection Method}\\
Let $f(x)=\sqrt{\pi\, x}-\cos(\pi\, x)$  over the interval $[0,1]$.  We would like to find $p$ such that $f(p)=0$.
\begin{itemize}
\item[a)] Show that the bisection method applied to this problem converges (apply the theorem from class).
  
\item[b)] How many iterations are needed to have a $10^{-q}$-accurate approximation to the true root where $q>1$? Write your answer in the form $n\geq C q$ where $C$ is an explicit constant that you need to provide.  
 
\end{itemize}



\vspace{1em}
%%%%%%%%%%%%%%%%%%%%%%%
\item \textbf{(C) Bisection Method}\\
Find a $10^{-5}$-accurate approximation to $\sqrt[4]{25}$ using the Bisection Algorithm. To do so, you will need to define a function $f(x)$ whose root is $\sqrt[4]{25}$. The function $f(x)$ must only use simple operations: multiplication and addition/subtraction. Use the corollary from class to determine the number of steps required to achieve the given accuracy.



 \vspace{1em}
  %%%%%%%%%%%%%%%%%%%%%%%
\item \textbf{(T) Bisection Method}\\
You will show that the bisection method may not converge monotonically. Provide a continuous function $f(x)$ and an interval $[a,b]$ so that the error at the $k$-th step, denoted $E_{k}=|p_k-p|$, increases between some iterations although the sequence $p_k$ converges to the unique root. To receive credit for this problem, you must justify your answer and prove that your example is convergent.

 
 
 
%%%%%%%%%%%%%%%%%%%%%%%
\item \textbf{(T) Stopping Criteria for General Root-Finding Algorithms}\\
Assume that we have a sequence $p_n$ for $n=1,2,...$ that is generated by an algorithm in order to find the root of a function $f(x)$. Let $\epsilon$ be the prescribed tolerance used to stop the iterative process. 

You may use, without proof, that  $\sum_{k=1}^n \frac{1}{k}$ diverges as $n\rightarrow \infty$.



\begin{itemize}
\item[a)] Consider the stopping criterium $|p_{n}-p_{n-1}|< \epsilon$ for $n>2$. Show that $p_n=\sum_{k=1}^n \frac{1}{k}$ satisfies the criterium when $n\geq\frac{1}{\epsilon}$; however, $p_n \rightarrow \infty$  (thus the sequence does not converge to a finite value).


\item[b)] Consider the stopping criterium $\frac{|p_{n}-p_{n-1}|}{|p_n|}< \epsilon$ for $n>2$ and $p_n\neq 0$. Show that $p_n=\sum_{k=1}^n \frac{1}{k}$ satisfies the criterium; however, $p_n \rightarrow \infty$ (thus the sequence does not converge to a finite root).

 
 \item[c)] Let  $f(x):=(x-1)^{10}$, whose root is $p=1$, and define the sequence $p_n=1+\frac{1}{n}$. Note that $p_n$ goes to the root in the limit. Show that the stopping criterium $|f(p_n)| < 10^{-3}$ is achieved for all $n>1$ but $|p-p_n|\leq 10^{-3}$ requires $n>1000$.
 

 
 
 
\end{itemize}



\end{enumerate}

\end{document}


