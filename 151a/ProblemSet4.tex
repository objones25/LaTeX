\documentclass[12pt]{article}
\usepackage{amsmath,amssymb}
\setlength{\oddsidemargin}{0in}
\setlength{\evensidemargin}{0in}
\setlength{\textheight}{9in}
\setlength{\textwidth}{6.5in}
\setlength{\topmargin}{-0.5in}
\usepackage{enumitem}
\usepackage[table]{xcolor}
\usepackage{graphicx}
\usepackage{listings}
\newcommand{\Adv}{{\mathbf{Adv}}}       
\newcommand{\prp}{{\mathrm{prp}}}             
\newcommand{\calK}{{\cal K}}
\newcommand{\outputs}{{\Rightarrow}}                



% Use \textbf{Solution:}\\ to begin your solution.

%%%%%%%%%%%%%%%%%%%%%%%%%%%%%%%%%%%%%%%%%%%%%%%%%%%%%%%%%%%%%%%%%%%%%%%%%%%
\title{\bf Math 151A: Problem Set 4}
\date{ }
\author{\bf Prof. Schaeffer}

\begin{document}
\maketitle


{\small \textbf{Instructions}:
\begin{itemize}
\item Due on Friday, May 12th by 1pm 
\item Late HW will not be accepted (but this assignment will be accepted until Monday, May 15th by 1pm)
\item Write down all of the details and attach your code to the end of the assignment for full credit (as a PDF).  
\item If you LaTeX your solutions, you will get $5\%$ extra credit. 
\item (T) are ``pencil-and-paper'' problems and (C) means that the problem includes a computational/programming component. 
\end{itemize}}

%Add \newpage  between problems if you are using this to LaTex your solutions
\vspace{1em}

%%%%%%%%%%%%%%%%%%%%%%%%%%%%%%%%%%%%%%%%%%%%%%%%%%%%%%%%
\begin{enumerate}[label=\bfseries Problem \arabic*:]






 
\item \textbf{(T) Lagrange Polynomials and Neville's Method}\\
Use Neville's method to obtain approximations to $f(0.1)$ using the Lagrange interpolating polynomials of degrees one, two, and three  if\\
 $f(0)=1$, $f(0.25)=1.65$, $f(0.5)=2.72$, and $f(0.75)=4.48$. You should explicitly compute the table associated with Neville's method:
 \begin{table}[h]
\centering
\rowcolors{1}{lightgray}{lightgray}
\resizebox{\textwidth}{!}{\begin{tabular}{|c|c|c|c|c|c|c|}\hline
    $x_i$ & $x-x_i$&  $Q_{i,0}$ & $Q_{i,1}$ & $Q_{i,2}$ & $Q_{i,3}$    \\ \hline
    $0$ & $0.1$ & $1$  & - & - & -    \\ \hline
    $0.25$ & $0.1-0.25$ & $1.65$  & $\frac{f(0.25)0.1-f(0)(0.1-0.25)}{0.25-0}$ & - &-    \\ \hline
    $0.5$ & $0.1-0.5$ & $2.72$  & $\frac{f(0.5)(0.1-0.25)-f(0.25)(0.1-0.5)}{0.5-0.25}$ & $\frac{\frac{1.07\cdot0.1+0.145}{0.25}(0.1)-\frac{0.65\cdot0.1+0.25}{0.25}(0.1-0.5)}{0.5-0}$ & -    \\ \hline
    $0.75$ & $0.1-0.75$ & $4.48$  & $\frac{f(0.75)(0.1-0.5)-f(0.5)(0.1-0.75)}{0.75-0.5}$ & $\frac{\frac{1.76\cdot0.1-0.2}{0.25}(0.1-0.25)-\frac{1.07\cdot0.1+0.145}{0.25}(0.1-0.75)}{0.75-0.25}$ & $\frac{\frac{0.69(0.1)^2+0.0175\cdot0.1+0.15875}{0.125}(0.1)-\frac{0.42(0.1)^2+0.22\cdot0.1+0.125}{0.125}(0.1-0.75)}{0.75-0}$     \\ \hline
\end{tabular}}   
\end{table}\\
$P_3(x)=2.88x^3+1.2x^2+2.12x+1$, $P_3(0.1)=1.22688$\\
(The true value is $f(0.1)=1.2214$.)

\vspace{1em}
 %%%%%%%%%%%%%%%%%%%%%%%%%%%%%%
\newpage
\item \textbf{(T) Lagrange Polynomials and Neville's Method}\\
Suppose $x_j=j$, for $j=0, 1, 2, 3$, and it is known that:

\begin{center} 
$P_{0,1}(x)=2x+1$, $P_{0,2}(x)=x+1$, $P_{1,2,3}(2.5)=3$ 
\end{center}
Determine $P_{0,1,2,3}(2.5)$.


\vspace{1em}
\textbf{Solution:}\par
$P_{0,1,2}(2.5)=\frac{P_{0,2}(2.5)(2.5-1)-P_{0,1}(2.5)(2.5-2)}{2-1}$\\
$P_{0,1,2,3}(2.5)=\frac{P_{1,2,3}(2.5)(2.5)-P_{0,1,2}(2.5)(2.5-3)}{3-0}=\frac{(3)(2.5)-((3.5)(1.5)-(6)(0.5))}{3}=2.875$
%%%%%%%%%%%%%%%%%%%%%%%%%%%%%%
\newpage
\item \textbf{(T) Lagrange Polynomials and Neville's Method}\\
Suppose $x_j=2\,j$, for $j=0, 1, 2, 3, 4$  and it is known that:

\begin{center} 
$P_{1,2}(1)=2$, \hspace{1em} $P_{1,2,3}(1)=1$,\hspace{1em}  $P_{1,4}(1)=6$.
\end{center}
Determine $P_{1,2,3,4}(1)$.

\vspace{1em}
\textbf{Solution:}\par
$P_{1,2,4}(1)=\frac{P_{1,4}(1)(1-2)-P_{1,2}(1)(1-4)}{4-2}$\\
$P_{1,2,3,4}(1)=\frac{P_{1,2,4}(1)(1-3)-P_{1,2,3}(1)(1-4)}{4-3}=\frac{(6)(-1)-(2)(-3)}{2}(-2)-(1)(-3)=3$
%%%%%%%%%%%%%%%%%%%%%%%%%%%%%%%
\newpage

 \item \textbf{(T) Newton's Divided Differences}
 \begin{itemize}
\item[a)] Find the degree-2 interpolating polynomial via Newton's divided difference for $f(x)  = \frac{x}{1+x}$ using nodes $x_0=0$, $x_1=1$, and $x_2=2$.
\item[b)] What are the degree-2 interpolating polynomials associated with Lagrange's construction and Neville's construction? Compare them to the solution of Part (a).
 \end{itemize}
 



\vspace{1em}
\textbf{Solution:}\par
\begin{itemize}
    \item [a)] Using nodes $x_0=0$, $x_1=1$, and $x_2=2$, $P_2(x)=f(x_0)+\frac{f(x_1)-f(x_0)}{x_1-x_0}(x-x_0)+\frac{\frac{f(x_2)-f(x_1)}{x_2-x_1}-\frac{f(x_1)-f(x_0)}{x_1-x_0}}{x_2-x_0}(x-x_0)(x-x_1)\\
    =0+\frac{\frac{1}{2}-0}{1-0}(x-0)+\frac{\frac{\frac{2}{3}-\frac{1}{2}}{2-1}-\frac{\frac{1}{2}-0}{1-0}}{2-0}(x-0)(x-1)\\
    =\frac{2}{3}x-\frac{1}{6}x^2$
    \item [b)] Nevilles's:$P_2(x)=\frac{2}{3}x-\frac{1}{6}x^2$\\
    Lagrange:$P_2(x)=\frac{2}{3}x-\frac{1}{6}x^2$\\
    The degree-2 interpolating polynomials associated with Lagrange's and Neville's construction will be the same by uniqueness.
\end{itemize}
%%%%%%%%%%%%%%%%%%%%%%%%%%%%%%%%%%%
\newpage



 \item \textbf{(T) Numerical Differentiation/Finite Difference}\\
Show that the following finite difference formula is first-order accurate:
\begin{align*}
f^{(2)}(x)= \frac{ f(x+2\, h)-2 f(x+h)+\,f(x)}{h^2}+{O}(h).
\end{align*}
\textit{For full credit, you must state any assumptions on $f(x)$ and justify each step in your solution. Hint: Apply Taylor's theorem to each of the terms on the right-hand side. }\\

\textbf{Solution:}\par
Suppose $x\in(a,b)$ where $f\in C^3[a,b]$ and $h\neq 0$ is sufficiently small to ensure $x+h,x+2h\in(a,b)$\\
Taylor expansions of $f(x+2h)$ and $f(x+h)$ about $x$:\\
$f(x+h)=f(x)+h\cdot f'(x)+\frac{h^2}{2}\cdot f''(x)+\frac{h^3}{6}\cdot f'''(\xi_1(x))$\\
$f(x+2h)=f(x)+2h\cdot f'(x)+\frac{4h^2}{2}\cdot f''(x)+\frac{8h^3}{6}\cdot f'''(\xi_2(x))$\\
$\Rightarrow f(x+2h)-2f(x+h)+f(x)=h^2\cdot f''(x)+h^3\cdot (\frac{4}{3}f'''(\xi_2(x))-\frac{1}{3}f'''(\xi_1(x)))$\\
$\Rightarrow f''(x)=\frac{f(x+2h)-2f(x+h)+f(x)}{h^2}-h\cdot (\frac{4}{3}f'''(\xi_2(x))-\frac{1}{3}f'''(\xi_1(x)))$ by algebra.\\
Thus, $O(h)=-h\cdot (\frac{4}{3}f'''(\xi_2(x))-\frac{1}{3}f'''(\xi_1(x)))$ which is linear.\\
Hence $f''(x)=\frac{f(x+2h)-2f(x+h)+f(x)}{h^2}+O(h)$ is first order accurate.
\end{enumerate}

\end{document}


