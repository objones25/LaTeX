\documentclass[10pt]{article}
\usepackage{graphicx}
\usepackage{amssymb}
\usepackage[fleqn]{amsmath}
\usepackage{nccmath}
\usepackage{cases}
\usepackage{hyperref}
\usepackage{multicol}
\usepackage{tikz}
\usepackage{pgfplots}
\usepackage{enumitem}
\usepackage{pdfpages}
\pgfplotsset{compat=1.18}
\usepackage{float}
\DeclareMathOperator*{\argmin}{arg\,min}

\title{\bf Math 164: Problem Set 9}
\author{\bf Owen Jones}
\begin{document}
\maketitle
\begin{enumerate}
    \item [\textbf{12.18}] The problem is equivalent to minimizing $\frac{1}{2}{\lVert \mathbf{x}-\mathbf{b}\rVert}^2$. 
    If $\mathbf{x}\in \mathcal{R}(\mathbf{A})$, $\exists \mathbf{y}\in\mathbb{R}^n$ s.t $\mathbf{Ay}=\mathbf{x}$\\
    It follows $\mathbf{x}^*=\mathbf{Ay}^*=\mathbf{A}{(\mathbf{A}^\top\mathbf{A})}^{-1}\mathbf{A}^\top\mathbf{b}$ using the least squares method with with variable $\mathbf{y}$.
    \item [\textbf{12.20}] Let $\mathbf{y}=\mathbf{x}-\mathbf{x}_0$, so our problem becomes\\
    $\underset{\mathbf{Ay}=\mathbf{b}-\mathbf{Ax}_0}{\min}\lVert \mathbf{y}\rVert$.
    Using Theorem $12.2$, $\mathbf{y}^*=\mathbf{A}^\top{(\mathbf{A}\mathbf{A}^\top)}^{-1}(\mathbf{b}-\mathbf{Ax}_0)$ is a unique minimizer to our constrained minimization problem.
    Since $\mathbf{x}^*=\mathbf{y}^*+\mathbf{x}_0$,\\ 
    $\mathbf{x}^*=\mathbf{A}^\top{(\mathbf{A}\mathbf{A}^\top)}^{-1}(\mathbf{b}-\mathbf{Ax}_0)+\mathbf{x}_0\\
    =\mathbf{A}^\top{(\mathbf{A}\mathbf{A}^\top)}^{-1}\mathbf{b}-\mathbf{A}^\top{(\mathbf{A}\mathbf{A}^\top)}^{-1}\mathbf{Ax}_0+\mathbf{I}_n\mathbf{x}_0\\
    =\mathbf{A}^\top{(\mathbf{A}\mathbf{A}^\top)}^{-1}\mathbf{b}+(\mathbf{I}_n-\mathbf{A}^\top{(\mathbf{A}\mathbf{A}^\top)}^{-1}\mathbf{A})\mathbf{x}_0$
    \item [\textbf{12.23}] Let $\mathbf{y}\in \mathcal{R}(\mathbf{A}^\top)$ satisfy $\mathbf{Ay}=\mathbf{b}$. 
    It follows $\exists \mathbf{z}\in\mathbb{R}^m$ s.t $\mathbf{A}^\top\mathbf{z}=\mathbf{y}$.\\
    Since $\mathbf{Ay}$ and $\mathbf{Ax}^*$ both equal $\mathbf{b}$\\
    $\mathbf{Ay}-\mathbf{Ax}^*=\mathbf{A}\mathbf{A}^\top\mathbf{z}-\mathbf{A}(\mathbf{A}^\top{(\mathbf{A}\mathbf{A}^\top)}^{-1}\mathbf{b})=0\\
    \Rightarrow \mathbf{A}\mathbf{A}^\top(\mathbf{z}-{(\mathbf{A}\mathbf{A}^\top)}^{-1}\mathbf{b})=0\\
    \Rightarrow \mathbf{z}={(\mathbf{A}\mathbf{A}^\top)}^{-1}\mathbf{b}$ because $\mathbf{A}\mathbf{A}^\top$ is invertible.\\
    Thus, $\mathbf{y}=\mathbf{A}^\top\mathbf{z}=\mathbf{A}^\top{(\mathbf{A}\mathbf{A}^\top)}^{-1}\mathbf{b}=\mathbf{x}^*$
    \item [\textbf{20.2}]\begin{enumerate}
        \item Let $f(\mathbf{x})=x^2_1+2x_1x_2+3x_2^2+4x_1+5x_2+6x_3$,\\ 
        $h(\mathbf{x})={[x_1+2x_2-3,4x_1+5x_3-6]}^\top$,\\
        and $l(\mathbf{x},\mathbf{\lambda})=f(\mathbf{x})+\mathbf{\lambda}^\top h(\mathbf{x})$.\\
        We first want to find an $(\mathbf{x}^*,\mathbf{\lambda}^*)$ that satisfies the Lagrange condition:
        \begin{align*}
            Dl(\mathbf{x}^*,\mathbf{\lambda}^*)=[D_xl(\mathbf{x}^*,\mathbf{\lambda}^*),D_\lambda(\mathbf{x}^*,\mathbf{\lambda}^*)]=\mathbf{0}^\top\\
        \end{align*}
        It follows
        \begin{align*}
            &Dl(\mathbf{x},\mathbf{\lambda})=\begin{bmatrix}
                2 & 2 & 0 & 1 & 4\\
                2 & 6 & 0 & 2 & 0\\
                0 & 0 & 0 & 0 & 5\\
                1 & 2 & 0 & 0 & 0\\
                4 & 0 & 5 & 0 & 0
            \end{bmatrix}\begin{bmatrix}
                x_1\\
                x_2\\
                x_3\\
                \lambda_1\\
                \lambda_2
            \end{bmatrix}-\begin{bmatrix}
                -4\\
                -5\\
                -6\\
                3\\
                6
            \end{bmatrix}=\mathbf{0}^\top\\
            &\Rightarrow \begin{bmatrix}
                2 & 2 & 0 & 1 & 4\\
                2 & 6 & 0 & 2 & 0\\
                0 & 0 & 0 & 0 & 5\\
                1 & 2 & 0 & 0 & 0\\
                4 & 0 & 5 & 0 & 0
            \end{bmatrix}\begin{bmatrix}
                x_1\\
                x_2\\
                x_3\\
                \lambda_1\\
                \lambda_2
            \end{bmatrix}=\begin{bmatrix}
                -4\\
                -5\\
                -6\\
                3\\
                6
            \end{bmatrix}
        \end{align*}
        which reduces to 
        \begin{align*}
            &\begin{bmatrix}
                x_1\\
                x_2\\
                x_3\\
                \lambda_1\\
                \lambda_2
            \end{bmatrix}=\begin{bmatrix}
                \frac{16}{5}\\
                -\frac{1}{10}\\
                -\frac{34}{25}\\
                -\frac{27}{5}\\
                -\frac{6}{5}
            \end{bmatrix}  
        \end{align*}
        by Gaussian elimination.\\
        Next, we want to show this point satisfies the SONC
        \begin{align*}
            &L(\mathbf{x}^*,\lambda^*)=\mathbf{F}(\mathbf{x}^*)+\mathbf{\lambda}^*\mathbf{H}(\mathbf{x}^*)
            =\begin{bmatrix}
                2 & 2 & 0\\
                2 & 6 & 0\\
                0 & 0 & 0
            \end{bmatrix}\\
            &\text{ because }H_k(\mathbf{x}^*)=0,k=1,2\\
            &T(\mathbf{x}^*):=\{\mathbf{v}\in\mathbb{R}^3:\begin{bmatrix}
                1 & 2 & 0\\
                4 & 0 & 5
            \end{bmatrix}\mathbf{v}=0\}=span\{{[\frac{5}{4},-\frac{5}{8},1]}^\top\}\\
            &\forall \mathbf{v}\in T(\mathbf{x}^*),\mathbf{v}^\top L(\mathbf{x}^*,\lambda^*)\mathbf{v}=\alpha^2 {[\frac{5}{4},-\frac{5}{8},1]}^\top\begin{bmatrix}
                2 & 2 & 0\\
                2 & 6 & 0\\
                0 & 0 & 0
            \end{bmatrix}\begin{bmatrix}
                \frac{5}{4}\\
                -\frac{5}{8}\\
                1
            \end{bmatrix}\\
            &=\alpha^2{[\frac{5}{4},-\frac{5}{8},1]}^\top\begin{bmatrix}
                \frac{5}{4}\\
                -\frac{5}{4}\\
                0
            \end{bmatrix}=\frac{75}{32}\alpha^2>0,\forall\alpha\neq0
        \end{align*}
        , so \begin{align*}
            &\begin{bmatrix}
                x_1\\
                x_2\\
                x_3\\
                \lambda_1\\
                \lambda_2
            \end{bmatrix}=\begin{bmatrix}
                \frac{16}{5}\\
                -\frac{1}{10}\\
                -\frac{34}{25}\\
                -\frac{27}{5}\\
                -\frac{6}{5}
            \end{bmatrix}  
        \end{align*} is a strict local minimizer.
        \item Let \begin{align*}
            &f(x)=4x_1+x_2^2\\
            &h(x)=x_1^2+x_2^2-9\\
            &l(x,\lambda)=f(x)+\lambda h(x)\\
            &\Rightarrow \nabla l(x,\lambda)=\begin{bmatrix}
                4+2\lambda x_1\\
                2x_2+2\lambda x_2\\
                x_1^2+x_2^2-9 
            \end{bmatrix}=\begin{bmatrix}
                0\\
                0\\
                0
            \end{bmatrix}
        \end{align*}
        We first observe $\lambda\neq 0$. If so, we contradict the first innequality. Thus, $x_1=-\frac{2}{\lambda}$.\\
        We have two cases for $x_2$. $x_2=0$ or $x_2\neq 0$\\
        The first case gives the first two candidates ${[3,0]}^\top\lambda=-\frac{2}{3},{[-3,0]}^\top\lambda=\frac{2}{3}$.\\
        The second case gives the second two candidates ${[2,\sqrt{5}]}^\top\lambda=-1,{[2,-\sqrt{5}]}^\top\lambda=-1$.\\
        All $4$ candidates are regular.\\
        \begin{align*}
            &L(\mathbf{x},\lambda)=\mathbf{F}(\mathbf{x})+\mathbf{\lambda}\mathbf{H}(\mathbf{x})=\begin{bmatrix}
                2\lambda & 0\\
                0 & 2+2\lambda
            \end{bmatrix}\\
            &T(\mathbf{x})=\{\mathbf{v}\in\mathbb{R}^2:[2x_1,2x_2]\mathbf{v}=0\}
        \end{align*}
        \begin{align*}
            &L({[3,0]}^\top,-\frac{2}{3})=\begin{bmatrix}
                -\frac{4}{3} & 0\\
                0 & \frac{2}{3}
            \end{bmatrix}\\
            &T({[3,0]}^\top)=span\{[0,1]^\top\}\\
            &\forall\mathbf{v}\in T({[3,0]}^\top)\neq\mathbf{0}, \mathbf{v}^\top\begin{bmatrix}
                -\frac{4}{3} & 0\\
                0 & \frac{2}{3}
            \end{bmatrix}\mathbf{v}=\alpha^2\frac{4}{9}>0\\
            &L({[-3,0]}^\top,\frac{2}{3})=\begin{bmatrix}
                \frac{4}{3} & 0\\
                0 & \frac{10}{3}
            \end{bmatrix}>0\\
            &L({[2,\sqrt{5}]}^\top,-1)=\begin{bmatrix}
                -2 & 0\\
                0 & 0
            \end{bmatrix}\\
            &T({[2,\sqrt{5}]}^\top)=span\{[-\sqrt{5},2]^\top\}\\
            &\forall\mathbf{v}\in T({[2,\sqrt{5}]}^\top)\neq\mathbf{0}, \mathbf{v}^\top\begin{bmatrix}
                -2 & 0\\
                0 & 0
            \end{bmatrix}\mathbf{v}=-\alpha^2 10<0\\
            &L({[2,\sqrt{5}]}^\top,-1)=\begin{bmatrix}
                -2 & 0\\
                0 & 0
            \end{bmatrix}\\
            &T({[2,-\sqrt{5}]}^\top)=span\{[\sqrt{5},2]^\top\}\\
            &\forall\mathbf{v}\in T({[2,\sqrt{5}]}^\top)\neq\mathbf{0}, \mathbf{v}^\top\begin{bmatrix}
                -2 & 0\\
                0 & 0
            \end{bmatrix}\mathbf{v}=-\alpha^2 10<0
        \end{align*}
        Thus, ${[3,0]}^\top\lambda=-\frac{2}{3},{[-3,0]}^\top\lambda=\frac{2}{3}$ are strict local minimizers and ${[2,\sqrt{5}]}^\top\lambda=-1,{[2,-\sqrt{5}]}^\top\lambda=-1$ are strict local maximizers.
        \item \begin{align*}
            &f(x)=x_1x_2\\
            &h(x)=x_1^2+4x_2^2-1\\
            &l(x,\lambda)=-x_1x_2+\lambda x_1^2+4\lambda x_2^2\\
            &\nabla l(x,\lambda)=\begin{bmatrix}
                x_2+2\lambda x_1\\
                x_1+8\lambda x_2\\
                x_1^2+4x_2^2-1
            \end{bmatrix}=\begin{bmatrix}
                0\\
                0\\
                0
            \end{bmatrix}
        \end{align*}
        $x_1=-8\lambda x_2\Rightarrow x_2(1-16\lambda^2)=0$. $x_2\neq 0$ by inspection.\\
        $\lambda=\frac{1}{4}\Rightarrow {[\frac{1}{\sqrt{2}},-\frac{1}{2\sqrt{2}}]}^\top,{[-\frac{1}{\sqrt{2}},\frac{1}{2\sqrt{2}}]}^\top$ are solutions.\\
        $\lambda=-\frac{1}{4}\Rightarrow {[\frac{1}{\sqrt{2}},\frac{1}{2\sqrt{2}}]}^\top,{[-\frac{1}{\sqrt{2}},-\frac{1}{2\sqrt{2}}]}^\top$ are solutions.
        \begin{align*}
            &L(\mathbf{x},\frac{1}{4})=\begin{bmatrix}
                \frac{1}{2} & 1\\
                1 & 2
            \end{bmatrix}\\
            &L(\mathbf{x},-\frac{1}{4})=\begin{bmatrix}
                -\frac{1}{2} & 1\\
                1 & -2
            \end{bmatrix}\\
            &T(\mathbf{x})=\{\mathbf{v}:[2x_1,8x_2]\mathbf{v}=0\}\\
            &T({[\frac{1}{\sqrt{2}},-\frac{1}{2\sqrt{2}}]}^\top)=T({[-\frac{1}{\sqrt{2}},\frac{1}{2\sqrt{2}}]}^\top)=span\{{[1,1]}^\top\}\\
            &T({[\frac{1}{\sqrt{2}},\frac{1}{2\sqrt{2}}]}^\top)=T({[-\frac{1}{\sqrt{2}},-\frac{1}{2\sqrt{2}}]}^\top)=span\{{[1,-1]}^\top\}
        \end{align*}
        For $\mathbf{v}\in span\{{[1,-1]}^\top\}, \mathbf{v}^\top L(\mathbf{x},-\frac{1}{4})\mathbf{v}<0$\\
        For $\mathbf{v}\in span\{{[1,1]}^\top\}, \mathbf{v}^\top L(\mathbf{x},\frac{1}{4})\mathbf{v}>0$\\
        Thus, ${[\frac{1}{\sqrt{2}},-\frac{1}{2\sqrt{2}}]}^\top,{[-\frac{1}{\sqrt{2}},\frac{1}{2\sqrt{2}}]}^\top$ are strict local maximizers and ${[\frac{1}{\sqrt{2}},\frac{1}{2\sqrt{2}}]}^\top,{[-\frac{1}{\sqrt{2}},-\frac{1}{2\sqrt{2}}]}^\top$ are strict local minimizers.
    \end{enumerate}
    \item [\textbf{20.3}] $l(\mathbf{x},\mathbf{\lambda})=(\mathbf{a}^\top\mathbf{x})(\mathbf{b}^\top\mathbf{x})+\lambda_1(x_1+x_2)+\lambda_2(x_2+x_3)=0$\\
    \begin{align*}
        &\nabla l=\begin{bmatrix}
            \nabla_\mathbf{x} l\\
            h(\mathbf{x})
        \end{bmatrix}=\begin{bmatrix}
            (\mathbf{ab}^\top+\mathbf{ba}^\top)\mathbf{x}+[\nabla h_1,\nabla h_2]\mathbf{\lambda}\\
            h(\mathbf{x})
        \end{bmatrix}\\
        &=\begin{bmatrix}
            \begin{bmatrix}
                0 & 1 & 0\\
                1 & 0 & 1\\
                0 & 1 & 0
            \end{bmatrix}\mathbf{x}+\begin{bmatrix}
                1 & 0\\
                1 & 1\\
                0 & 1
            \end{bmatrix}\mathbf{\lambda}\\
            h(\mathbf{x})
        \end{bmatrix}\\
        &=\begin{bmatrix}
            x_2+\lambda_1\\
            x_1+x_2+\lambda_1+\lambda_2\\
            x_2+\lambda_2\\
            x_1+x_2\\
            x_2+x_3
        \end{bmatrix}=\mathbf{0}
    \end{align*} 
    By inspection, $\mathbf{x}^*=\mathbf{0},\mathbf{\lambda}^*=\mathbf{0}$ solves the system uniquely.\\
    $L(\mathbf{x}^*,\mathbf{\lambda}^*)=\begin{bmatrix}
        0 & 1 & 0\\
        1 & 0 & 1\\
        0 & 1 & 0
    \end{bmatrix},\\
    T(\mathbf{x}^*)=span\{{[1,-1,1]}^\top\}$\\
    $\forall\mathbf{v}\in T(\mathbf{x}^*), \mathbf{v}^\top L(\mathbf{x}^*,\mathbf{\lambda}^*)\mathbf{v}=-4\alpha^2<0$,
    so $\mathbf{0}$ is a strict local maximizer.
    \item [\textbf{20.4}] $\nabla l(\mathbf{x}^\top,\mathbf{\lambda}^\top)=\begin{bmatrix}
        x_1+\lambda\\
        x_1+4+4\lambda\\
        h(\mathbf{x}^*)
    \end{bmatrix}=\mathbf{0}\\
    \Rightarrow x_1=\frac{4}{3}\Rightarrow\nabla f(\mathbf{x}^*)={[\frac{4}{3},\frac{16}{3}]}^\top$
    \item [\textbf{20.9}] $f(\mathbf{x})=\frac{\mathbf{x}^\top\begin{bmatrix}
            18 & -4\\
            -4 & 12
    \end{bmatrix}\mathbf{x}}{\mathbf{x}^\top\begin{bmatrix}
            2 & 0\\
            0 & 2
    \end{bmatrix}\mathbf{x}}$\\
    Observe that if $\mathbf{x}^*$ is a maximizer, $\alpha\mathbf{x}$ for some scalar $\alpha$ is also a maximizer because\\
    $f(\alpha\mathbf{x}^*)=\frac{{(\alpha\mathbf{x}^*)}^\top\begin{bmatrix}
        18 & -4\\
        -4 & 12
    \end{bmatrix}(\alpha\mathbf{x}^*)}{{(\alpha\mathbf{x}^*)}^\top\begin{bmatrix}
        2 & 0\\
        0 & 2
    \end{bmatrix}(\alpha\mathbf{x}^*)}\\
    =\frac{\alpha^2}{\alpha^2}\frac{\mathbf{x}^\top\begin{bmatrix}
        18 & -4\\
        -4 & 12
\end{bmatrix}\mathbf{x}}{\mathbf{x}^\top\begin{bmatrix}
        2 & 0\\
        0 & 2
\end{bmatrix}\mathbf{x}}=f(\mathbf{x}^*)$\\
It follows we can solve an equivalent problem to maximize $\mathbf{x}^\top\begin{bmatrix}
    18 & -4\\
    -4 & 12
\end{bmatrix}\mathbf{x}$ constrained to $\mathbf{x}^\top\begin{bmatrix}
    2 & 0\\
    0 & 2
\end{bmatrix}\mathbf{x}=1$\\
$l(\mathbf{x},\mathbf{\lambda})=\mathbf{x}^\top\begin{bmatrix}
    18 & -4\\
    -4 & 12
\end{bmatrix}\mathbf{x}+\mathbf{\lambda}(1-\mathbf{x}^\top\begin{bmatrix}
    2 & 0\\
    0 & 2
\end{bmatrix}\mathbf{x})$\\
$\nabla l_x=2(\begin{bmatrix}
    18 & -4\\
    -4 & 12
\end{bmatrix}-\mathbf{\lambda}\begin{bmatrix}
    2 & 0\\
    0 & 2
\end{bmatrix})\mathbf{x}=\mathbf{0}\\
\Leftrightarrow (\lambda \mathbf{I}_2-\begin{bmatrix}
    9 & -2\\
    -2 & 6
\end{bmatrix})\mathbf{x}=0$\\
which is equivalent to solving an eigenvalue eigenvector problem.\\
$\det(\lambda \mathbf{I}_2-\begin{bmatrix}
    9 & -2\\
    -2 & 6
\end{bmatrix})=(\lambda-10)(\lambda-5)$ $\lambda=10$ will give us a much larger value.\\
$\mathbf{x}^*=\frac{1}{\sqrt{10}}\begin{bmatrix}
    2\\
    -1
\end{bmatrix}$ gives us a solution that satisfies $h(\mathbf{x}^*)=0$.\\
Thus, any scalar multiple of $\mathbf{x}^*$ will give us a maximizer.
\item [\textbf{20.10}] Let $\mathbf{Q}_0=\frac{1}{2}(\begin{bmatrix}
    3 & 4\\
    0 & 3
\end{bmatrix}+\begin{bmatrix}
    3 & 0\\
    4 & 3
\end{bmatrix})$\\
$\nabla l(\mathbf{x}^*,\mathbf{\lambda}^*)=2\mathbf{Q}_0\mathbf{x}^*-2\mathbf{\lambda}^*\mathbf{x}=0$\\
which is equivalent to solving for $\mathbf{Q}_0$ eigenvalues and corresponding eigenvectors.\\
$\det\begin{bmatrix}
    \lambda-3 & -2\\
    -2 & \lambda-3
\end{bmatrix}=\lambda^2-6\lambda+5=(\lambda-1)(\lambda-5)$\\
Thus, the maximizers are the eigenvectors for $\lambda=5$ constrained to $\lVert x\rVert=1$ $[\frac{1}{\sqrt{2}},\frac{1}{\sqrt{2}}],[-\frac{1}{\sqrt{2}},-\frac{1}{\sqrt{2}}]$
\item [\textbf{20.14}] Suppose $\mathbf{x}^*=[1,1]$ is a solution to $\nabla l(\mathbf{x},\mathbf{\lambda})=\begin{bmatrix}
    a+2\lambda x_1\\
    b+2\lambda x_2\\
    x_1^2+x_2^2-2
\end{bmatrix}=\begin{bmatrix}
    0\\
    0\\
    0
\end{bmatrix}$.\\
It follows $a=-2\lambda$ and $b=-2\lambda$, so $a=b$.
\end{enumerate}
\end{document}