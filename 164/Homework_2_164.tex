\documentclass[10pt]{article}
\usepackage{graphicx}
\usepackage{amssymb}
\usepackage[fleqn]{amsmath}
\usepackage{nccmath}
\usepackage{cases}
\usepackage{hyperref}
\usepackage{multicol}
\usepackage{tikz}
\usepackage{pgfplots}
\usepackage{enumitem}
\pgfplotsset{compat=1.18}
\usepackage{float}

\title{\bf Math 164: Problem Set 2}
\date{1/17/2024}
\author{\bf Owen Jones}
\begin{document}
\maketitle
\begin{enumerate}
    \item [\textbf{3.2}] \begin{itemize}
        \item [\textbf{a.}] 
        $\begin{bmatrix}
            \mathbf{e_1} & \mathbf{e_2} & \mathbf{e_3}
        \end{bmatrix}
        \begin{bmatrix}
            1 & 2 & 4\\
            3 & -1 & 5\\
            -4 & 5 & 3 
        \end{bmatrix}=\begin{bmatrix}
            \mathbf{e_1'} & \mathbf{e_2'} & \mathbf{e_3'}
        \end{bmatrix}\\
        \Rightarrow \textbf{\textit{T}}=\begin{bmatrix}
            1 & 2 & 4\\
            3 & -1 & 5\\
            -4 & 5 & 3 
        \end{bmatrix}^{-1}=\frac{1}{42}\begin{bmatrix}
            28 & -14 & -14\\
            29 & -19 & -7\\
            -11 & 13 & 7
        \end{bmatrix}$
        \item [\textbf{b.}] 
        $\begin{bmatrix}
            \mathbf{e_1} & \mathbf{e_2} & \mathbf{e_3}
        \end{bmatrix}=\begin{bmatrix}
            \mathbf{e_1'} & \mathbf{e_2'} & \mathbf{e_3'}
        \end{bmatrix}\begin{bmatrix}
            1 & 2 & 3\\
            1 & -1 & 0\\
            3 & 4 & 5 
        \end{bmatrix}\\
        \Rightarrow \textbf{\textit{T}}=\begin{bmatrix}
            1 & 2 & 3\\
            1 & -1 & 0\\
            3 & 4 & 5 
        \end{bmatrix}$
    \end{itemize}
    \item [\textbf{3.3}] 
    $\begin{bmatrix}
        \mathbf{e_1} & \mathbf{e_2} & \mathbf{e_3}
    \end{bmatrix}
    =\begin{bmatrix}
        \mathbf{e_1'} & \mathbf{e_2'} & \mathbf{e_3'}
    \end{bmatrix}\begin{bmatrix}
        2 & 2 & 3\\
        1 & -1 & 0\\
        -1 & 2 & 1 
    \end{bmatrix}\\
    \Rightarrow \textbf{\textit{T}}=\begin{bmatrix}
        2 & 2 & 3\\
        1 & -1 & 0\\
        -1 & 2 & 1 
    \end{bmatrix}\\
    \Rightarrow \textbf{\textit{T}}^{-1}=\begin{bmatrix}
        1 & -4 & 3\\
        1 & -5 & -3\\
        -1 & 6 & 4 
    \end{bmatrix}\\
    \Rightarrow \textit{\textbf{\textit{B}}}=\textbf{\textit{T}}\textit{\textbf{\textit{A}}}\textbf{\textit{T}}^{-1}\\
    \Rightarrow \textit{\textbf{\textit{B}}}=\begin{bmatrix}
        3 & -10 & -8\\
        -1 & 8 & 4\\
        2 & -13 & -7 
    \end{bmatrix}$
    \item [\textbf{3.5}] $\det(\lambda I_4-\textbf{\textit{A}})
    =(\lambda+1)\det(\lambda I_3-\begin{bmatrix}
        1 & 0 & 0\\
        5 & 2 & 1\\
        1 & 0 & 3
    \end{bmatrix})\\
    =(\lambda+1)(\lambda-1)\det(\lambda I_2-\begin{bmatrix}
        2 & 1\\
        0 & 3
    \end{bmatrix})=(\lambda+1)(\lambda-1)(\lambda-2)(\lambda-3)$\\
    $\begin{bmatrix}
        3+1 & 0 & 0 & 0\\
        -1 & 3-1 & 0 & 0\\
        -2 & -5 & 3-2 & -1\\
        1 & -1 & 0 & 3-3
    \end{bmatrix}\begin{bmatrix}
        0\\
        0\\
        1\\
        1
    \end{bmatrix}=\begin{bmatrix}
        0\\
        0\\
        0\\
        0
    \end{bmatrix}$\\
    $\begin{bmatrix}
        2+1 & 0 & 0 & 0\\
        -1 & 2-1 & 0 & 0\\
        -2 & -5 & 2-2 & -1\\
        1 & -1 & 0 & 2-3
    \end{bmatrix}\begin{bmatrix}
        0\\
        0\\
        1\\
        0
    \end{bmatrix}=\begin{bmatrix}
        0\\
        0\\
        0\\
        0
    \end{bmatrix}$\\
    $x_1+2x_2=0\\
    2x_1+5x_2+3x_3+x_4=0\\
    x_1-x_2-4x_4=0\\
    \Rightarrow x_2+3x_3+x_4=0\\
    -3x_2-4x_4=0\\
    \Rightarrow 3x_3-\frac{1}{3}x_4$\\
    $\begin{bmatrix}
        -1+1 & 0 & 0 & 0\\
        -1 & -1-1 & 0 & 0\\
        -2 & -5 & -1-2 & -1\\
        1 & -1 & 0 & -1-3
    \end{bmatrix}\begin{bmatrix}
        24\\
        -12\\
        1\\
        9
    \end{bmatrix}=\begin{bmatrix}
        0\\
        0\\
        0\\
        0
    \end{bmatrix}$\\
    $x_1=0\\
    2x_1+5x_2+x_3+x_4=0\\
    x_1-x_2-x_4=0\\
    \Rightarrow 5x_2+x_3+x_4=0\\
    x_2+2x_4=0\\
    9x_4=x_3$\\
    $\begin{bmatrix}
        1+1 & 0 & 0 & 0\\
        -1 & 1-1 & 0 & 0\\
        -2 & -5 & 1-2 & -1\\
        1 & -1 & 0 & 1-3
    \end{bmatrix}\begin{bmatrix}
        0\\
        -2\\
        9\\
        1
    \end{bmatrix}=\begin{bmatrix}
        0\\
        0\\
        0\\
        0
    \end{bmatrix}$\\
    Giving us eigenvectors $v_1=\begin{bmatrix}
        0\\
        0\\
        1\\
        1
    \end{bmatrix},v_2=\begin{bmatrix}
        0\\
        0\\
        1\\
        0
    \end{bmatrix},v_3=\begin{bmatrix}
        24\\
        -12\\
        1\\
        9
    \end{bmatrix},v_4\begin{bmatrix}
        0\\
        -2\\
        9\\
        1
    \end{bmatrix}$ with corresponding eigenvalues $3,2,-1,1$ for \textbf{\textit{A}}.\\
    Let $[v_1,v_2,v_3,v_4]$ be our new basis for $\mathbb{R}^4$.
    $\textbf{\textit{T}}={[v_1,v_2,v_3,v_4]}^{-1}[e_1,e_2,e_3,e_4]={[v_1,v_2,v_3,v_4]}^{-1}$
    If $\textbf{\textit{B}}$ is our linear transformation with respect to basis $[v_1,v_2,v_3,v_4]$,
    then $\textbf{\textit{B}}
    =\textbf{\textit{T}}\textbf{\textit{A}}\textbf{\textit{T}}^{-1}\\
    ={[v_1,v_2,v_3,v_4]}^{-1}\textbf{\textit{A}}[v_1,v_2,v_3,v_4]
    ={[v_1,v_2,v_3,v_4]}^{-1}[\lambda_1v_1,\lambda_2v_2,\lambda_3v_3,\lambda_4v_4]
    =
    \begin{bmatrix}
        3 & 0 & 0 & 0\\
        0 & 2 & 0 & 0\\
        0 & 0 & -1 & 1\\
        0 & 0 & 0 & 1
    \end{bmatrix}$
    \item [\textbf{3.6}] We can rewrite the characteristic polynomial for $\textbf{\textit{I}}_\textbf{\textit{n}}-\textbf{\textit{A}}$,\\ 
    $\det(\lambda' \textbf{\textit{I}}_\textbf{\textit{n}}-(\textbf{\textit{I}}_\textbf{\textit{n}}-\textbf{\textit{A}}))$,
    as $\det((\lambda'-1) \textbf{\textit{I}}_\textbf{\textit{n}}+\textbf{\textit{A}})$.\\ 
    It follows $\det((\lambda'-1) \textbf{\textit{I}}_\textbf{\textit{n}}+\textbf{\textit{A}})={(-1)}^n\det((1-\lambda')\textbf{\textit{I}}_\textbf{\textit{n}}-\textbf{\textit{A}})$.
    If $\lambda_i$ $i=1\ldots n$ are the eigenvalues of \textbf{\textit{A}}, then $\det(\lambda\textbf{\textit{I}}_\textbf{\textit{n}}-\textbf{\textit{A}})=0$ if $\lambda=\lambda_i$ for $i=1\ldots n$.
    Take $\lambda_i'=1-\lambda_i\Rightarrow 
    {(-1)}^n\det((1-\lambda_i')\textbf{\textit{I}}_\textbf{\textit{n}}-\textbf{\textit{A}})\\
    ={(-1)}^n\det((1-(1-\lambda_i)')\textbf{\textit{I}}_\textbf{\textit{n}}-\textbf{\textit{A}})\\
    ={(-1)}^n\det(\lambda_i\textbf{\textit{I}}_\textbf{\textit{n}}-\textbf{\textit{A}})=0$.\\
    Hence, $\det(\lambda' \textbf{\textit{I}}_\textbf{\textit{n}}-(\textbf{\textit{I}}_\textbf{\textit{n}}-\textbf{\textit{A}}))=0$ if $\lambda'=1-\lambda_i\Leftrightarrow 1-\lambda_i$ are the eigenvalues of $\textbf{\textit{I}}_\textbf{\textit{n}}-\textbf{\textit{A}}$.
    \item[\textbf{3.8}] $\mathcal{N}(A)\triangleq\{\textbf{x}\in\mathbb{R}^3:\textbf{Ax}=0\}$. 
    $\det(0\textit{\textbf{I}}_\textit{\textbf{3}}-\textbf{A})=4(-2)-(-2)(4)=0$
    Row-reduce \textbf{A} to reduced row-echelon form:\\
    $\begin{bmatrix}
        4 & -2 & 0\\
        2 & 1 & -1\\
        2 & -3 & 1
    \end{bmatrix}
    =\begin{bmatrix}
        4 & -2 & 0\\
        0 & 0 & 0\\
        2 & -3 & 1
    \end{bmatrix} R_2=R_2+R_3-R_1,\\
    =\begin{bmatrix}
        2 & -1 & 0\\
        0 & 0 & 0\\
        2 & -3 & 1
    \end{bmatrix} R_1=\frac{1}{2}R_1,\\
    =\begin{bmatrix}
        2 & -1 & 0\\
        0 & 0 & 0\\
        0 & -2 & 1
    \end{bmatrix} R_3=R_3-R_1,\\
    \Rightarrow 2x_1-x_2=0, -2x_2+x_3=0$. Let $x_2=s\Rightarrow x_1=\frac{s}{2},x_3=2s$.\\
    Thus, $\mathcal{N}(A)=span(\begin{bmatrix}
        1\\
        2\\
        4
    \end{bmatrix})$.
    \item [\textbf{3.16}] All principal minors are nonnegative:\\ 
    $\det(2)=2,\\
    \det\begin{bmatrix}
        2 & 2\\
        2 & 2
    \end{bmatrix}=0,\\
    \det(\textit{\textbf{A}})=2\det\begin{bmatrix}
        2 & 2\\
        2 & 0
    \end{bmatrix}-2\det\begin{bmatrix}
        2 & 2\\
        2 & 0
    \end{bmatrix}+2\det\begin{bmatrix}
        2 & 2\\
        2 & 2
    \end{bmatrix}=0$.\\ 
    $\det(\lambda I_3-\textbf{\textit{A}})=(\lambda-2)\det\begin{bmatrix}
        \lambda-2 & 2\\
        -2 & \lambda
    \end{bmatrix}+2\det\begin{bmatrix}
        -2 & -2\\
        -2 & \lambda
    \end{bmatrix}-2\det\begin{bmatrix}
        -2 & \lambda-2\\
        -2 & 2
    \end{bmatrix}\\
    =(\lambda-2)(\lambda^2-2\lambda+4)+2(-2\lambda-4)-2(-8+2\lambda)\\
    =\lambda^3-2\lambda^2+4\lambda-2\lambda^2-4\lambda-8-4\lambda-8+16-4\lambda\\
    =\lambda^3-4\lambda^2-8\lambda$.\\
    $\det((-1) I_3-\textbf{\textit{A}})=3,\det((-2) I_3-\textbf{\textit{A}})=-8$, so by IVT, \textbf{\textit{A}} must have a negative eigenvalue. Thus, \textbf{\textit{A}} cannot be positive semidefinite.
    \item [\textbf{3.17}] \begin{enumerate}
        \item \textbf{\textit{Q}} is said to be indefinite if the matrix takes on both positive and negative eigenvalues. 
        $\det(\lambda I_3-\textbf{\textit{Q}})=\lambda\det\begin{bmatrix}
            \lambda & -1\\
            -1 & \lambda
        \end{bmatrix}+1\det\begin{bmatrix}
            -1 & -1\\
            -1 & \lambda
        \end{bmatrix}-1\det\begin{bmatrix}
            -1 & \lambda\\
            -1 & -1
        \end{bmatrix}=\lambda(\lambda-1)(\lambda+1)-2(\lambda+1)=(\lambda^2-\lambda-2)(\lambda+1)={(\lambda+1)}^2(\lambda-2)\Rightarrow$ \textbf{\textit{Q}} has both positive and negative eigenvalues. 
        Thus, \textbf{\textit{Q}} is indefinite.
        \item $\mathbf{x}^\top\textbf{\textit{Q}}\mathbf{x}=\langle \mathbf{x},\begin{bmatrix}
            x_2+x_3\\
            x_1+x_3\\
            x_1+x_2
        \end{bmatrix}\rangle=\langle \mathbf{x},\begin{bmatrix}
            -x_1\\
            -x_2\\
            -x_3
        \end{bmatrix}\rangle=\langle \mathbf{x},-\mathbf{x}\rangle=-{\lVert \mathbf{x}\rVert}^2< 0$ for all $\mathbf{x}\neq0$, so \textbf{\textit{Q}} is negative definite. 
    \end{enumerate}
    \item [\textbf{3.18}] \begin{enumerate}
        \item $f(x_1,x_2,x_3)=x^2_2$. For $x\in\mathbb{R},x^2\ge0\Rightarrow x^2_2\ge0$ for all $\mathbf{x}\in\mathbb{R}^3$. Thus, $f$ is positive semidefinite.
        \item $f(x_1,x_2,x_3)=x_1^2+2x_2^2-x_1x_3=\mathbf{x}^\top\begin{bmatrix}
            1 & 0 & -\frac{1}{2}\\
            0 & 2 & 0\\
            -\frac{1}{2} & 0 & 0
        \end{bmatrix}\mathbf{x}$. 
        Let $\textbf{\textit{A}}=\begin{bmatrix}
            1 & 0 & -\frac{1}{2}\\
            0 & 2 & 0\\
            -\frac{1}{2} & 0 & 0
        \end{bmatrix}$.\\ 
        $\det(\lambda I_3-\textbf{\textit{A}})=\det\begin{bmatrix}
            \lambda-1 & 0 & \frac{1}{2}\\
            0 & \lambda-2 & 0\\
            0 & 0 & \lambda-\frac{1}{4(\lambda-1)}
        \end{bmatrix}=(\lambda-2)(\lambda^2-\lambda-\frac{1}{4})=(\lambda-2)(\lambda-\frac{1}{2}(1-\sqrt{2}))(\lambda-\frac{1}{2}(1+\sqrt{2}))$. 
        Because \textbf{\textit{A}} has positive and negative eigenvalues, \textbf{\textit{A}} is indefinite.
        \item $f(x_1,x_2,x_3)=x_1^2+x^2_3+2x_1x_2+2x_1x_3+2x_2x_3=\mathbf{x}^\top\begin{bmatrix}
            1 & 1 & 1\\
            1 & 0 & 1\\
            1 & 1 & 1
        \end{bmatrix}\mathbf{x}$. Let $\textbf{\textit{A}}=\begin{bmatrix}
            1 & 1 & 1\\
            1 & 0 & 1\\
            1 & 1 & 1
        \end{bmatrix}$. $\det(\lambda I_3-\textbf{\textit{A}})=(\lambda-1)\det\begin{bmatrix}
            \lambda & -1\\
            -1 & \lambda-1
        \end{bmatrix}+\det\begin{bmatrix}
            -1 & -1\\
            -1 & \lambda-1
        \end{bmatrix}-\det\begin{bmatrix}
            -1 & \lambda\\
            -1 & -1
        \end{bmatrix}=\lambda^3-2\lambda^2-2\lambda=\lambda(\lambda-1+\sqrt{3})(\lambda-1-\sqrt{3})$. 
        Because \textbf{\textit{A}} has positive and negative eigenvalues, \textbf{\textit{A}} is indefinite.
    \end{enumerate}
    \item [\textbf{3.20}] $f(x_1,x_2,x_3)=x_1^2+x_2^2+5x_3^2+2\xi x_1x_2-2x_1x_3+4x_2x_3=\mathbf{x}^\top
    \begin{bmatrix}
        1 & \xi & -1\\
        \xi & 1 & 2\\
        -1 & 2 & 5
    \end{bmatrix}
    \mathbf{x}$\\
    Suffices to find values for $\xi$ s.t 
    $\det\begin{bmatrix}
        1 & \xi\\
        \xi & 1
    \end{bmatrix}>0$ and 
    $\det\begin{bmatrix}
        1 & \xi & -1\\
        \xi & 1 & 2\\
        -1 & 2 & 5
    \end{bmatrix}>0$\\
    $\det\begin{bmatrix}
        1 & \xi\\
        \xi & 1
    \end{bmatrix}=1-\xi^2\Rightarrow -1<\xi<1$\\
    $\det\begin{bmatrix}
        1 & \xi & -1\\
        \xi & 1 & 2\\
        -1 & 2 & 5
    \end{bmatrix}=
    1\det\begin{bmatrix}
        1 & 2\\
        2 & 5
    \end{bmatrix}
    -\xi\det\begin{bmatrix}
        \xi & 2\\
        -1 & 5
    \end{bmatrix}
    -1\begin{bmatrix}
        \xi & 1\\
        -1 & 2
    \end{bmatrix}\\
    =1-\xi(5\xi+2)-(2\xi+1)=1-5\xi^2-2\xi-2\xi-1=-5\xi^2-4\xi\Rightarrow \xi(5\xi+4)<0\Rightarrow -\frac{4}{5}<\xi<0$.\\
    Thus, $-\frac{4}{5}<\xi<0$ for the quadratic form to be positive definite.
    \item [\textbf{3.21}] We are given \textbf{\textit{Q}} is a symmetric positive definite matrix.
    \begin{enumerate}
        \item [Positivity:] Given that \textbf{\textit{Q}} is positive definite $\mathbf{x}^\top\textbf{\textit{Q}}\mathbf{x}>0$ for all $\mathbf{x}\neq0$, and $\mathbf{x}^\top\textbf{\textit{Q}}\mathbf{x}=0$ if $\mathbf{x}=0$.  
        \item [Symmetry:] ${\langle \mathbf{x},\mathbf{y}\rangle}_\textbf{\textit{Q}}=\mathbf{x}^\top\textbf{\textit{Q}}\mathbf{y}={(\textbf{\textit{Q}}^\top\mathbf{x})}^\top\mathbf{y}={(\textbf{\textit{Q}}\mathbf{x})}^\top\mathbf{y}=\langle\textbf{\textit{Q}}\mathbf{x},\mathbf{y}\rangle=\langle\mathbf{y},\textbf{\textit{Q}}\mathbf{x}\rangle=\mathbf{y}^\top\textbf{\textit{Q}}\mathbf{x}={\langle\mathbf{y},\mathbf{x}\rangle}_\textbf{\textit{Q}}$
        \item [Additivity:] ${\langle \mathbf{x}+\mathbf{y},\mathbf{z}\rangle}_\textbf{\textit{Q}}={\langle \mathbf{z},\mathbf{x}+\mathbf{y}\rangle}_\textbf{\textit{Q}}={\mathbf{z}}^\top\textbf{\textit{Q}}(\mathbf{x}+\mathbf{y})={(\textbf{\textit{Q}}\mathbf{z})}^\top(\mathbf{x}+\mathbf{y})\\
        =\langle\textbf{\textit{Q}}\mathbf{z},\mathbf{x}+\mathbf{y}\rangle=\langle\textbf{\textit{Q}}\mathbf{z},\mathbf{x}\rangle+\langle\textbf{\textit{Q}}\mathbf{z},\mathbf{y}\rangle=\mathbf{z}^\top\textbf{\textit{Q}}\mathbf{x}+\mathbf{z}^\top\textbf{\textit{Q}}\mathbf{y}\\
        ={\langle \mathbf{z},\mathbf{x}\rangle}_\textbf{\textit{Q}}+{\langle \mathbf{z},\mathbf{y}\rangle}_\textbf{\textit{Q}}={\langle \mathbf{x},\mathbf{z}\rangle}_\textbf{\textit{Q}}+{\langle \mathbf{y},\mathbf{z}\rangle}_\textbf{\textit{Q}}$
        \item [Homogeneity:] ${\langle \mathbf{rx},\mathbf{y}\rangle}_\textbf{\textit{Q}}={(\mathbf{rx})}^\top\textbf{\textit{Q}}\mathbf{y}=\mathbf{r}{(\mathbf{x})}^\top\textbf{\textit{Q}}\mathbf{y}=\mathbf{r}{\langle \mathbf{x},\mathbf{y}\rangle}_\textbf{\textit{Q}}$
    \end{enumerate}
\end{enumerate}
\end{document}