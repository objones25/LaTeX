\documentclass[10pt]{article}
\usepackage{graphicx}
\usepackage{amssymb}
\usepackage[fleqn]{amsmath}
\usepackage{nccmath}
\usepackage{cases}
\usepackage{hyperref}
\usepackage{multicol}
\usepackage{tikz}
\usepackage{pgfplots}
\usepackage{enumitem}
\usepackage{pdfpages}
\pgfplotsset{compat=1.18}
\usepackage{float}
\DeclareMathOperator*{\argmin}{arg\,min}

\title{\bf Math 164: Problem Set 7}
\date{2/23/2024}
\author{\bf Owen Jones}
\begin{document}
\maketitle
\begin{enumerate}
    \item [\textbf{9.1}] \begin{enumerate}
        \item $\mathbf{x}^{(k+1)}=\mathbf{x}^{(k)}-{\mathbf{F}(\mathbf{x}^{(k)})}^{-1}\mathbf{g}^{(k)}\\
        =\mathbf{x}^{(k)}-\frac{4{(\mathbf{x}^{(k)}-\mathbf{x_0})}^3}{12{(\mathbf{x}^{(k)}-\mathbf{x_0})}^2}\\
        \Rightarrow \mathbf{x}^{(k+1)}=\frac{2}{3}\mathbf{x}^{(k)}+\frac{1}{3}\mathbf{x_0}$\\
        subtracting $\mathbf{x_0}$ from both sides $\mathbf{x}^{(k+1)}-\mathbf{x_0}=\frac{2}{3}(\mathbf{x}^{(k)}-\mathbf{x_0})$
        \item From part (a), $\mathbf{y}^{(k+1)}=\lvert\mathbf{x}^{(k+1)}-\mathbf{x_0}\rvert=\frac{2}{3}\lvert\mathbf{x}^{(k)}-\mathbf{x_0}\rvert=\frac{2}{3}\mathbf{y}^{(k)}$, so the sequence satisfies $\mathbf{y}^{(k+1)}=\frac{2}{3}\mathbf{y}^{(k)}$.
        \item From part (b) we have $\mathbf{y}^{(k+1)}=\frac{2}{3}\mathbf{y}^{(k)}\Rightarrow \mathbf{y}^{(k)}\rightarrow 0$. Hence, $\mathbf{x}^{(k)}\rightarrow \mathbf{x_0}$ for any $\mathbf{x}^{(0)}$.
        \item From part (c), $\mathbf{x}^{(k)}\rightarrow \mathbf{x_0}$ for any $\mathbf{x}^{(0)}$. 
        From part (b), $\displaystyle \lim_{k\rightarrow \infty}\frac{\lvert\mathbf{x}^{(k+1)}-\mathbf{x_0}\rvert}{\lvert\mathbf{x}^{(k)}-\mathbf{x_0}\rvert}=\lim_{k\rightarrow \infty}\frac{\mathbf{y}^{(k+1)}}{\mathbf{y}^{(k)}}=\frac{2}{3}>0$. 
        Thus, the order of convergence is linear.
        \item The theorem assumes $\mathbf{F}(\mathbf{\mathbf{x}^*})$ is invertible, but $\mathbf{x}^*=\mathbf{x_0}$.\\
         Thus, $F(\mathbf{x}^*)=12{(\mathbf{x}^*-\mathbf{x_0})}^2=0$, so $\mathbf{F}(\mathbf{\mathbf{x}^*})$ is singular.
    \end{enumerate}
    \item [\textbf{9.3}] \begin{enumerate}
        \item $\mathbf{x}^{(k+1)}=\mathbf{x}^{(k)}-{\mathbf{F}(\mathbf{x}^{(k)})}^{-1}\mathbf{g}^{(k)}\\
        =\mathbf{x}^{(k)}-\frac{\frac{4}{3}{(\mathbf{x}^{(k)})}^\frac{1}{3}}{\frac{4}{9}{(\mathbf{x}^{(k)})}^\frac{-2}{3}}=-2\mathbf{x}^{(k)}$
        \item From part (a) we have $\mathbf{x}^{(k)}={(-2)}^{k}\mathbf{x_0}$ which diverges for any $\mathbf{x_0}\neq0$.
    \end{enumerate}
    \item [\textbf{9.4}]\begin{enumerate}
        \item $f(x)\ge 0$ $\forall x\in\mathbb{R}^2$, so it suffices to show $x={[1,1]}^\top \Leftrightarrow f(x)=0$.\\
        $f({[1,1]}^\top)=0$ by plugging in numbers\\
        $f(x)=0\Leftrightarrow x_2-x_1^2=0,{(1-x_1)}^2=0\Rightarrow x_1=1,x_2=x_1^2\Rightarrow x={[1,1]}^\top$
        \item \begin{align*}
            & \nabla f(x)={[400x_1^3-400x_1x_2+2(x_1-1),200(x_2-x_1^2)]}^\top\\
            & F(x)=\begin{bmatrix}
                1200x_1^2-400x_2+2 & -400x_1\\
                -400x_1 & 200
            \end{bmatrix}\\
            &\Rightarrow {F(x)}^{-1}=\frac{1}{80000(x_1^2-x_2)+400}\begin{bmatrix}
                200 & 400x_1\\
                400x_1 & 1200x_1^2-400x_2+2
            \end{bmatrix}\\
            & \mathbf{x}^{(1)}=\mathbf{x}^{(0)}-\frac{1}{400}\begin{bmatrix}
                200 & 0\\
                0 & 2
            \end{bmatrix}\begin{bmatrix}
                -2\\
                0
            \end{bmatrix}=\begin{bmatrix}
                1\\
                0
            \end{bmatrix}\\
            & \mathbf{x}^{(2)}=\mathbf{x}^{(1)}-\frac{1}{80400}\begin{bmatrix}
                200 & 400\\
                400 & 1202
            \end{bmatrix}\begin{bmatrix}
                400\\
                -200
            \end{bmatrix}=\begin{bmatrix}
                1\\
                1
            \end{bmatrix}
        \end{align*}
        \item \begin{align*}
            &\mathbf{x}^{(1)}=\mathbf{x}^{(0)}-0.05\cdot\begin{bmatrix}
                -2\\
                0
            \end{bmatrix}=\begin{bmatrix}
                0.1\\
                0
            \end{bmatrix}\\
            & \mathbf{x}^{(2)}=\mathbf{x}^{(1)}-0.05\cdot\begin{bmatrix}
                -1.4\\
                -2
            \end{bmatrix}=\begin{bmatrix}
                0.17\\
                0.1
            \end{bmatrix}
        \end{align*}
    \end{enumerate}
    \item [\textbf{9.5}] Because the case $\mathbf{x}_0=\mathbf{x}^*$ is trivial, assume $\mathbf{x}_0\neq\mathbf{x}^*$\\
    From standard Newton's method $\min f(\mathbf{x})=f(\mathbf{x}^*)=f(\mathbf{x}_0-{\mathbf{F}(\mathbf{x}_0)}^{-1}\mathbf{g}^{(k)})$.\\
    It follows $f(\mathbf{x}_0-{\mathbf{F}(\mathbf{x}_0)}^{-1}\mathbf{g}^{(k)})\le f(\mathbf{x}_0-\alpha{\mathbf{F}(\mathbf{x}_0)}^{-1}\mathbf{g}^{(k)})$ $\forall \alpha\ge 0$\\
    Thus, $alpha_0=\underset{\alpha\ge 0}{\argmin}f(\mathbf{x}_0-\alpha{\mathbf{F}(\mathbf{x}_0)}^{-1}\mathbf{g}^{(k)})=1$.\\
    Hence, $f(\mathbf{x}_0-\alpha_0{\mathbf{F}(\mathbf{x}_0)}^{-1}\mathbf{g}^{(k)})$ is equivalent to the standard Newton algorithm, so it also converges in a single step.
    \item [\textbf{10.1}] We will show by induction that the set $\{\mathbf{d}^{(0)},\ldots,\mathbf{d}^{(n-1)}\}$ is $\mathbf{Q}$-conjugate.\\
    The base case $k=0$ is trivial because there is only one vector in the set.\\
    Induction Hypothesis: Assume for some $k<n-1$ the set $\{\mathbf{d}^{(0)},\ldots,\mathbf{d}^{(k)}\}$ is $\mathbf{Q}$-conjugate.\\
    Induction step: Fix some $j=0\ldots k$. WTS ${\mathbf{d}^{(k+1)}}^\top\mathbf{Q}\mathbf{d}^{(j)}=0$\\
    $\displaystyle{\mathbf{d}^{(k+1)}}^\top\mathbf{Q}\mathbf{d}^{(j)}={\mathbf{p}^{(k+1)}}^\top\mathbf{Q}\mathbf{d}^{(j)}-\sum_{i=1}^{k}\frac{{\mathbf{p}^{(k+1)}}^\top\mathbf{Q}\mathbf{d}^{(i)}}{{\mathbf{d}^{(i)}}^\top\mathbf{Q}\mathbf{d}^{(i)}}\mathbf{d}^{(i)}\mathbf{Q}\mathbf{d}^{(j)}$\\
    By induction hypothesis: $\{\mathbf{d}^{(0)},\ldots,\mathbf{d}^{(k)}\}$ is $\mathbf{Q}$-conjugate, so $\mathbf{d}^{(i)}\mathbf{Q}\mathbf{d}^{(j)}$ $\forall i\neq j$.\\
    Thus $\displaystyle{\mathbf{d}^{(k+1)}}^\top\mathbf{Q}\mathbf{d}^{(j)}={\mathbf{p}^{(k+1)}}^\top\mathbf{Q}\mathbf{d}^{(j)}-\frac{{\mathbf{p}^{(k+1)}}^\top\mathbf{Q}\mathbf{d}^{(j)}}{{\mathbf{d}^{(j)}}^\top\mathbf{Q}\mathbf{d}^{(j)}}\mathbf{d}^{(j)}\mathbf{Q}\mathbf{d}^{(j)}\\
    ={\mathbf{p}^{(k+1)}}^\top\mathbf{Q}\mathbf{d}^{(j)}-{\mathbf{p}^{(k+1)}}^\top\mathbf{Q}\mathbf{d}^{(j)}=0$\\
    Hence, by induction, the set $\{\mathbf{d}^{(0)},\ldots,\mathbf{d}^{(k)}\}$ is $\mathbf{Q}$-conjugate.
    \item [\textbf{10.4}]\begin{enumerate}
        \item Because $\mathbf{Q}$ is symmetric, there exists an orthogonal eigenbasis $\{\mathbf{d}^{(1)},\ldots,\mathbf{d}^{(n)}\}$ for $\mathbf{Q}$ with real eigenvalues.
        It follows for any $i\neq j$ ${\mathbf{d}^{(i)}}^\top\mathbf{Q}\mathbf{d}^{(j)}=\lambda_j {\mathbf{d}^{(i)}}^\top\mathbf{d}^{(j)}=0$. Thus, $\{\mathbf{d}^{(1)},\ldots,\mathbf{d}^{(n)}\}$ is $\mathbf{Q}$-conjugate.
        \item Let $\lambda_i:=\frac{{\mathbf{d}^{(i)}}^\top\mathbf{Q}\mathbf{d}^{(i)}}{{\mathbf{d}^{(i)}}^\top \mathbf{d}^{(i)}}$. 
        $\{\mathbf{d}^{(1)},\ldots,\mathbf{d}^{(n)}\}$ must be linearly independent because the set is $\mathbf{Q}$-conjugate. 
        Let $\mathbf{D}=\begin{bmatrix}
           {\mathbf{d}^{(1)}}^\top\\
           \vdots\\
           {\mathbf{d}^{(n)}}^\top 
        \end{bmatrix}$ which is invertible because $\{\mathbf{d}^{(1)},\ldots,\mathbf{d}^{(n)}\}$ is linearly independent.
        Because the set $\{\mathbf{d}^{(1)},\ldots,\mathbf{d}^{(n)}\}$ is assumed to be orthogonal, $\lambda_i {\mathbf{d}^{(i)}}^\top\mathbf{d}^{(j)}=0$ $\forall i\neq j$.
        Because the set $\{\mathbf{d}^{(1)},\ldots,\mathbf{d}^{(n)}\}$ is $\mathbf{Q}$-conjugate, we have ${\mathbf{d}^{(i)}}^\top\mathbf{Q}\mathbf{d}^{(j)}=0$ $\forall i\neq j$.
        Thus $\lambda_i {\mathbf{d}^{(i)}}^\top\mathbf{d}^{(j)}={\mathbf{d}^{(i)}}^\top\mathbf{Q}\mathbf{d}^{(j)}$. 
        By how we defined $\lambda_i$, we have $\lambda_i {\mathbf{d}^{(i)}}^\top\mathbf{d}^{(i)}={\mathbf{d}^{(i)}}^\top\mathbf{Q}\mathbf{d}^{(i)}$. 
        It follows $\mathbf{D}\mathbf{Q}\mathbf{d}^{(i)}=\mathbf{D}(\lambda_i\mathbf{d}^{(i)})$ $\forall i=1\ldots n$ because\\
        $\begin{bmatrix}
            {\mathbf{d}^{(1)}}^\top\mathbf{Q}\mathbf{d}^{(i)}\\
            \vdots\\
            {\mathbf{d}^{(n)}}^\top\mathbf{Q}\mathbf{d}^{(i)} 
         \end{bmatrix}
         =\begin{bmatrix}
            \lambda_i{\mathbf{d}^{(1)}}^\top\mathbf{d}^{(i)}\\
            \vdots\\
            \lambda_i{\mathbf{d}^{(n)}}^\top\mathbf{d}^{(i)} 
         \end{bmatrix}$.\\
         Because $\mathbf{D}$ is invertible, we have $\mathbf{Q}\mathbf{d}^{(i)}=\lambda_i\mathbf{d}^{(i)}$  $\forall i=1\ldots n$.
         Hence each vector in the set $\{\mathbf{d}^{(1)},\ldots,\mathbf{d}^{(n)}\}$ is an eigenvector of $\mathbf{Q}$.
    \end{enumerate}
    \item [\textbf{10.5}] Premultiplying both sides by ${\mathbf{d}^{(k)}}^\top\mathbf{Q}$ we obtain\\ 
    ${\mathbf{d}^{(k)}}^\top\mathbf{Q}\mathbf{d}^{(k+1)}=\gamma_k{\mathbf{d}^{(k)}}^\top\mathbf{Q}\mathbf{g}^{(k+1)}+{\mathbf{d}^{(k)}}^\top\mathbf{Q}\mathbf{d}^{(k)}$.\\
    Using the fact $\mathbf{d}^{(k)}$ and $\mathbf{d}^{(k+1)}$ are $\mathbf{Q}$-conjugate ${\mathbf{d}^{(k)}}^\top\mathbf{Q}\mathbf{d}^{(k+1)}=0$.\\
    Thus, $\gamma_k=-\frac{{\mathbf{d}^{(k)}}^\top\mathbf{Q}\mathbf{d}^{(k)}}{{\mathbf{d}^{(k)}}^\top\mathbf{Q}\mathbf{g}^{(k+1)}}$
    \item [\textbf{10.7}] $\phi(\alpha)=\frac{1}{2}{(\mathbf{x}_0+\mathbf{D}\alpha)}^\top\mathbf{Q}(\mathbf{x}_0+\mathbf{D}\alpha)-{(\mathbf{x}_0+\mathbf{D}\alpha)}^\top\mathbf{b}\\
    =\frac{1}{2}\mathbf{x}_0^\top\mathbf{Q}\mathbf{x}_0+\alpha^\top\mathbf{D}^\top\mathbf{Q}\mathbf{x}_0+\frac{1}{2}\alpha^\top\mathbf{D}^\top\mathbf{Q}\mathbf{D}\alpha-\alpha^\top\mathbf{D}^\top\mathbf{b}-\mathbf{x}_0^\top\mathbf{b}\\
    =\frac{1}{2}\alpha^\top\mathbf{D}^\top\mathbf{Q}\mathbf{D}\alpha-\alpha^\top(\mathbf{D}^\top\mathbf{b}-\mathbf{D}^\top\mathbf{Q}\mathbf{x}_0)+(\frac{1}{2}\mathbf{x}_0^\top\mathbf{Q}\mathbf{x}_0-\mathbf{x}_0^\top\mathbf{b})$.\\
    Since we have shown $\phi(\alpha)$ is a quadratic function on $\mathbb{R}^r$, it suffices to show $\mathbf{D}^\top\mathbf{Q}\mathbf{D}>0$.\\
    Since $\mathbf{Q}>0$, $\alpha^\top\mathbf{D}^\top\mathbf{Q}\mathbf{D}\alpha={(\mathbf{D}\alpha)}^\top\mathbf{Q}(\mathbf{D}\alpha)\ge0$, and $\alpha^\top\mathbf{D}^\top\mathbf{Q}\mathbf{D}\alpha=0$ iff $\mathbf{D}\alpha=0$.\\
    Since $rank(\mathbf{D})=r$ and $\alpha\in\mathbb{R}^r$, $\mathbf{D}\alpha=0$ iff $\alpha=0$. Thus, $\alpha^\top\mathbf{D}^\top\mathbf{Q}\mathbf{D}\alpha=0$ iff $\alpha=0$.
    Thus, $\mathbf{D}^\top\mathbf{Q}\mathbf{D}>0$.
\end{enumerate}
\end{document}