\documentclass[10pt]{article}
\usepackage{graphicx}
\usepackage{amssymb}
\usepackage[fleqn]{amsmath}
\usepackage{nccmath}
\usepackage{cases}
\usepackage{hyperref}
\usepackage{multicol}
\usepackage{tikz}
\usepackage{pgfplots}
\usepackage{enumitem}
\pgfplotsset{compat=1.18}
\usepackage{float}
\DeclareMathOperator*{\argmin}{arg\,min}

\title{\bf Math 164: Problem Set 3}
\date{1/26/2024}
\author{\bf Owen Jones}
\begin{document}
\maketitle
\begin{enumerate}
    \item [\textbf{5.3}] \begin{enumerate}
        \item $f(\mathbf{x})=(\mathbf{a}^\top \mathbf{x})(\mathbf{b}^\top \mathbf{x})\\
        \frac{\partial}{\partial \mathbf{x}_i}(\mathbf{a}^\top \mathbf{x})=\frac{\partial}{\partial\mathbf{x}_i}\langle\mathbf{a},\mathbf{x}\rangle=\mathbf{a}_i\\
        \text{ and similarly }\frac{\partial}{\partial \mathbf{x}_i}(\mathbf{b}^\top \mathbf{x})=\frac{\partial}{\partial\mathbf{x}_i}\langle\mathbf{b},\mathbf{x}\rangle=\mathbf{b}_i\\
        \text{ so by the product rule }\frac{\partial}{\partial \mathbf{x}_i}(\mathbf{a}^\top \mathbf{x})(\mathbf{b}^\top \mathbf{x})=(\mathbf{a}^\top \mathbf{x})\mathbf{b}_i+(\mathbf{b}^\top \mathbf{x})\mathbf{a}_i\\
        \Rightarrow \nabla f(\mathbf{x})=\begin{bmatrix}
            (\mathbf{a}^\top \mathbf{x})\mathbf{b}_1+(\mathbf{b}^\top \mathbf{x})\mathbf{a}_1\\
            (\mathbf{a}^\top \mathbf{x})\mathbf{b}_2+(\mathbf{b}^\top \mathbf{x})\mathbf{a}_2\\
            \vdots\\
            (\mathbf{a}^\top \mathbf{x})\mathbf{b}_n+(\mathbf{b}^\top \mathbf{x})\mathbf{a}_n
        \end{bmatrix}\\
        \Rightarrow \nabla f(\mathbf{x})=(\mathbf{b}\mathbf{a}^\top+\mathbf{a}\mathbf{b}^\top)\mathbf{x}$
    \item $\mathbf{F(x)}=\begin{bmatrix}
        \frac{\partial^2 f}{\partial \mathbf{x}_1^2} & \frac{\partial^2 f}{\partial \mathbf{x}_2 \partial \mathbf{x}_1} & \cdots & \frac{\partial^2 f}{\partial \mathbf{x}_n \partial \mathbf{x}_1}\\
        \frac{\partial^2 f}{\partial \mathbf{x}_1\partial \mathbf{x}_2} & \frac{\partial^2 f}{\partial \mathbf{x}_2^2} & \cdots & \frac{\partial^2 f}{\partial \mathbf{x}_n \partial \mathbf{x}_2}\\
        \vdots & \vdots & \ddots & \vdots\\
        \frac{\partial^2 f}{\partial \mathbf{x}_1\partial \mathbf{x}_n} & \frac{\partial^2 f}{\partial \mathbf{x}_2 \partial \mathbf{x}_n} & \cdots & \frac{\partial^2 f}{\partial \mathbf{x}_n^2}
    \end{bmatrix}\\
    \frac{\partial}{\partial \mathbf{x}_i}(\mathbf{a}^\top \mathbf{x})(\mathbf{b}^\top \mathbf{x})=(\mathbf{a}^\top \mathbf{x})\mathbf{b}_i+(\mathbf{b}^\top \mathbf{x})\mathbf{a}_i\Rightarrow \frac{\partial^2}{\partial \mathbf{x}_i\partial \mathbf{x}_j}(\mathbf{a}^\top \mathbf{x})(\mathbf{b}^\top \mathbf{x})\\
    =\mathbf{a}_j\mathbf{b}_i+\mathbf{b}_j\mathbf{a}_i\\
    \Rightarrow \mathbf{F(x)}=\begin{bmatrix}
        2\mathbf{a}_1\mathbf{b}_1 & \mathbf{a}_1\mathbf{b}_2+\mathbf{a}_2\mathbf{b}_1 & \cdots & \mathbf{a}_1\mathbf{b}_n+\mathbf{a}_n\mathbf{b}_1\\
        \mathbf{a}_2\mathbf{b}_1+\mathbf{a}_1\mathbf{b}_2 & 2\mathbf{a}_2\mathbf{b}_2 & \cdots & \mathbf{a}_2\mathbf{b}_n+\mathbf{a}_n\mathbf{b}_2\\
        \vdots & \vdots & \ddots & \vdots\\
        \mathbf{a}_n\mathbf{b}_1+\mathbf{a}_1\mathbf{b}_n & \mathbf{a}_n\mathbf{b}_2+\mathbf{a}_2\mathbf{b}_n& \cdots & 2\mathbf{a}_n\mathbf{b}_n
    \end{bmatrix}\\
    =\mathbf{b}\mathbf{a}^\top+\mathbf{a}\mathbf{b}^\top$
    \end{enumerate}
    \item [\textbf{5.5}] $\frac{\partial}{\partial s}\mathbf{f(g(s,t))}=\frac{\partial \mathbf{f}}{\partial \mathbf{g}_1}\frac{\partial \mathbf{g}_1}{\partial s}+\frac{\partial \mathbf{f}}{\partial \mathbf{g}_2}\frac{\partial \mathbf{g}_2}{\partial s}\\
    =\frac{2s+t}{2}\cdot 4+\frac{4s+3t}{2}\cdot 2=8s+5t$\\
    $\frac{\partial}{\partial t}\mathbf{f(g(s,t))}=\frac{\partial \mathbf{f}}{\partial \mathbf{g}_1}\frac{\partial \mathbf{g}_1}{\partial t}+\frac{\partial \mathbf{f}}{\partial \mathbf{g}_2}\frac{\partial \mathbf{g}_2}{\partial t}\\
    =\frac{2s+t}{2}\cdot 3+\frac{4s+3t}{2}\cdot 1=5s+3t$
    \item [\textbf{5.10}] \begin{enumerate}
        \item $f(\mathbf{x})=f(\mathbf{x_0})+Df(\mathbf{x_0})(\mathbf{x}-\mathbf{x_0})+\frac{1}{2}{(\mathbf{x}-\mathbf{x_0})}^\top D^2f(\mathbf{x_0})(\mathbf{x}-\mathbf{x_0})+R_3\\
        =1\cdot e^{-0}+0+1
        +[e^{-0},-1\cdot e^{-0}+1]\begin{bmatrix}
            x_1-1\\
            x_2-0
        \end{bmatrix}\\
        +\frac{1}{2}[x_1-1,x_2-0]\begin{bmatrix}
            0 & -e^{-0}\\
            -e^{-0} & 1\cdot e^{-0}
        \end{bmatrix}\begin{bmatrix}
            x_1-1\\
            x_2-0
        \end{bmatrix}+R_3\\
        =2+(x_1-1)+\frac{1}{2}[-x_2,1-x_1+x_2]\begin{bmatrix}
            x_1-1\\
            x_2
        \end{bmatrix}+R_3\\
        =1+x_1+(1-x_1)(1-x_2)+\frac{x^2_2}{2}+R_3$
        \item $f(\mathbf{x})=f(\mathbf{x_0})+Df(\mathbf{x_0})(\mathbf{x}-\mathbf{x_0})+\frac{1}{2}{(\mathbf{x}-\mathbf{x_0})}^\top D^2f(\mathbf{x_0})(\mathbf{x}-\mathbf{x_0})+R_3\\$
        $f(\mathbf{x_0})=1^4+2\cdot1^2\cdot1^2+1^4=4\\
        f_{x_1}(\mathbf{x_0})=4\cdot1^3+4\cdot1\cdot1^2=8,f_{x_2}(\mathbf{x_0})=4\cdot1^2\cdot1+4\cdot1^3=8\\
        f_{x_1x_1}(\mathbf{x_0})=12\cdot1^2+4\cdot1^2=16,f_{x_1x_2}(\mathbf{x_0})=8\cdot1\cdot1=8,f_{x_2x_2}(\mathbf{x_0})=4\cdot1^2+12\cdot1^2=16\\
        f(\mathbf{x})=4+[8,8]\begin{bmatrix}
            x_1-1\\
            x_2-1
        \end{bmatrix}+[x_1-1,x_2-1]\begin{bmatrix}
            8 & 4\\
            4 & 8
        \end{bmatrix}\begin{bmatrix}
            x_1-1\\
            x_2-1
        \end{bmatrix}+R_3\\
        =4+8(x_1-1)+8(x_2-1)+[8(x_1-1)+4(x_2-1),4(x_1-1)+8(x_2-1)]\begin{bmatrix}
            x_1-1\\
            x_2-1
        \end{bmatrix}+R_3\\
        =4+8(x_1-1)+8(x_2-1)+8(x_1-1)(x_2-1)+8{(x_1-1)}^2+8{(x_2-1)}^2+R_3\\
        =12+8x_1^2+8x_2^2-16x_1-16x_2+8x_1x_2+R_3$
        \item $f(\mathbf{x_0})=e^{1-0}+e^{1+0}+1+0+1=2e+2\\
        f_{x_1}(\mathbf{x_0})=e^{1-0}+e^{1+0}+1=2e+1,f_{x_2}(\mathbf{x_0})=-e^{1-0}+e^{1+0}+1=1\\
        f_{x_1x_1}(\mathbf{x_0})=e^{1-0}+e^{1+0}=2e,f_{x_1x_2}(\mathbf{x_0})=-e^{1-0}+e^{1+0}=0,f_{x_2x_2}(\mathbf{x_0})=e^{1-0}+e^{1+0}=2e\\
        \Rightarrow f(\mathbf{x})=2e+2+[2e+1,1]\begin{bmatrix}
            x_1-1\\
            x_2
        \end{bmatrix}+\frac{1}{2}[x_1-1,x_2]\begin{bmatrix}
            2e & 0\\
            0 & 2e
        \end{bmatrix}\begin{bmatrix}
            x_1-1\\
            x_2
        \end{bmatrix}+R_3\\
        =2e+2+(2e+1)(x_1-1)+x_2+e{(x_1-1)}^2+ex_2^2+R_3\\
        =ex_1^2+ex_2^2+x_1+x_2+e+1+R_3$
    \end{enumerate}
    \item [\textbf{6.1}] \begin{enumerate}
        \item $\mathbf{d}^\top\nabla f(\mathbf{x}^*)=[0,-1]\begin{bmatrix}
            1\\
            1
        \end{bmatrix}=-1<0$ for $\mathbf{d}={[0,-1]}^\top$ where $\exists\alpha_0$ s.t $\mathbf{x^*}+\alpha\mathbf{d}\in\Omega$  $\forall\alpha\in[0,\alpha_0]$, so by FONC, $\mathbf{x^*}$ is not a local minimum.
        \item $\mathbf{d}=\{{[d_1,d_2]}^\top:x_1,x_2\ge0\}$, so $\mathbf{d}^\top\nabla f(\mathbf{x^*})=d_1\ge0$, so by FONC, $\mathbf{x^*}$ is possibly a local minimum.
        \item $\mathbf{x}^*$ is an interior point of $\Omega$ where $\nabla f(\mathbf{x^*})=\mathbf{0}$ and $\mathbf{F (x^*)}>0$, so by SOSC, $\mathbf{x^*}$ is definitely a local minimum.
        \item $\mathbf{d}=\{{[d_1,d_2]}^\top:x_1,x_2\ge0\}$, so $\mathbf{d}^\top\nabla f(\mathbf{x^*})=[0,1]\begin{bmatrix}
            1\\
            0
        \end{bmatrix}=\mathbf{0}$. However $\mathbf{d}^\top\mathbf{F (x^*)}\mathbf{d}=[0,1]\begin{bmatrix}
            0\\
            -1
        \end{bmatrix}=-1<0$, so by SONC, $\mathbf{x^*}$ is not a local minimum.
    \end{enumerate}
    \item [\textbf{6.4}] If $\mathbf{x^*}$ is an interior point of $\Omega$, then there exists an open ball $B (\mathbf{x^*},\delta_1)$ centered at $\mathbf{x^*}$ s.t $B (\mathbf{x^*},\delta)\subset\Omega$.
    If $\mathbf{x^*}$ is a local minimizer, then there exists an open ball $B (\mathbf{x^*},\delta_2)$ centered at $\mathbf{x^*}$ s.t $f (\mathbf{x^*})\le f (\mathbf{x})$ for all $\mathbf{x}\in B (\mathbf{x^*},\delta_2)\cap\Omega$.
    Pick $\delta=\min\{\delta_1,\delta_2\}$.  
    Because $B (\mathbf{x^*},\delta)\subset\Omega$ and $\Omega\subset\Omega'$, it follows that $B (\mathbf{x^*},\delta)\subset\Omega'$. 
    Thus, there exists a neighborhood of values in $\Omega'$ s.t $f (\mathbf{x^*})\le f (\mathbf{x})$.
    Hence, $\mathbf{x^*}$ is also a local minimizer over $\Omega'$.\\
    For a counterexample when $\mathbf{x^*}$ is a boundary point:\\
    Consider the function $f (\mathbf{x})=\mathbf{x}^\top\begin{bmatrix}
        1 & 0\\
        0 & -1
    \end{bmatrix}\mathbf{x}$ over the region $\Omega=\{\mathbf{x}={[x_1,x_1]}^\top:x_1\ge x_2\ge0\}$. 
    $\mathbf{x^*}={[0,0]}^\top$ is a boundary point and a local minimizer over $\Omega$, but $\mathbf{x^*}$ is not a local minimizer over $\mathbb{R}^2$.
    \item [\textbf{6.7}] Let $\mathbf{y^*}:=\underset{\mathbf{y}\in\Omega'}{\argmin}f(\mathbf{y}-\mathbf{x_0})$.
    It follows $\forall \mathbf{y}\in\Omega'$ $f(\mathbf{y^*}-\mathbf{x_0})\le f(\mathbf{y}-\mathbf{x_0})$.
    If $\mathbf{y^*}\in\Omega'\Rightarrow \mathbf{y^*}-\mathbf{x_0}\in\Omega$.
    Let $\mathbf{x}=\mathbf{y}-\mathbf{x_0}$.
    It follows $f(\mathbf{y^*}-\mathbf{x_0})\le f(\mathbf{x})$ $\forall\mathbf{x}\in\Omega$
    Thus, $\mathbf{y^*}-\mathbf{x_0}=\underset{\mathbf{x}\in\Omega}{\argmin}f(\mathbf{x})\Rightarrow\mathbf{y^*}=\underset{\mathbf{x}\in\Omega}{\argmin}f(\mathbf{x})+\mathbf{x_0}$.
    \item [\textbf{6.10}] \begin{enumerate}
        \item Because $Q=\begin{bmatrix}
            2 & 5\\
            -1 & 1
        \end{bmatrix}$ is not symmetric, we can replace the matrix with $Q_0=\frac{1}{2}(Q+Q^\top)=\frac{1}{2}\begin{bmatrix}
            4 & 4\\
            4 & 2
        \end{bmatrix}$.
        It follows $\nabla f(x)=\begin{bmatrix}
            4 & 4\\
            4 & 2
        \end{bmatrix}\mathbf{x}+\begin{bmatrix}
            3\\
            4
        \end{bmatrix}$ and $\mathbf{F(x)}=\begin{bmatrix}
            4 & 4\\
            4 & 2
        \end{bmatrix}$.\\
        Plugging in $\mathbf{d}=\begin{bmatrix}
            1\\
            0
        \end{bmatrix}$ at $\mathbf{x_0}=\begin{bmatrix}
            0\\
            1
        \end{bmatrix}$ we obtain\\
        $\mathbf{d}^\top\nabla f(\mathbf{x_0})=\begin{bmatrix}
            1\\
            0
        \end{bmatrix}^\top\begin{bmatrix}
            4 & 4\\
            4 & 2
        \end{bmatrix}
        \begin{bmatrix}
            0\\
            1
        \end{bmatrix}
        +\begin{bmatrix}
            1\\
            0
        \end{bmatrix}^\top\begin{bmatrix}
            3\\
            4
        \end{bmatrix}=7$
        \item Want to find all points where $\nabla f(x)=\mathbf{0}$\\
        $\mathbf{x}=-\begin{bmatrix}
            4 & 4\\
            4 & 2
        \end{bmatrix}^{-1}\begin{bmatrix}
            3\\
            4
        \end{bmatrix}=-\begin{bmatrix}
            -\frac{1}{4} & \frac{1}{2}\\
            \frac{1}{2} & -\frac{1}{2}
        \end{bmatrix}
        \begin{bmatrix}
            3\\
            4
        \end{bmatrix}=\begin{bmatrix}
            -\frac{5}{4}\\
            \frac{1}{2}
        \end{bmatrix}$ is the only point that satisfies the FONC.\\
        The Hessian is not positive semidefinite because the determinant is negative. Thus, $\begin{bmatrix}
            -\frac{5}{4}\\
            \frac{1}{2}
        \end{bmatrix}$ does not satisfy the SONC. Hence, $\begin{bmatrix}
            -\frac{5}{4}\\
            \frac{1}{2}
        \end{bmatrix}$ is not a local minimum, so $f$ does not have a minimizer.
    \end{enumerate}
    \item [\textbf{6.11}] \begin{enumerate}
        \item $\nabla f(0)=\begin{bmatrix}
            0\\
            -2\cdot0
        \end{bmatrix}=\begin{bmatrix}
            0\\
            0
        \end{bmatrix}$. Any vector dotted with $\mathbf{0}$ gives $\mathbf{0}$, so $\mathbf{d}^\top\nabla f(\mathbf{0})\ge\mathbf{0}$ for all feasible directions.
        \item $[x_1,x_2]^\top=\mathbf{0}$ is a local maximizer because $\mathbf{F(x)}=\mathbf{x}^\top\begin{bmatrix}
            0 & 0\\
            0 & 1
        \end{bmatrix}\mathbf{x}\le0$ is negative semidefinite. $[x_1,x_2]^\top=\mathbf{0}$ is not strict because if we take $x_1\neq0$ and $x_2=0$ we obtain $f({[x_1,x_2]}^\top)=\mathbf{0}$. 
    \end{enumerate}
    \item [\textbf{6.14}] \begin{enumerate}
        \item $\nabla f(\mathbf{x})=\begin{bmatrix}
            0\\
            1
        \end{bmatrix}$ which is nonzero $\forall\mathbf{x}\in\Omega$. 
        Thus, for $\mathbf{x}$ to satisfy the FONC, $\mathbf{x}\in\partial\Omega=\{\mathbf{x}\in\mathbb{R}^2:x_1^2+x_2^2=1\}$ and $\mathbf{d}^\top\nabla f(\mathbf{x})=d_2\ge0$ for all feasible directions.
        It follows ${[0,1]}^\top$ is the only point that satisfies this condition.
        This can be clearly seen by graphing $x^2+y^2=1$. 
        \item $\mathbf{F(x)}=\mathbf{0}^{2\times2}$, so $\mathbf{F(x)}\ge0$. Thus, any point satisfies the SONC. Hence, ${[0,1]}^\top$ satisfies the SONC.
        \item ${[0,1]}^\top$ is not a local minimizer. 
        Consider the set of points $\Omega'=\{\mathbf{x}\in\mathbb{R}^2:x_1=\sqrt{1-x_2^2},x_2\in[0,1)\}$. 
        Clearly ${(\sqrt{1-x_2^2})}^2+x_2^2=1\ge1\Rightarrow\Omega'\subset\Omega$.
        We can make points in $\Omega$ as close to $[0,1]^\top$ as we want, but $\forall\mathbf{x}\in\Omega'$ $f (\mathbf{x})<f({[0,1]}^\top)$. Hence, $f$ has no minimizer.
    \end{enumerate}
    \item [\textbf{6.17}]\begin{enumerate}
        \item $\nabla f(\mathbf{x})={[\frac{1}{x_1},\frac{1}{x_2}]}^\top$. Because $\frac{1}{x}\neq0$ for $x\in\mathbb{R}$, there are no points where $\nabla f(\mathbf{x})=\mathbf{0}$. Hence, $\mathbf{x^*}$ can't be an interior point.
        \item $\mathbf{F(x)}=\begin{bmatrix}
            -\frac{1}{x_1^2} & 0\\
            0 & -\frac{1}{x_2^2}
        \end{bmatrix}$ which is negative definite for all $x$ because $-\mathbf{F(x)}$ is positive definite. Thus, every $\mathbf{x}$ satisfies the SONC.
    \end{enumerate}
\end{enumerate}
\end{document}