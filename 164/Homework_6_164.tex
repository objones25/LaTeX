\documentclass[10pt]{article}
\usepackage{graphicx}
\usepackage{amssymb}
\usepackage[fleqn]{amsmath}
\usepackage{nccmath}
\usepackage{cases}
\usepackage{hyperref}
\usepackage{multicol}
\usepackage{tikz}
\usepackage{pgfplots}
\usepackage{enumitem}
\usepackage{pdfpages}
\pgfplotsset{compat=1.18}
\usepackage{float}
\DeclareMathOperator*{\argmin}{arg\,min}

\title{\bf Math 164: Problem Set 5}
\date{2/9/2024}
\author{\bf Owen Jones}
\begin{document}
\maketitle
\begin{enumerate}
    \item [\textbf{6.29}]
    \begin{enumerate}
        \item $\displaystyle \frac{1}{n}\sum_{i=1}^{n}{(ax_i+b-y_i)}^2\\
    =\frac{1}{n}\sum_{i=1}^{n}a^2x_i^2+b^2+y_i^2-2ax_i y_i+2ax_i b-2by_i\\
    =a^2\overline{X^2}+b^2+\overline{Y^2}+2a\overline{XY}+2ab\overline{X}-2b\overline{Y}\\
    =\mathbf{z}^\top\begin{bmatrix}
        \overline{X^2} & \overline{X}\\
        \overline{X} & 1
    \end{bmatrix}\mathbf{z}-
    2\begin{bmatrix}
        \overline{XY} & \overline{Y}
    \end{bmatrix} \mathbf{z}+\overline{Y^2}$
        \item By the FONC $\nabla f=2\begin{bmatrix}
            \overline{X^2} & \overline{X}\\
            \overline{X} & 1
        \end{bmatrix}\mathbf{z}-\begin{bmatrix}
            \overline{XY}\\
            \overline{Y}
        \end{bmatrix}=0$\\
        $\Rightarrow \mathbf{z}^*=
        \begin{bmatrix}
            \overline{X^2} & \overline{X}\\
            \overline{X} & 1
        \end{bmatrix}^{-1}\begin{bmatrix}
            \overline{XY}\\
            \overline{Y}
        \end{bmatrix}$ is the only solution.\\
        $=\frac{1}{\overline{X^2}-\overline{X}^2}\begin{bmatrix}
            1 & -\overline{X}\\
            -\overline{X} & \overline{X^2}
        \end{bmatrix}
        \begin{bmatrix}
            \overline{XY}\\
            \overline{Y}
        \end{bmatrix}\\
        =\begin{bmatrix}
            \frac{\overline{XY}-\overline{X}\cdot\overline{Y}}{\overline{X^2}-\overline{X}^2}\\
            \frac{\overline{X^2}\cdot\overline{Y}-\overline{XY}\cdot\overline{X}}{\overline{X^2}-\overline{X}^2}
        \end{bmatrix}$
        \item WTS $\overline{Y}=\overline{X}\frac{\overline{XY}-\overline{X}\cdot\overline{Y}}{\overline{X^2}-\overline{X}^2}+\frac{\overline{X^2}\cdot\overline{Y}-\overline{XY}\cdot\overline{X}}{\overline{X^2}-\overline{X}^2}\\
        =\frac{\overline{X}\cdot\overline{XY}-\overline{X}^2\cdot\overline{Y}}{\overline{X^2}-\overline{X}^2}+\frac{\overline{X^2}\cdot\overline{Y}-\overline{XY}\cdot\overline{X}}{\overline{X^2}-\overline{X}^2}\\
        =\frac{\overline{Y}(\overline{X^2}-\overline{X}^2)}{\overline{X^2}-\overline{X}^2}\\
        =\overline{Y}$
        \end{enumerate} 
        \item [\textbf{6.30}] Let $\displaystyle f(\mathbf{x})=\frac{1}{p}\sum_{i=1}^{p}{\lVert \mathbf{x}-\mathbf{x}^{(p)}\rVert}^2=\frac{1}{p}\sum_{i=1}^{p}{(\mathbf{x}-\mathbf{x}^{(p)})}^\top(\mathbf{x}-\mathbf{x}^{(p)})\\
        \Rightarrow\nabla f=\frac{1}{p}\sum_{i=1}^{p}2(\mathbf{x}-\mathbf{x}^{(i)})=0\\
        \Rightarrow \underset{\mathbf{x}}{\argmin} f(\mathbf{x})=\mathbf{x}^*=\frac{1}{p}\sum_{i=1}^{p}\mathbf{x}^{(i)}$ which is just the mean (or centroid).\\
        Because the hessian $F(\mathbf{x})=\mathbf{I}_n>0$, the mean of $f$ is a strict local minimizer.
        \item [\textbf{6.31}] Because $\Omega$ is convex, let $\phi(\alpha)=f(\mathbf{x}^*+\alpha\mathbf{d})$ for some feasible direction $\mathbf{d}$ and $0<\alpha_0$ s.t $\mathbf{x}^*+\alpha\mathbf{d}\in\Omega,\forall\alpha\in[0,\alpha_0]$.
        By the MVT we have $\frac{\phi(\alpha)-\phi(0)}{\alpha}=\mathbf{d}^\top\nabla f(\xi)$ for $0\le\xi\le \alpha$.
        Because $\lim_{\alpha\rightarrow 0}\frac{\phi(\alpha)-\phi(0)}{\alpha}=\mathbf{d}^\top\nabla f(\mathbf{x}^*)\ge c\lVert \mathbf{d}\rVert>0$, we can find an $\alpha_0$ small enough s.t $\frac{\phi(\alpha)-\phi(0)}{\alpha}>0\Rightarrow\phi(\alpha)>\phi(0)$ for sufficiently small $\alpha$.
        Hence, $f(\mathbf{x}^*)$ is strict local minimizer.
        \item [\textbf{6.32}] Because $\Omega$ is convex, let $\phi(\alpha)=f(\mathbf{x}^*+\alpha\mathbf{d})$ for some feasible direction $\mathbf{d}$ and $0<\alpha_0$ s.t $\mathbf{x}^*+\alpha\mathbf{d}\in\Omega,\forall\alpha\in[0,\alpha_0]$.
        By Taylor's Theorem, $\phi(\alpha)=f(\mathbf{x}^*)+\alpha\mathbf{d}^\top\nabla f(\mathbf{x}^*)+\frac{\alpha^2}{2}\mathbf{d}^\top F(\mathbf{x}^*)\mathbf{d}+o(\alpha^3)\\
        \ge f(\mathbf{x}^*)+\alpha\cdot 0+\frac{\alpha^2}{2}c{\lVert\mathbf{d}\rVert}^2 +o(\alpha^3)>f(\mathbf{x}^*)$ for sufficiently small $\alpha$.
        Hence, $f(\mathbf{x}^*)$ is strict local minimizer.
        \item [\textbf{6.34}] Let $\mathbf{u}={[u_1,u_2,\ldots, u_n]}^\top$. Using the system\\
        $x_n=\alpha x_{n-1}+\beta u_n\\
        =\alpha(\alpha x_{n-1}+\beta u_{n-1})+\beta u_n\\
        \ldots\\
        =\beta \sum_{i=1}^{n}\alpha^{n-i}u_i=\mathbf{v}^\top \mathbf{u}$ where $\mathbf{v}={[\beta\alpha^{n-1},\beta\alpha^{n-1},\ldots \beta ]}^\top$\\
        This gives us the quadratic form minimization problem $f(\mathbf{u})=r\mathbf{u}^\top\mathbf{u}-q\mathbf{v}^\top \mathbf{u}$ with minimizer $\mathbf{u}^*=\frac{q}{2r}\mathbf{v}$ satisfying $\nabla f(\mathbf{u}^*)=2r\mathbf{u}^*-q\mathbf{v}=0$ and $F(\mathbf{u}^*)=2r\mathbf{I}_n>0$.
\end{enumerate}
\end{document}