\documentclass[10pt]{article}
\usepackage{graphicx}
\usepackage{amssymb}
\usepackage[fleqn]{amsmath}
\usepackage{nccmath}
\usepackage{cases}
\usepackage{hyperref}
\usepackage{multicol}
\usepackage{tikz}
\usepackage{pgfplots}
\usepackage{enumitem}
\pgfplotsset{compat=1.18}
\usepackage{float}
\usepackage{pdfpages}
\DeclareMathOperator*{\lcm}{lcm}

\title{\bf Math 116: Problem Set 7}
\author{\bf Owen Jones}
\begin{document}
\maketitle
\begin{enumerate}[label= \arabic*.]
    \item \begin{enumerate}
        \item [(reflexive)] $f(X)\equiv f(X)\pmod{P(X)}\Leftrightarrow P(X)\mid (f(X)-f(X))$. However, $f(X)-f(X)=0$ and any polynomial divides $0$.
        \item [(symmetric)] Suppose $f(X)\equiv g(X)\pmod{P(X)}\Leftrightarrow P(X)\mid (f(X)-g(X))$. It follows $P(X)\mid (g(X)-f(X))$. Thus, $g(X)\equiv f(X)\pmod{P(X)}$
        \item [(transitive)] Suppose $f(X)\equiv g(X)\pmod{P(X)}$ and $g(X)\equiv h(X)\pmod{P(X)}$. It follows $P(X)\mid (f(X)-g(X))$ and $P(X)\mid (g(X)-h(X))$. Thus, $P(X)\mid ((f(X)-g(X))+(g(X)-h(X)))\Rightarrow P(X)\mid(f(X)-h(X))\Leftrightarrow f(X)\equiv h(X)\pmod{P(X)}$.
    \end{enumerate}
    \item Suppose $f_1(X)\equiv f_2(X)\pmod{P(X)}$ and $g_1(X)\equiv g_2(X)\pmod{P(X)}$.\\
    Thus, $P(X)\mid(f_1(X)-f_2(X))$ and $P(X)\mid(g_1(X)-g_2(X))$.\\

    $P(X)\mid((f_1(X)-f_2(X))+(g_1(X)-g_2(X)))\\
    \Rightarrow P(X)\mid((f_1(X)+g_1(X))-(f_2(X)+g_2(X)))\\
    \Leftrightarrow f_1(X)+g_1(X)\equiv f_2(X)+g_2(X)\pmod{P(X)}$.\\
    
    $P(X)\mid (g_1(X)(f_1(X)-f_2(X))+f_2(X)(g_1(X)-g_2(X)))\\
    \Rightarrow P(X)\mid (f_1(X)g_1(X)-f_2(X)g_2(X))\\
    \Leftrightarrow f_1(X)g_1(X)\equiv f_2(X)g_2(X)\pmod{P(X)}$
    \item $8X^4-12X^3+8X-3=(2C-1)(4X^3-4X^2-3X+2)+2X^2+X-1\\
4X^3-4X^2-3X+2=(2X-3)(2X^2+X-1)+2X-1\\
2X^2+X-1=(X+1)(2X-1)+0$\\
$\gcd(8X^4-12X^3+8X-3,4X^3-4X^2-3X+2)=X-\frac{1}{2}$
\item $\begin{array}{c c c}
    & y(x) & z(x)\\
    X^3+2X+2 & 1 & 0\\
    X^2+3X+4 & 0 & 1\\
    2X+4 & 1 & 4X+3\\
    2 & 2X+3 & 2X^2+X+4
\end{array}$\\
$1=(X+4)(X^3+2X+2)+(X^2+3X+2)(X^3+2X+2)$
\item \begin{enumerate}
    \item If $x$ and $p-1$ are coprime, there exists some integer $y$ s.t $xy\equiv 1\pmod{p-1}$. \\
    Because $g_2\equiv g^x\pmod{p}\Rightarrow g_2^y\equiv g^{xy}\equiv g^{k(p-1)}\cdot g\equiv g\pmod{p}$ by Fermat's Little Theorem.\\
    Suppose $m\in\{0,1,\ldots,p-1\}$. Because $g$ is a primitive root, there exists some $q$ s.t $g^q\equiv m\pmod{p}$. 
    Let $q'\equiv qy\pmod{p}$.
    $g_2^{q'}\equiv g_2^{qy}\equiv g^q\equiv m\pmod{m}$. 
    Thus, for any arbitrary $m\in\{0,1,\ldots,p-1\}$, there exists an exponent $q'$ s.t $g_2^{q'}\equiv m\pmod{p}$.
    (Surjectivity+finite domain and codomain of smae size implies a bijection)
    Thus, $g_2$ is a primitive root.
    \item Suppose $x$ is not coprime to $p-1$. It follows there exists some proper divisor $k=\frac{p-1}{\gcd(p-1,x)}\in\mathbb{F_p}$ of $p-1$ s.t $xk\equiv 0\pmod{p-1}$. 
    Since $h^k\equiv g^{xk}\equiv 1\pmod{p}$ then $h$ only cycles through $k<p$ elements of $\mathbb{F_p}$, so $h$ is not a primitive root.
    \item $\phi(p-1)$ because we want the number of integers less than $p-1$ that are coprime to $p-1$
\end{enumerate}
\item \begin{enumerate}
    \item $600=2^3\cdot 3\cdot 5^2$. If $r\mid 600$, then $r$ must share all of it's prime factors with $600$. 
    It follows $r=2^{k_1}\cdot 3^{k_2}\cdot 5^{k_3}$ where $k_1\le 3,k_2\le 1$ and $k_3\le 2$.
    Since $r<600$, at least of of the inequalities must be strict.
    If $k_1<3\Rightarrow r\mid 300,k_2<0\Rightarrow r\mid 200$, and $k_3<2\Rightarrow r\mid 120$.
    \item Since $601$ is prime, $\phi(601)=600$. $k$ is the smallest integer s.t $7^k\equiv 1\pmod{601}$, so $k\mid \phi(601)$ by previous hw. Thus, by part (a), if $k<600\Rightarrow k\mid 120,200$, or $300$.
    \item $7^{300}=600\pmod{601},7^{120}\equiv 423\pmod{601},7^{200}\equiv 576\pmod{601}$. 
    If $k$ divided $120,200$, or $300$ then at least one of our computed exponentiations would be congruent $1\pmod{601}$. 
    \item If $k\mid 600$, but $k\nmid 120,k\nmid 200$, and $k\nmid 300$, then $k\ge 600$. 
    Thus, $k=600$ by definition of being the smallest integer s.t $7^k\equiv 1\pmod{601}$.
    Hence, $7$ must be a primitive root.
    If it weren't, there would be two integers $q_1,q_1$ where $\lvert q_1-q_2\rvert<600$ s.t $7^{q_1}\equiv 7^{q_2}\pmod{601}\Rightarrow 7^{\lvert q_1-q_2\rvert}\equiv 1\pmod{601}$ (because multiplication is well defined) which contradicts that $600$ is the smallest integer s.t $7^k\equiv 1\pmod{601}$.
\end{enumerate}
\item Let $m_i=\frac{p-1}{q_i}$. If $g^{m_i}\not\equiv 1\pmod{p}$ for all $i$, then $g$ is a primitive root. 
\item $65537=2^{16}+1$. It follows we just need to show $3^{2^{15}}\not\equiv 1\pmod{65537}$. Using Python $3^{2^{15}}\equiv 65536\pmod{65537}$, so $3$ is primitive root.
\item \begin{enumerate}
    \item ${(3^k)}^{32}\equiv 3^{32k}\equiv 2^{32}\equiv 1\pmod{65537}\Rightarrow 2^{16}\mid 2^5k\Rightarrow 2^{11}\mid k$ where $2^{11}=2048$ Since ${(3^k)}^{16}\equiv 3^{16k}\equiv 2^{16}\equiv -1\pmod{65537}\Rightarrow 2^{16}\nmid 2^4k\Rightarrow 4096\nmid k$ where $2^{12}=4096$
    \item We only need to check the odd multiples of $2048$, $i=1,3,\ldots 31$. We obtain $3^{55296}\equiv 2\pmod{65537}$ 
\end{enumerate}
\item \begin{enumerate}
    \item $X$ and $X+1$ are clearly irreducible because they are degree $1$. $X^2+X+1$ is irreducible because it has no roots in $\mathbb{F}_2$. 
    There are $2^2=4$ polynomials of degree $2$ with coefficients in $\mathbb{F}_p$, 
    so we need to check $X^2$, $X^2+1$ and $X^2+X$ are all reducible. 
    $X^2+X$ can be reduced into polynomials $X$ and $X+1$.
    $X^2$ can be reduced into polynomials $X$ and $X$.
    $X^2+1$ can be reduced into polynomials $X+1$ and $X+1$.
    \item If $X^4+X+1$ is reducible, then it must be factor into polynomials of degree $2$ and $2$ or $3$ and $1$.
    We do division with remainder on $X^4+X+1$ to check if $X$, $X+1$, or $X^2+X+1$ are factors.\\
    $X^4+X+1=(X^3+1)X+1\\
    X^4+X+1=(X^3+X^2+X)(X+1)+1\\
    X^4+X+1=(X^2+X)(X^2+X+1)+1$\\
    Since $X^4+X+1$ doesn't have any linear or quadratic factors, it must be irreducible.
    \item $X^4\equiv X+1\pmod{X^4+X+1}\Leftrightarrow X^4+X+1\mid (X^4-(X+1))$. $X^4-(X+1)\equiv X^4+X+1\equiv0\pmod{X^4+X+1}$.\\ 
    Since multiplication is well defined $X^8\equiv{(X^4)}^2\equiv{(X+1)}^2\equiv X^2+1\pmod{X^4+X+1}$ and $X^{16}\equiv{(X^8)}^2\equiv {(X^2+1)}^2\equiv X^4+1\equiv (X+1)+1\equiv X\pmod{X^4+X+1}$.\\
    \item Since $X$ and $X^4+X+1$ are coprime $X$ has an inverse $\pmod{X^4+X+1}$. It follows $X^{15}\equiv X^{-1}X^{16}\equiv X^{-1}\cdot X\equiv 1\pmod{X^4+X+1}$
\end{enumerate}
\item \begin{enumerate}
    \item $X^2+1$ doesn't have any roots in $\mathbb{F}_3$, so it must be irreducible. $0^2+1=1,1^2+1=2,2^2+1=2$. 
    \item Extended Euclidean Algorithm\\ 
    $\begin{array}{c c c}
        & y(X) & z(X)\\
        X^2+1 & 1 & 0\\
        2X+1 & 0 & 1\\
          2 & 1 & X+1
    \end{array}\\
    (2X+1)(2X+2)\equiv X^2+2\equiv 1\pmod{X^2+1}$.
    $2+2X$ is the inverse of $1+2X$.
\end{enumerate}
\end{enumerate}
\end{document}