\documentclass[10pt]{article}
\usepackage{graphicx}
\usepackage{amssymb}
\usepackage[fleqn]{amsmath}
\usepackage{nccmath}
\usepackage{cases}
\usepackage{hyperref}
\usepackage{multicol}
\usepackage{tikz}
\usepackage{pgfplots}
\usepackage{enumitem}
\pgfplotsset{compat=1.18}
\usepackage{float}
\usepackage{pdfpages}

\title{\bf Math 116: Practice Midterm}
\date{2/16/2024}
\author{\bf Owen Jones}
\begin{document}
\maketitle
\begin{enumerate}[label=\arabic*.]
    \item $\begin{array}{c c c}
        & x & y\\
        26 & 1 & 0\\
        17 & 0 & 1\\
        9 & 1 & -1\\
        8 & -1 & 2\\
        1 & 2 & -3
    \end{array}\\
    m = -3(E(m)-6)\pmod{26}$\\
    $BANG$\\
    $675$
    \item $n\mid x^3-y^3\Rightarrow n\mid (x-y)(x^2+xy+y^2)$.\\
    Let $d=\gcd(x-y,n)$. If $d=1$ then $\exists$ integers $s,t$ s.t $n(x^2+xy+y^2)s+nt=(x^2+xy+y^2)\Rightarrow n\mid (x^2+xy+y^2)$ which is is known to be false.
    If $d=n$ then $n\mid (x-y)$ which we also know to be false. Thus, $d$ must be a non-trivial factor of $n$.
    \item $a_0=2^{65}\equiv 8192\pmod{n}\\
    a_1={(-129)}^2\equiv -1\pmod{n}\\
    \Rightarrow$ 8321 is probably prime.\\
    \item Let $c\in Im(E_k)$, i.e there exists some $m\in \mathcal{P}$ s.t $c=E_k(m)$.
    By (1), we obtain $D_k(c)=m\Rightarrow c=E_k(m)=E_k(D_k(c))$. 
    It suffices to show $Im(E_k)=\mathcal{C}$. 
    $E_k$ must be 1 to 1 for (1) to hold because $D_k$ can't map an encrypted message to multiiple plaintext messages. 
    Because $\mathcal{C}$ and $\mathcal{P}$ are the same size, $E_k$ must be onto. Thus, $\mathcal{C}=Im(E_k)$.
\end{enumerate}
\end{document}