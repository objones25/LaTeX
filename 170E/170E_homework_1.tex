\documentclass[10pt]{article}
\usepackage{amsmath,amssymb}
\setlength{\oddsidemargin}{0in}
\setlength{\evensidemargin}{0in}
\setlength{\textheight}{9in}
\setlength{\textwidth}{6.5in}
\setlength{\topmargin}{-0.5in}
\usepackage{enumitem}
\usepackage{graphicx}

\title{\bf Math 170E: Homework 1}
\date{7/2/2023}
\author{\bf Owen Jones}

\begin{document}
\maketitle

\begin{enumerate}[label=\textbf{Problem \arabic*.}]
    \item \begin{itemize}
        \item [(1)] $S=\{HHH, HHT, HTH, HTT, THH, THT, TTH, TTT\}$
        \item [(2)] 
        \begin{itemize}
            \item [$\bullet$] $P(A)=P(\{HTT, THT, TTH\})=\frac{3}{8}$
            \item [$\bullet$] $P(B)=P(\{HHH,HTH,THH,TTH\})=\frac{1}{2}$
            \item [$\bullet$] $P(C)=P(\{HHH, HTT, THT, TTH\})=\frac{1}{2}$
        \end{itemize}
    \end{itemize}
    \item \begin{itemize}
        \item [(1)]
        $N(S)=N(\{(1,1),(1,2),(1,3),(1,4),(1,5),(1,6),\\
                 (2,1),(2,2),(2,3),(2,4),(2,5),(2,6),\\
                 (3,1),(3,2),(3,3),(3,4),(3,5),(3,6),\\
                 (4,1),(4,2),(4,3),(4,4),(4,5),(4,6),\\
                 (5,1),(5,2),(5,3),(5,4),(5,5),(5,6),\\
                 (6,1),(6,2),(6,3),(6,4),(6,5),(6,6)\})=36$
        \item [(2)] $\{(1,4),(2,2),(2,4),(2,6),(3,4),(4,1),(4,2),(4,3),(4,4),(4,5),(4,6),(5,4),(6,2),(6,4),(6,6)\}$
        \item [(3)] $P(x:\text{sum equal to 7})=P(\{(1,6),(2,5),(3,4),(4,3),(5,2),(6,1)\})=\frac{1}{6}$
    \end{itemize}
    \item $(A\cap B)\subseteq A$ and $0\le P(A\cap B)$, so $0\le P(A\cap B)\le P(A)$.
    Given $P(A\cup B)=P(A)+P(B)-P(A\cap B)$, it follows $P(B)\le P(A\cup B)\le P(A)+P(B)$. Hence, $P(A\cup B)$ assumes a maximum when $A$ and $B$ are mutually exclusive, and $P(A\cup B)$ assumes a minimum when $A\subset B$.
    \item \begin{itemize}
        \item [(1)] $P(\{6,7,8,9\})=\frac{N(\{6,7,8,9\})}{10^4}=\frac{4!}{10^4}=0.0024$
        \item [(2)] $P(\{7,7,8,8\})=\frac{N(\{7,7,8,8\})}{10^4}=\frac{\frac{4!}{2!2!}}{10^4}=0.0012$
        \item [(3)] $P(\{7,8,8,8\})=\frac{N(\{7,8,8,8\})}{10^4}=\frac{\frac{4!}{1!3!}}{10^4}=0.0004$
    \end{itemize}
    \item \begin{itemize}
        \item [(1)] $2^8-2^5=224$
        \item [(2)] $\frac{3!}{2!\cdot1!}\frac{5!}{5!\cdot0!}+\frac{3!}{2!\cdot1!}\frac{5!}{4!\cdot1!}=18$
    \end{itemize}
    \item \begin{itemize}
        \item [(1)] $\displaystyle\sum_{r=0}^{n}(-1)^r
        \begin{pmatrix}
            n\\
            r
        \end{pmatrix}
        =\sum_{r=0}^{n}(-1)^r(1)^{n-r}
        \begin{pmatrix}
            n\\
            r
        \end{pmatrix}=
        (1+(-1))^n=0^n=0$ for all natural numbers $n$ by the binomial theorem.
        \item [(2)] Consider the parity of $r$. When $r$ is odd, $(-1)^r
        \begin{pmatrix}
            n\\
            r
        \end{pmatrix}$ is negative, and when $r$ is even, $(-1)^r
        \begin{pmatrix}
            n\\
            r
        \end{pmatrix}$ is positive.
        $\begin{pmatrix}
            n\\
            r
        \end{pmatrix}$ denotes the number of ways in which subsets of size $r$ can be created from a set of size $n$. 
        Because $(-1)^r
        \begin{pmatrix}
            n\\
            r
        \end{pmatrix}$ is positive when $r$ is even, the sum of the positive terms will yield the total number of subsets containing an even number of elements. Likewise, $(-1)^r
        \begin{pmatrix}
            n\\
            r
        \end{pmatrix}$ is negative when $r$ is odd, so the sum of the negative terms will yield the total number of subsets containing an odd number of elements. 
        Because $\displaystyle\sum_{r=0}^{n}(-1)^r
        \begin{pmatrix}
            n\\
            r
        \end{pmatrix}
        =0$, it follows the sum of the positive terms equal the sum of the negative terms of the series.
        Hence, the number of subsets containing an even number of elements equal the number of subsets containing an odd number of elements.
    \end{itemize}
\end{enumerate}
\end{document}