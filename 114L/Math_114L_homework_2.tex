\documentclass[10pt]{article}
\usepackage{graphicx}
\usepackage{amssymb}
\usepackage[fleqn]{amsmath}
\usepackage{nccmath}
\usepackage{cases}
\usepackage{hyperref}
\usepackage{multicol}
\usepackage{pgfplots}
\usepackage{enumitem}
\pgfplotsset{compat=1.18}
\usepackage{float}

\title{\bf Math 114L\@: Problem Set 1}
\author{\bf Owen Jones}
\begin{document}
\maketitle
\section*{Question 1}
Suppose for the sake of contradiction that $T$ is not satisfiable. 
It follows by The Completeness Theorem for PL that $T$ is not consistent. 
This implies there exists some $\phi$ s.t $T\vdash \phi$ and $T\vdash \lnot\phi$.  
Because a deduction from a set of formulas $T$ requires a finite sequence of steps, there exists a finite subset $T_0\subseteq T$ that contains all the required formulas to prove $\phi$ and $(\lnot\phi)$.
It follows $T_0\vdash \phi$ and $T_0\vdash \lnot\phi$, and by soundness, $T_0\models\phi$ and $T_0\models(\lnot\phi)$.
$T_0$ is satisfiable by assumption, so there exists $\upsilon$ s.t $\upsilon\models T_0$.
However, $\upsilon(\phi)=$T and $\upsilon((\lnot\phi))=$T cannot both be true.
Thus, $T_0\models\phi$ and $T_0\models(\lnot\phi)$ cannot both be true.
Moreover, $T\vdash \phi$ and $T\vdash (\lnot\phi)$ cannot both be true, so $T$ is consistent.
Because every consistent set of formulas is satisfiable, $T$ must also be satisfiable.
\section*{Question 2}
Suppose $\Gamma\cup\{\phi\}$ logically implies $\psi$. 
Consider some assignment $\upsilon$ s.t $\upsilon\models \Gamma$. 
$\upsilon\models(\phi\rightarrow\psi)$ iff $\upsilon\not\models\phi$ or $\upsilon\models\psi$.
If $\upsilon\models\phi$, then $\upsilon\models\Gamma\cup\{\phi\}$. Because $\upsilon\models\Gamma\cup\{\phi\}$, $\upsilon\models\psi$ by assumption.
If $\upsilon\not\models\phi$ then $(\phi\rightarrow\psi)$ is vacuously true.\\
Suppose $\Gamma$ logically implies $(\phi\rightarrow\psi)$.
Consider some assignment $\upsilon$ s.t $\upsilon\models \Gamma\cup\{\phi\}$.
Then $\upsilon\models\phi$, and because $\upsilon\models\Gamma$, $\upsilon\models(\phi\rightarrow\psi)$.
$\upsilon\models(\phi\rightarrow\psi)$ iff $\upsilon\not\models\phi$ or $\upsilon\models\psi$.
Because $\upsilon\models\phi$, $\upsilon\models\psi$ must be true.
\section*{Question 3} 
A can see the hats of B and C. 
If A saw $2$ white hats, A would deduce they are wearing a black hat because not all the hats are white.
A's answer signals to B that B or C is wearing a black hat.
If B saw C wearing a white hat, B would deduce they are wearing a black hat because otherwise A would have known the color of their (A's) own hat.
Since neither A nor B knew the color of their hats, C can confidently conclude they are wearing a black hat.
\section*{Question 4}
\begin{itemize}
    \item Trivally, if $\Gamma_0\models\phi$ then $\Gamma\models\phi$ for any $\Gamma_0\subset\Gamma$.\\
    Suppose $\Gamma$ logically implies $\phi$. 
    It follows $\Gamma\cup\{(\lnot\phi)\}$ is not satisfiable.
    By compactness, there exists some $\Gamma'\subset\Gamma\cup\{(\lnot\phi)\}$ that is not satisfiable.
    In particular, we want to choose a $\Gamma'$ with the least number of elements. 
    Let $\Gamma_0=\Gamma'\setminus\{(\lnot\phi)\}$. 
    Thus, $\Gamma_0$ logically implies $\phi$ because $\Gamma_0\cup\{(\lnot\phi)\}$ is not satisfiable. 
    Moreover, because of the way we chose $\Gamma'$, for any proper $\Gamma_1\subset\Gamma_0$, $\Gamma_1\cup\{(\lnot\phi)\}$ is satisfiable.
    In other words, $\Gamma_0$ is independent.
    \item Consider the infinite set $\Gamma=\{A_1,\lnot(A_1\rightarrow(\lnot A_2)),\lnot(A_1\rightarrow(A_2\rightarrow(\lnot A_3))),\ldots\}$ for propositional variables $A_1,A_2,A_3,\ldots\in PL_0$.\\
    Case 1: $\Gamma_0$ is empty\\
    $\Gamma_0\not\models A_1$\\
    Case 2: $\Gamma_0=\{\psi_k\}$ contains one formula\\
    Let $\psi_k$ be the k-th formula in the sequence.
    $\Gamma\models\psi_{k+1}$ but $\Gamma_0\not\models \psi_{k+1}$ where $\psi_{k+1}\equiv\lnot(\psi_k\rightarrow(\lnot A_{k+1}))$.\\
    Case 3: WLOG let $m>k$ $\Gamma_0=\{\psi_k,\psi_m,\ldots\}$ contains at least two formulas\\
    $\Gamma_0$ is logically equivalent to $\Gamma_0\setminus{\psi_k}$, so $\Gamma_0$ is not independent.\\
    $\Gamma_0$ cannot be both logically equivalent to $\Gamma$ and independent.
    \item If $\Gamma$ is finite, we showed earlier in the problem we can find a logically equivalent and independent subset $\Gamma_0$. 
    We set $\Delta=\Gamma_0$. 
    If $\Gamma$ is infinite, we showed that some sets have no logically equivalent and independent subsets. 
    If that is the case, let $\Delta=\Gamma$. Otherwise, we choose $\Delta$ to be a logically equivalent and independent subset in a similar manner to a finite set.
\end{itemize}
\section*{Question 5}
\begin{itemize}
    \item [(a)] $\Gamma=\{A_1,(\lnot A_1)\}$
    \item [(b)] $\Gamma=\{A_1,A_2,(A_1\rightarrow (\lnot A_2))\}$ 
    \item [(c)] $\Gamma=\{A_1,A_2,A_3,(\lnot(A_1\rightarrow (\lnot A_2))\rightarrow (\lnot A_3))\}$
\end{itemize}
\section*{Question 6}
\begin{itemize}
    \item [(a)] $\forall i,j\in\{1,2,\ldots, n\},(A_{i,j}\rightarrow A_{j,i})\land(\lnot A_{i,i})$
    \item [(b)] $(\forall i,j\in\{1,2,\ldots, n\},(A_{i,j}\rightarrow A_{j,i})\land(\lnot A_{i,i}))\land(\exists i\in\{1,2,\ldots, n\},\forall j\in\{1,2,\ldots, n\}\rightarrow(\lnot A_{i,j}))$
    \item [(c)] $(\forall i,j\in\{1,2,\ldots, n\},(A_{i,j}\rightarrow A_{j,i})\land(\lnot A_{i,i}))\land(\forall i\in\{1,2,\ldots, n\},\exists j,k\in\{1,2,\ldots, n\}j\neq k \land A_{i,k}\land A_{i,k})$
\end{itemize}
\section*{Question 7}
If there are $n$ propositional variables and each $p_i$ can be assigned T or F, there are a total of $2^n$ ways to assign $\vec{p}=\{p_1,p_2,\ldots,p_n\}$. 
For each inequivalent formula $\chi_i$ $F^{\chi_i}_{\vec{p}}(\vec{x})$ is either T or F, giving us 
$\displaystyle\sum_{i=0}^{2^n}\begin{pmatrix}
    2^n\\
    i
\end{pmatrix}=2^{2^n}$ potential inequivalent formulas.\\
We'll show each one of those inequivalent formulas are achievable by induction.\\
Base case: $n=1$\\
\begin{table}[h]
    \begin{tabular}{|l|l|l|l|}
    \hline
    $p_1$ & $\lnot p_1$ & $p_1\land\lnot p_1$ & $p_1\lor\lnot p_1$ \\ \hline
    T     & F           & F                   & T                  \\ \hline
    F     & T           & F                   & T                  \\ \hline
    \end{tabular}
    \end{table}\\
giving us the four possible inequivalent formulas.\\
Induction hypothesis: Assume for some $n$ each of the $2^{2^n}$ possible inequivalent formulas are achievable.\\
Induction step: \\
We define each of the possible inequivalent functions as follows:\\
$(p_{n+1}\land\chi_i)\lor(\lnot p_{n+1}\land\chi_j)$ $\forall i,j\in\{1,\ldots,N\}$ giving us a total of $N^2={(2^{2^n})}^2=2^{2^{n+1}}$ possible inequivalent formulas.
To show each is logically inequivalent, we consider $(p_{n+1}\land\chi_i)\lor(\lnot p_{n+1}\land\chi_j)$ and $(p_{n+1}\land\chi_k)\lor(\lnot p_{n+1}\land\chi_l)$.
When $p_{n+1}$ is assigned to be true, $\chi_i$ is equivalent to $\chi_k$ iff $i=k$, and when $p_{n+1}$ is assigned to be false, $\chi_j$ is equivalent to $\chi_l$ iff $j=l$ by the induction hypothesis.
Thus, $(p_{n+1}\land\chi_i)\lor(\lnot p_{n+1}\land\chi_j)$ is equivalent to $(p_{n+1}\land\chi_k)\lor(\lnot p_{n+1}\land\chi_l)$ iff $i=k$ and $j=l$.\\
Hence, by induction, there are a total of $2^{2^n}$ inequivalent formulas for $n$ propositional variables.
\end{document}