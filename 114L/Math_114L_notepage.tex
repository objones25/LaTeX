\documentclass[10pt]{article}
\usepackage{graphicx}
\usepackage{amssymb}
\usepackage[fleqn]{amsmath}
\usepackage{nccmath}
\usepackage{cases}
\usepackage{hyperref}
\usepackage{multicol}
\usepackage{pgfplots}
\usepackage{amsthm}
\usepackage{enumitem}
\pgfplotsset{compat=1.18}
\usepackage{float}

\begin{document}
\section*{Formulas of PL}
Formulas: We define them recursively by the $3$ clauses
\begin{itemize}
    \item [a)] Each propositional variable $A_i$ is a formula.
    \item [b)] $(\lnot\phi)$, $(\phi\land\psi)$, $(\phi\lor\psi)$, and $(\phi\rightarrow\psi)$ are also formulas if $\phi$ and $\psi$ are.
    \item [c)] No string that is not built by a) or b) is a formula. 
\end{itemize}
Definition: A set of strings is propositionally closed if it contains all the propositional variables and is closed under sentential conncetives. A string is a formula if it belongs to every propositionally closed set $S$.\\
Induction on formulas:
\begin{itemize}
    \item [1)] Check that each propositional variable has property $P$.
    \item [2)] If $\phi$ has property $P$ then so does $(\lnot\phi)$
    \item [3)] If $\phi$ ans $\psi$ have property $P$ then so does $(\phi\bullet\psi)$ for any binary conncetive $\lor,\land,\rightarrow$.
\end{itemize}
Unique readability:\\
For every formula exactly one of the following is true:
\begin{itemize}
    \item [1)] $\phi$ is a prop. variable
    \item [2)] There is a unique formula $\psi$ s.t $\phi=(\lnot\psi)$.
    \item [3)] There are unique formulas $\psi,\chi$ s.t $\phi=(\psi\lor\chi)$, $(\psi\land\chi)$, or $(\psi\rightarrow\chi)$
\end{itemize}
\section*{Semantics of PL}
We assign a function $\upsilon:\{A_i|i\in\mathbb{N}\}\rightarrow\{T,F\}$ to each propositional variable and extend to all formulas by induction on formulas.\\
Theorem: For any truth assigment $\upsilon:\{A_i|i\in\mathbb{N}\}\rightarrow\{T,F\}$, there is a unique extension $\upsilon:PL\rightarrow\{T,F\}$.\\
Definition: A formula $\varphi\in PL$ is satisfiable if there is an assigment $\upsilon:\{A_i|i\in\mathbb{N}\}\rightarrow\{T,F\}$ s.t $\upsilon(\varphi)=T$\\
Tarski conditions:
\begin{itemize}
    \item $\upsilon\models A_i\leftrightarrow \upsilon(A_i)=T$
    \item $\upsilon\models\lnot\varphi\leftrightarrow\upsilon\not\models\varphi$
    \item $\upsilon\models(\varphi\land\psi)\leftrightarrow\upsilon\models\varphi$ and $\upsilon\models\psi$
    \item $\upsilon\models(\varphi\lor\psi)\leftrightarrow\upsilon\models\varphi$ or $\upsilon\models\psi$
    \item $\upsilon\models(\varphi\rightarrow\psi)\leftrightarrow\upsilon\not\models\varphi$ or $\upsilon\models\psi$
\end{itemize}
Tautologies:
\begin{itemize}
    \item $\models\varphi\rightarrow(\psi\rightarrow\phi)$
    \item $\models(\varphi\rightarrow\psi)\rightarrow((\varphi\rightarrow(\psi\rightarrow\chi))\rightarrow(\varphi\rightarrow\chi))$
    \item $\models(\varphi\rightarrow\psi)\rightarrow((\varphi\rightarrow(\lnot\psi))\rightarrow(\lnot\varphi))$
    \item $\models(\lnot(\lnot\varphi))\rightarrow\varphi$
    \item $\models\varphi\rightarrow(\psi\rightarrow(\varphi\land\psi))$
    \item $\models\varphi\rightarrow(\varphi\lor\psi)$
    \item $\models(\varphi\rightarrow\chi)\rightarrow((\psi\rightarrow\chi)\rightarrow((\varphi\lor\psi)\rightarrow\chi))$
\end{itemize}
\section*{Logical Consequence and Soundness}
$T\models\varphi$ means ``for every assigment that satisfies all formulas in $T$ then $\upsilon$ also satisfies $\varphi$''\\
Modus Ponens: For any two formulas $\varphi,\psi$ $\varphi,(\varphi\rightarrow\psi)\models\psi$.\\
Definition: A deduction or proof from a set of formulas $T$ is a finite sequence of formulas $\chi_0,\chi_1,\ldots,\chi_k$ such that for each $n\le k$ one of the following holds:
\begin{itemize}
    \item [(D1)] $\chi_n\in T$ (assumption)
    \item [(D2)] $\chi_n\equiv \chi_i$ for some $i<n$ (repetition)
    \item [(D3)] $\chi_n$ is an axiom
    \item [(D4)] $\chi_n$ can be inferred with MP for some $\chi_i,\chi_j$ with $i,j<n$
\end{itemize}
$T\vdash \chi$ iff there is a proof $\chi_0,\chi_1,\ldots,\chi_k$ from $T$ s.t $\chi\equiv\chi_k$. Then $\chi$ is a theorem of $T$.\\
Definition: A set of formulas is deductively closed if it contains all the axioms and it is closed under MP.\\
Lemma: For every $T$ and $\varphi$ $T\vdash\varphi$ iff $\varphi\in S$ for every deductively closed $S\supseteq T$.\\
Theorem: (Soundness) For any set of formulas $\Gamma$ and formula $\varphi$ if $\Gamma\vdash\varphi$ then $\Gamma\models\varphi$.
\section*{Consistency and Completeness}
Definition: A set of formulas $\Gamma$ is consistent if there is no formula $\varphi$ s.t $\Gamma\vdash\varphi$ and $\Gamma\vdash\lnot\varphi$.\\
Definition: A set of formulas is strongly complete if $\forall\varphi\in PL$ either $\varphi\in \Gamma$ or $\lnot\varphi\in\Gamma$.\\
Step 1: A set of formulas $S$ is consistent and strongly complete iff there is an assigment $\upsilon$ to the propositional variables s.t for every formula $\varphi$ $\upsilon\models\varphi\leftrightarrow\varphi\in S$.\\
The Deduction Theorem: For any set of formulas $T$ and all $\varphi,\psi$ if $\Gamma,\varphi\vdash \psi$ then $\Gamma\vdash(\varphi\rightarrow\psi)$\\
Step 2: If $\Gamma$ is consistent then for every formula $\varphi$ either $\Gamma\cup\varphi$ or $\Gamma\cup\varphi$ is consistent.\\
Step 3: (The Completeness Theorem for PL) 
\begin{itemize}
    \item [(1)] Every consistent set of formulas is satisfiable
    \item [(2)] If $\Gamma\models\chi$ then $\Gamma\vdash\chi$
\end{itemize}
\section*{Structures}
Language: is a set of constants, function symbols, and relation symbols. $\mathcal{L}=\{c_j\}_{j\in J}\cup\{R_i\}_{i\in I}\cup\{R_k\}_{k\in K}$\\
Terms and formulas:\\
Let $\mathcal{L}$ be a language we consider $S\supseteq \mathcal{L}$ of symbols where we add: 
\begin{itemize}
    \item [(1)] The logical symbols $\lnot,\lor,\land,\rightarrow,\forall,\exists,=$
    \item [(2)] Parentheses $()$
    \item [(3)] The individual variables $v_0,v_1,\ldots$
\end{itemize}
Definition: An $\mathcal{L}$-term is defined by recursion where:
\begin{itemize}
    \item [a)] Each variable $v_i$ is a term
    \item [b)] Each constant symbol is a term
    \item [c)] If $\tau_1,\ldots,\tau_n$ are terms and $f$ is an $n-ary$ function symbol then $f(\tau_1,\ldots,\tau_n)$ is a term
\end{itemize}
Definition: An $\mathcal{L}$-formula is defined by recursion where:
\begin{itemize}
    \item [a)] If $s,t$ are terms then $s=t$ is a formula
    \item [b)] If $\tau_1,\ldots,\tau_n$ are terms and $R$ is an $n-ary$ relation symbol then $R(\tau_1,\ldots,\tau_n)$ is a formula
    \item [c)] If $\varphi,\psi$ are formulas and $v$ is a variable $(\lnot\varphi),(\varphi\bullet\psi),\exists v\varphi,\forall v\varphi$ are formulas.
\end{itemize}
Let $\varphi$ be an $\mathcal{L}$-formula then
\begin{itemize}
    \item [(1)] $\varphi$ is quantifier free if $\exists$ and $\forall$ do not occur in $\varphi$
    \item [(2)] The variable $x_i$ if free in $\varphi$ if it is not quantified
\end{itemize}
\section*{Semantics}
A structure in the language $\mathcal{L}=\{c_j\}_{j\in J}\cup\{R_i\}_{i\in I}\cup\{R_k\}_{k\in K}$ is a pair $A=(A,I)$ where:
\begin{itemize}
    \item $A$ is a non-empty set (we call $A$) to be the universe 
    \item $I$ is an interpretation that assigns:
    \begin{itemize}
        \item $I(c)=c^A$ an element of $A$.
        \item $I(R)=R^A$ is an $n-ary$ relation $R^A\subset A^n$ to each $n-ary$ relation $R$.
        \item $I(f)=f^A:A^n\rightarrow A$ an $n-ary$ function for each function symbol.
    \end{itemize}
\end{itemize}
Definition: An isomorphism $\sigma:A\rightarrow B$ where both are $\mathcal{L}$-structures is a bijection such that:
\begin{itemize}
    \item [(1)] For each constant symbol $c_A$ $\sigma(c_A)=c_B$
    \item [(2)] For each $n-ary$ relation symbol $R$ and $a_1,a_2,\ldots,a_n\in A$, $(a_1,a_2,\ldots,a_n)\in R^A\leftrightarrow (\sigma(a_1),\sigma(a_2),\ldots,\sigma(a_n))\in R^B$
    \item [(3)] For eahc $n-ary$ function symbol $f$ and $a_1,a_2,\ldots,a_n\in A$, $=\sigma(f^A(a_1,a_2,\ldots,a_n))=f^B(\sigma(a_1),\sigma(a_2),\ldots,\sigma(a_n))$ 
\end{itemize}
If $A$ and $B$ are isomorphic then we denote this by $A\cong B$\\
Theorem: Let $M$ and $N$ be $\mathcal{L}$-structures and $f:M\rightarrow N$ an isomorphism. 
If $\upsilon$ is an assigment $\upsilon:Var\rightarrow M$ then $f\circ\upsilon:Var\rightarrow N$ is an assignment on $N$ and for any formula $\varphi$ and tuple $\overline{m}\in M$ $(M,\upsilon)\models\varphi(\overline{m})$ iff $(M,f\circ\upsilon)\models\varphi(f(\overline{m}))$.\\
Definition: An automorphism $\sigma$ of $M$ is an isomorphism of $M$ onto $M$.
Definition: (Substructure) Let $A=(A,I)$ $B=(B,J)$ be $\mathcal{L}$-structures. $A$ is a substructure of $B$ if 
\begin{itemize}
    \item $A\subseteq B$
    \item For every constant symbol $c^B=c^A\in A$
    \item For any $n-ary$ function and $a_1,a_2,\ldots,a_n\in A$, $f^A(a_1,a_2,\ldots,a_n)=f^B(a_1,a_2,\ldots,a_n)$
    \item For any $n-ary$ relation and $a_1,a_2,\ldots,a_n\in A$ $R^B(a_1,a_2,\ldots,a_n)\leftrightarrow R^A(a_1,a_2,\ldots,a_n)$
\end{itemize}
Definition: An assigment into a structure $A$ is any association of objects in $A$ with variables $\pi:Variables\rightarrow A$.\\
The value of an assigment is proven by induction on formulas.\\
Satisfaction: Let $A$ be and $\mathcal{L}$-structure and $\pi$ an assigment in $A$ and $\varphi$ an $\mathcal{L}$-formula\\
We say $A,\pi\models\phi\leftrightarrow value(\varphi,\pi)=1\leftrightarrow$ the assigment $\pi$ satisfies the formula in the structure $A$.\\
Tarski conditions:
\begin{itemize}
    \item $A,\pi\models s=t\leftrightarrow {value}^A(s,\pi)={value}^A(t,\pi)$
    \item $A,\pi\models \lnot\varphi\leftrightarrow A,\pi\not\models \varphi$
    \item $A,\pi\models\varphi\land\psi\leftrightarrow$ $A,\pi\models\varphi$ and $A,\pi\models\psi$
    \item $A,\pi\models\varphi\lor\psi\leftrightarrow$ $A,\pi\models\varphi$ or $A,\pi\models\psi$
    \item $A,\pi\models\varphi\rightarrow\psi\leftrightarrow$$A,\pi\not\models\varphi$ or $A,\pi\models\psi$
    \item $A,\pi\models\exists v\varphi\leftrightarrow$ there is $a\in A$ such that $(A,\pi(v:=a))\models\varphi$
    \item $A,\pi\models\forall v\varphi\leftrightarrow$ for all $a\in A$ $(A,\pi(v:=a))\models\varphi$
\end{itemize}
Definition: A proposition $\Phi$ in ordinary English about an $\mathcal{L}$-structure $A$ is expressed or formalized by a sentence if $\Phi$ and $\varphi$ mean the same thing ($\phi$ is a statement)$\rightarrow$ $\varphi$ is a sentence if all variables are quantified.
\section*{Definable Sets}
Let $R\subseteq A^n$ where $A$ is an $\mathcal{L}$-structure and $R$ is a relation in the universe of $A^n$. We say that $R$ is definable if there is (1) an $\mathcal{L}$-formula $\varphi(\overline{x},\overline{y})$ (2) a tuple $\overline{a}_0\in A^{\lvert\overline{y}\rvert}$ such that $A\models \varphi(\overline{b},\overline{a}_0)$ iff $\overline{b}\in A$ and we say that it is definable over $a_0$\\
If $R$ is definable over $a_0$ then there is an $\mathcal{L}$-formula $\varphi(\overline{x},\overline{y})$ such that:\\ 
$(m_1,m_2,\ldots,m_n)\in R$ iff $(M,v)\models\varphi(\overline{m},a_0)\leftrightarrow(M,\sigma\circ v)\models\varphi(\sigma(\overline{m}),a_0)$
\section*{Theories}
Definition: Let $\mathcal{L}$ be a fixed language on $\mathcal{L}$-theory $T$ is any (possibly infinite) set of sentences $T$. The members of $T$ are called axioms.\\
Definition 2: An $\mathcal{L}$-structure $M$ is a model of an $\mathcal{L}$-theory $T$ if for any sentence $\varphi\in T$ $M\models \phi$ i.e $M\models T$.\\
Definition 3: The models of a theory $T$ $T$ is the class of $\mathcal{L}$-structures $Mod(T)=\{M|M\text{ is an }\mathcal{L}\text{-structure and }M\models T\}$\\
Definition: Let $\mathcal{L}$ be a fixed language and let $\Phi$ be a property of $\mathcal{L}$-structures we say that it is elementary if there is an $\mathcal{L}$-theory $T$ such that:\\
For $M$ and $\mathcal{L}$-structure $M$ has property $\Phi\leftrightarrow M\models T$ 
Definition: Let $M$ and $N$ be $\mathcal{L}$-structures then they are elementary equivalent if $\forall\varphi$ $\mathcal{L}$-sentence $M\models\varphi$ iff $N\models\varphi$ we denote this by $M\equiv N$,
\section*{Logical Consequence and Proofs}
Definition: Let T be an $\mathcal{L}$-Theory and $\varphi$ an $\mathcal{L}$-sentence we say that $\varphi$ is a logical consequence $T\models\varphi$ iff every $\mathcal{L}$-structure $M\models T$ then $M\models \varphi$.\\
Hilbert's axiom schemes\\
(a) Propositional Axiom Schemes: The set of logical tautologies\\
(b) Predicate Axiom Schemes\begin{itemize}
    \item $\forall\upsilon\varphi(\upsilon,\vec{u})\rightarrow \varphi(\tau,\vec{u})$ $\tau$ free for $\upsilon$ in $\varphi(\upsilon,\vec{u})$
    \item $\forall\upsilon(\varphi\rightarrow\psi)\rightarrow(\varphi\rightarrow\forall\upsilon\psi)$ $\upsilon$ not free in $\varphi$
    \item $\varphi(\tau,\vec{u})\rightarrow\exists\upsilon\varphi(\upsilon,\vec{u})$ $\tau$ free for $\upsilon$ in $\varphi(\upsilon,\vec{u})$
\end{itemize}
(c) Rules of Inference \begin{itemize}
    \item From $\varphi$ and $\varphi\rightarrow\psi$ infer $\psi$ MP
    \item From $\varphi$ infer $\forall\upsilon\varphi$ (Generalization)
    \item From $\varphi\rightarrow\psi$ infer $\exists\upsilon\varphi\rightarrow\psi$ provided $\upsilon$ is not free in $\psi$ (Exists Elimination)
\end{itemize}
(d) Identity Axioms \begin{itemize}
    \item $\upsilon=\upsilon$ $\upsilon=\upsilon'\rightarrow\upsilon'=\upsilon$ $\upsilon=\upsilon'\rightarrow((\upsilon'\rightarrow\upsilon'')\rightarrow(\upsilon\rightarrow\upsilon''))$
    \item $(v_1=w_1\land\ldots\land v_n=w_n)\rightarrow(R(v_1,\ldots,v_n)\rightarrow R(w_1,\ldots,w_n))$ any $n-ary$ relation symbol
    \item $(v_1=w_1\land\ldots\land v_n=w_n)\rightarrow(f(v_1,\ldots,v_n)=f(w_1,\ldots,w_n))$ any $n-ary$ function symbol
\end{itemize}
Definition: A deduction from a theory $T$ is any sequence of formulas $\varphi_0,\ldots\varphi_n$ where each $\varphi_i$ is either:
\begin{itemize}
    \item an axiom $\varphi_i\in$ Hilbert's Axioms
    \item a hypothesis $\varphi_i\in T$
    \item repetition $\varphi_i=\varphi_j$ for $j<i$
    \item or follows from MP, Generalization, or Exists Elimination
\end{itemize}
Definition: (Soundness for first order logic) Let $T$ be a theory and $\chi$ a sentence if $T\vdash\chi$ then $T\models\chi$\\
Lemma: Let $M$ be an $\mathcal{L}$-structure and $\chi$ an axiom in the Hilbert's list (not inference) then $M\models\chi$
\section*{Completeness Theorem}
Definition 1: $\Gamma$ a theory and $\varphi$ a sentence if $\Gamma\models\varphi$ then $\Gamma\vdash\varphi$\\
Definition 2: If $\Gamma$ is consistent then it is satisfiable\\
Consistency: A theory $T$ is consistent if there is no sentence $\varphi$ such that $T\vdash\varphi$ and $T\vdash\lnot\varphi$\\
An $\mathcal{L}$-theory is satisfiable if there is an $\mathcal{L}$-structure $M$ such that $M\models T$\\
An $\mathcal{L}$-theory is complete if for every sentence $\varphi$ either $\varphi\in\Gamma$ or $\lnot\varphi\in\Gamma$\\
Definition: A theory $T$ has the Henkin property if:
\begin{itemize}
    \item [(a)] it is consistent
    \item [(b)] it is complete
    \item [(c)] if $\exists v \varphi\in H$ then there is a constant $c$ such that $\varphi(c)\in H$
\end{itemize}
Idea: add a bunch of constants to witness the existentials\\
Definition: Fix a Henkin set we say that it is deductively closed if for every sentence $\chi$ if $H\vdash\chi$ then $\chi\in H$
For all sentences $\varphi,\psi,\exists v\varphi(v),\forall v\varphi(v)$
\begin{itemize}
    \item [(a)] $\lnot\varphi\leftrightarrow\varphi\not\in H$
    \item [(b)] $\varphi\land\psi\in H\leftrightarrow\varphi\in H$ and $\psi\in H$
    \item [(c)] $\varphi\lor\psi\in H\leftrightarrow\varphi\in H$ or $\psi\in H$
    \item [(d)] $\varphi\rightarrow\psi\in H\leftrightarrow\varphi\not\in H$ or $\psi\in H$
    \item [(e)] $\exists v\varphi(v)\in H\leftrightarrow$ there is some constant $c$ such that $\varphi(c)\in H$
    \item [(f)] $\forall v\varphi(v)\in H\leftrightarrow$ for all constants $c$ $\varphi(c)\in H$
\end{itemize} 
Every Theory can be extended to a Henkin Set\\
Every Henkin set is satisfiable\\
\section*{Compactness Theorem}
(PL Logic) Suppose $T$ is an infinite set of formulas. Prove that if every finite subset $T_0\subset T$ is satisfiable, then $T$ is satisfiable.\\
(First Order) If every finite subset of a theory $T$ in a finite vocabulary has a model, then $T$ has a model.\\
(Compactness-Completeness) Let $\mathcal{L}$ be a language and $T$ be a first order theory. The following are equivalent:\\
\begin{itemize}
    \item [1.] $T$ is consistent
    \item [2.] $T$ is finitely satisfiable
    \item [3.] $T$ is satisfiable
    \item [4.] $T$ is finitely satisfiable
\end{itemize}
\section*{Applications}
\begin{itemize}
    \item [1.] Any set $A$ can be lineraly ordered
    \item [2.] Any torsion free group can be lineraly ordered
    \item [3.] There is a non-archimedian field $F$ such that $F\equiv(\mathbb{R},+,\cdot,0,1)$
    \item [4.] Ramsey's theorem: For every natural number $k$ there is a natural number $n$ such that for every coloring $c:{[n]}^2\rightarrow\{R,B\}$ there is $S\subset\{1,\ldots,n\}$ of size $k$ that is monocromatic.
    \item [5.] Let $\mathcal{L}$ be the language of fields. For every $\phi$, a $\mathcal{L}$ sentence, there is some $N\subset\mathbb{N}$ such that for every prime number $p\ge N$ such that $(\mathbb{C},+,\cdot,0,1)\models\phi$ iff $(F^{alg}_p,+,\cdot,0,1)\models\phi$.  
\end{itemize}
\section*{Elementary Substructure}
Let $M$ be a structure, and let $A$ be a subset of $M$. 
Then $A$ is an elementary substructure of $M$ if for every formula $\phi(v_1, \dots, v_n)$ and every tuple $\bar{a} \in A^n$, if $M \models \phi(\bar{a})$, then $A \models \phi(\bar{a})$.
\section*{Tarski Vaught criterion}
Let $M$ be a structure, and let $A,B \subseteq M$. 
Then $A$ is an elementary substructure of $M$ with respect to $B$, denoted by $A \prec_B M$, if for every formula $\phi(v_1, \dots, v_n)$ and every tuple $\bar{a} \in A^n$, if $M \models \phi(\bar{a}, \bar{b})$ for some $\bar{b} \in B^n$, then $A \models \phi(\bar{a})$.
\section*{Tarski Gödel}
\subsection{Arithmetization}

The document begins by explaining the concept of arithmetization, which is the process of encoding mathematical statements and proofs as numerical sequences. This is a crucial technique used in the proofs of both Tarski's and Gödel's theorems. It introduces the coding of sequences and terms, including the coding of the language of Peano Arithmetic (PA) and how terms and formulas are represented arithmetically.

\subsection{The Theorems}

\begin{itemize}
    \item \textbf{Tarski's Undefinability Theorem}: This theorem asserts that the set of all true arithmetic statements (Truth(N)) is not arithmetically definable. In other words, there is no single arithmetic formula that can capture all and only the truths about natural numbers.
    \item \textbf{Gödel's First Incompleteness Theorem}: This theorem states that any consistent formal system that is capable of expressing basic arithmetic is incomplete. There are true statements within the system that cannot be proven using the system's axioms and rules.
    \item \textbf{Gödel's Second Incompleteness Theorem}: This theorem extends the first by showing that such a system cannot prove its own consistency.
\end{itemize}

\subsection{Problems}

The document concludes with a set of problems designed to reinforce the concepts discussed. These problems involve proving various properties of the coding functions and the relations used in the proofs of the theorems.

\section{Key Concepts and Definitions}

\begin{itemize}
    \item \textbf{Arithmetization}: The technique of representing statements, sequences, and proofs in arithmetic form. This involves encoding symbols, terms, and formulas as natural numbers.
    \item \textbf{Sequence Coding}: A method for encoding finite sequences of natural numbers. The document details specific functions and properties related to this coding.
    \item \textbf{Formulas and Proofs}: The document explains how formulas and proofs are encoded, and how these encodings are used to state and prove the theorems of Tarski and Gödel.
\end{itemize}

\subsection{Theorem Statements}

\begin{itemize}
    \item \textbf{Tarski's Theorem}: The set of arithmetical truths (Truth(N)) is not arithmetical.
    \item \textbf{Gödel's First Incompleteness Theorem}: No arithmetical theory can be both sound (all its theorems are true) and complete (it proves all true statements).
    \item \textbf{Gödel's Second Incompleteness Theorem}: No consistent arithmetical theory can prove its own consistency.
\end{itemize}

\section{Proof Techniques}

\begin{itemize}
    \item \textbf{Self-Reference and Diagonalization}: Techniques used in the proofs involve creating self-referential statements and using diagonalization to show that certain sets cannot be captured by arithmetic formulas.
    \item \textbf{Arithmetical Functions and Relations}: Various arithmetical functions and relations are defined and used to construct the necessary statements for the proofs.
\end{itemize}

\section*{Midterm Solutions}
\subsection*{Exercise 1 (Propositional logic)}
Let $L = \{A_i : i \in \mathbb{N}\}$ be a propositional language. Let $\varphi$ and $\theta$ be $L$-sentences and let $\Sigma$ be an $L$-theory.

\begin{enumerate}[label=(\arabic*)]
    \item State the definition of a $\Sigma$-proof of $\varphi$.
    \item Suppose that $\Sigma \vdash (\varphi \rightarrow \theta)$. Prove that $\Sigma \vdash ((\neg \theta) \rightarrow (\neg \varphi))$.
\end{enumerate}

\textbf{Solution:}
\begin{enumerate}[label=(\arabic*)]
    \item See course notes.
    \item The easiest way to do this problem is to use the Completeness Theorem for propositional logic. This says that if $\Sigma$ is an $L$-theory, and $\varphi$ is a propositional formula, then $\Sigma \vdash \varphi$ iff $\Sigma \models \varphi$. Suppose that $\Sigma \vdash (\varphi \rightarrow \theta)$. Then $\Sigma \models (\varphi \rightarrow \theta)$. We want to show that $\Sigma \models ((\neg \theta) \rightarrow (\neg \varphi))$, which will show $\Sigma \vdash ((\neg \theta) \rightarrow (\neg \varphi))$. Suppose that $v$ is a truth assignment that satisfies $\Sigma$. By assumption, $\Sigma \models (\varphi \rightarrow \theta)$, so $v$ satisfies $(\varphi \rightarrow \theta)$. But $(\varphi \rightarrow \theta)$ is logically equivalent to $((\neg \theta) \rightarrow (\neg \varphi))$, so we conclude that $v$ also satisfies $((\neg \theta) \rightarrow (\neg \varphi))$. This completes the proof.
\end{enumerate}

\subsection*{Exercise 2 (Definable sets)}
\begin{enumerate}[label=(\arabic*)]
    \item Let $M = (A; |, 1)$, where $A = \mathbb{N} \setminus \{0\} = \{1, 2, 3, \ldots \}$, $|$ is the binary relation on $A$ defined by $n|m$ if and only if $n$ divides $m$, and $1$ is a distinguished point. Show that $P = \{p \in \mathbb{N} : p \text{ is a prime number}\}$ is a definable set.
    \item Consider the structure $M_1 = (\mathbb{Z}; +)$ with the standard interpretation. Prove that $\mathbb{N}$ is not a definable subset of $\mathbb{Z}$ in the language $L = \{+\}$.
\end{enumerate}

\textbf{Solution:}
\begin{enumerate}[label=(\arabic*)]
    \item Since a positive integer $x$ is prime iff $x \neq 1$ and every positive integer divisor of $x$ is either $1$ or $x$, the set $P$ of primes is defined by the formula
    \[
    \varphi(x) = ``x \neq 1 \land \forall y(y|x \rightarrow (y = 1 \lor y = x))".
    \]
    (Note that the number $1$ is not prime, by definition.) Alternatively, since a positive integer $x$ is prime iff $x$ has exactly two positive integer divisors, the set $P$ of primes is also defined by the formula
    \[
    \psi(x) = ``\exists y \exists z(y \neq z \land y|x \land z|x \land \forall w(w|x \rightarrow (w = y \lor w = z)))".
    \]
    \item The only rigorous way we know of to show that a subset is not definable is to use the following Fact: if $D \subseteq A$ is definable and $G : A \rightarrow A$ is an automorphism, then whenever $a \in D$ we have $G(a) \in D$. Thus, to show that a set is not definable, we need to use a nontrivial automorphism of the structure. There is only one nontrivial automorphism of $(\mathbb{Z}; +)$, namely the map $G : \mathbb{Z} \rightarrow \mathbb{Z}$ given by $G(n) = -n$ for all $n \in \mathbb{Z}$. Now just observe that $1 \in \mathbb{N}$ but $G(1) = -1 \notin \mathbb{N}$. Hence, the Fact implies that $\mathbb{N}$ is not a definable subset of $(\mathbb{Z}; +)$.
\end{enumerate}

\subsection*{Exercise 3 (Sentences and isomorphism)}
Let $L = \{\leq\}$. Consider

\begin{enumerate}[label=(\alph*)]
    \item $M_1 = (\mathbb{N}; \leq)$, standard interpretation.
    \item $M_2 = (\mathbb{Q}; \leq)$, standard interpretation.
\end{enumerate}

\begin{enumerate}[label=(\arabic*)]
    \item Find an $L$-sentence $\varphi$ such that $M_1 \models \varphi$ and $M_2 \not\models \varphi$.
    \item Conclude that $M_1$ and $M_2$ are not isomorphic, i.e. $M_1 \not\cong M_2$.
    \item Provide an example of an automorphism of $M_2$.
\end{enumerate}

\textbf{Solution:}
\begin{enumerate}[label=(\arabic*)]
    \item There are many possibilities. One example of such a sentence $\varphi$ is
    \[
    \exists x \forall y (x \leq y).
    \]
    Another possibility is
    \[
    \exists x \exists y (x \neq y \land x \leq y \land \neg \exists z (z \neq x \land z \neq y \land x \leq z \land z \leq y)),
    \]
    which says that there exist consecutive elements $x$ and $y$ with no element in between them.
    \item We proved in class that isomorphic models are always elementarily equivalent. Since $M_1$ and $M_2$ are not elementarily equivalent by part (1), we conclude that $M_1 \not\cong M_2$.
    \item There are many examples of automorphisms of $M_2$ besides the identity map. For example, $G(q) = q + 1$ and $H(q) = q / 2$ are automorphisms.
\end{enumerate}

\subsection*{Exercise 4 (True or False)}
\begin{enumerate}[label=(\arabic*)]
    \item Let $L$ be a finite language containing only relational symbols and constants. Let $M$ and $N$ be $L$-structures, then $M \equiv N$ if and only if $M \cong N$.
    \item Consider the structures $M_1 = (\mathbb{Z}; +, 0)$ and $M_2 = (2\mathbb{Z}; +, 0)$. Then $M_1 \equiv M_2$.
    \item Let $L$ be a fixed language, and let $T$ be a first-order theory. If there is an $L$-structure $M \models T$, then $T$ is consistent.
    \item $(\mathbb{N}, +)$ is a substructure of $(\mathbb{Z}, +)$.
\end{enumerate}

\textbf{Solution:}
\begin{enumerate}[label=(\arabic*)]
    \item False. We only know this is true for finite structures $M$ and $N$. We will see soon that $(\mathbb{R}; \leq) \equiv (\mathbb{Q}; \leq)$ but these structures are not isomorphic.
    \item True. The structures $M_1$ and $M_2$ are even isomorphic via $G : \mathbb{Z} \rightarrow 2\mathbb{Z}$ given by $G(n) = 2n$. Hence they are elementarily equivalent. (Given the tools we have at our disposal at this point in the course, the only way we know how to show that two structures are elementarily equivalent is to show they are isomorphic.)
    \item True. This follows from the Soundness Theorem for first-order logic.
    \item True. One just has to check that $\mathbb{N}$ is closed under $+$.
\end{enumerate}
\end{document}