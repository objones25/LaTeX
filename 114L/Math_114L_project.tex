\documentclass[10pt]{article}
\usepackage{graphicx}
\usepackage{amssymb}
\usepackage[fleqn]{amsmath}
\usepackage{nccmath}
\usepackage{cases}
\usepackage{hyperref}
\usepackage{multicol}
\usepackage{pgfplots}
\usepackage{enumitem}
\pgfplotsset{compat=1.18}
\usepackage{float}

\title{\bf Math 114L\@: Project}
\author{\bf Owen Jones}
\begin{document}
\maketitle
\section*{Problem 1}
To prove that there exists an $\mathcal{L}$ structure $M$ $M\models\bigcup_{i\in\mathbb{N}}T_i$, we will use the compactness theorem.
The compactness theorem for first order logic states that if every finite subset of an $\mathcal{L}$-theory is consistent, then the $\mathcal{L}$-theory is satisfiable.
By assumption, for any finite subset $A\in\mathbb{N}$, the theory $T_A$ is consistent. 
Because every finite subset $T_A$ of $\bigcup_{i\in\mathbb{N}}T_i$ is consistent, there must exist an $\mathcal{L}$-structure $M$ $M\models\bigcup_{i\in\mathbb{N}}T_i$.
\section*{Problem 2}
\subsection*{Part A}
\begin{itemize}
    \item [(a)] We need to show $\sigma$ is a bijection that preserves the structure of $\mathcal{L}_0$. 
    Let $a,a'\in\mathbb{Z}$. 
    Clearly, $\sigma(a)=a+1=a'+1=\sigma(a')$ iff $a=a'$. 
    For every $b\in \mathbb{Z}$, there exist $b-1\in\mathbb{Z}$ s.t $\sigma(b-1)=b$.
    Thus, $\sigma$ is a bijection because the function is injective and surjective.
    $\sigma(s(x))=\sigma(x+1)=x+2=s(x+1)=s(\sigma(x))$, so $\sigma$ preserves the structure of $\mathcal{L}_0$. 
    Thus, $\sigma$ is an automorphism.
    \item [(b)] Suppose there is a relation $R$ s.t some element $a\in\mathbb{Z}$ were definable. 
    Let $\sigma$ be the automorphism from part $a$. 
    $R$ is definable if $a\in R\leftrightarrow \sigma(a)\in R$. 
    Since $\sigma(a)\neq a$, $a\in\mathbb{Z}$ is not definable.
    \item [(c)] Suppose there as a relation $R$ s.t $2\mathbb{Z}$ is definable. 
    Let $\sigma$ be the automorphism from part $a$. 
    $R$ is definable if $a\in R\leftrightarrow\sigma(a)\in R$.
    If $a\in2\mathbb{Z}$, then $\sigma(a)=a+1$ is odd.
    Thus, the set of even integers can't be definable.
\end{itemize}
\subsection*{Part B}
\begin{itemize}
    \item [(a)] Let $n$ be an integer. 
    $\varphi(x):(x=\underbrace{s(s(s(\ldots s(0))))}_{n\text{ many times}})$ is a formula that defines any non-negative integer $n$.
    $\varphi(x):(0=\underbrace{s(s(s(\ldots s(x))))}_{-n\text{ many times}})$ is a formula that defines any negative integer $n$.
    \item [(b)] Let $\sigma$ be an automorphism of $\mathcal{M}_1$.
    We show by induction that $\sigma$ is the identity.
    $\sigma(0)=0$ because automorphisms preserve constants.
    Let $k\in \mathbb{Z}$. 
    Assume $\sigma(k)=k$. 
    $\sigma(s(k))=s(\sigma(k))$ because automorphisms preserve functions.
    Thus, $\sigma(k+1)=\sigma(k)+1=k+1$.
    Likewise, $k=\sigma(k)=\sigma(s(k-1))=s(\sigma(k-1))=\sigma(k-1)+1$, so $\sigma(k-1)=k-1$
    Hence, for any $k\in\mathbb{Z}$ $\sigma(k)=k$. Therefore, $\sigma$ must be the identity.
    \item[(c)] Define a new theory $T^*:=Th(\mathcal{M}_1)\cup\{\lnot(a=n):n\in\mathbb{Z}\}$. We showed previously every integer is definable.
    We show $T^*$ is satisfiable by considering any finite subset of $T^*$. 
    $\mathcal{M}_1\models Th(\mathcal{M}_1)$ and for any finite subset of $\{\lnot(a=n):n\in\mathbb{Z}\}$ only finitely many witnesses $n\in\mathbb{Z}$. 
    We can let $a$ be the successor to the largest $n$ in the subset. 
    Since, $T^*$ is finitely satisfiable, there exists a model $\mathcal{N}$ $\mathcal{N}\models T^*$ that is elementary equivalent to $\mathcal{M}_1$, contains $\mathcal{M}_1$ as an elementary substructure, but has an element $a\not\in\mathbb{Z}$.
    \item [(d)] To show $s_N$ is injective we consider $x,y\in N$. If $s_N(x)=s_N(y)\Rightarrow x+1=y+1\Rightarrow x=y$.
    To show $s_N$ is surjective, we consider two cases:\\
    Case 1: $y\in\mathbb{Z}$. This case is trivial. We know there exists $y-1\in\mathbb{Z}$ $s_N(y-1)=y$.\\
    Case 2: $y\not\in\mathbb{Z}$. For cases where $y\not\in\mathbb{Z}$ e.g $a$, we must find an element $b$ s.t $s_N(b)=a$. We can do so by similarly letting $b=a-1\not\in\mathbb{Z}$.\\
    Since $s_N$ is injective and surjective, $s_N$ is a bijection onto $N$.   
\end{itemize}
\subsection*{Part C}
    \begin{itemize}
        \item [(a)] We need to show $\sigma_1(x)$ is a bijection that preserves the structure of $\mathcal{N}$. 
        To show $\sigma_1(x)$ is injective, let $x,y\in N$. 
        Suppose $\sigma_1(x)=\sigma(y)$. 
        It follows either both $\sigma_1(x),\sigma(y)_1\not\in A$ or $\sigma_1(x),\sigma(y)_1\in A$. 
        The first case is trivial because $\sigma_1$ is the identity. 
        The second case $\sigma_1(x)=\sigma(y)=s_N^k(a)$ for some $k\in\mathbb{Z}$. 
        It follows $x=y=s_N^{k-1}(a)$ because $s_N^{-1}$ is the inverse of $s_N$.
        Hence, $\sigma_1$ is injective.
        To show $\sigma_1(x)$ is surjective, let $b\in N$.
        It follows either $b\not\in A$ or $b\in A$. 
        The first case is trivial as $\sigma_1$ is the identity.
        The second case $b=s_N^k(a)$ for some $k\in\mathbb{Z}$.
        It follows there exists $x=s_N^{k-1}(a)$ s.t $\sigma_1(x)=b$.
        Hence, $\sigma_1$ is surjective.
        $0\not\in A$, so $\sigma_1(0)=0$ i.e $\sigma_1$ preserves constants.
        $s_N(\sigma_1(x))=s_N(x)=\sigma_1(s_N(x))$ for $x\not\in A$ and $s_N(\sigma_1(x))=s_N(s_N(x))=\sigma_1(s_N(x))$ for $x\in A$ i.e $\sigma_1$ preserves functions.
        Since $\sigma_1$ is a bijection that preserves the structure of $\mathcal{N}$, $\sigma_1$ is an automorphism.
        \item [(b)] In $2Ca$ we showed $\sigma_1$ is an automorphism that moves $a$. Thus, $a$ cannot be definable in $\mathcal{N}$.
    \end{itemize}
\subsection*{Part D}
    \begin{itemize}
        \item [(a)] Assume $2\mathbb{Z}$ is definable in $\mathcal{M}_1$ by a formula $\phi(x)$. 
        Let $B\subseteq N$ be the definable subset of $N$ defined by $\phi(x)$ in $\mathcal{N}$. 
        Moreover $\mathcal{M}_1\models\phi(n)\Leftrightarrow n\in 2\mathbb{Z}$ and $\mathcal{N}\models\phi(n)\Leftrightarrow n\in B$.
        In $\mathcal{M}_1$ $B$ corresponds to $2\mathbb{Z}$, and since $\mathcal{M}_1$ and $\mathcal{N}$ are elementary equivalent.
        Since $\phi(x)$ defines $2\mathbb{Z}$ in $\mathcal{M}_1$, $B$ must have similar properties in $\mathcal{N}$ by elementary equivalence.
        In $\mathcal{M}_1$ $s$ maps even integers to odd integers to odd integers and vice versa. 
        $\mathcal{M}_1\models\phi(s(n))\Leftrightarrow n\not\in 2\mathbb{Z}$ and $\mathcal{M}_1\models\phi(n)\Leftrightarrow s(n)\not\in 2\mathbb{Z}$
        To maintain elementary equivalence, $s_N$ must map elements in $B$ to elements not in $B$ and vice versa.
        Formally, if we consider any $x\in N$, $x\in B\Leftrightarrow s_N(x)\not\in B$ by properties of the successor function. 
        Thus, for any $x\in N$ either $x\in B$ or $s_N(x)\in B$ but not both.
        \item [(b)] $\sigma_1$ is an automorphism on $N$. 
        If $B$ were definable then $x\in B$ iff $\sigma_1(x)\in B$.
        However, we showed in Part D(a) for and $x\in N$ or $s_N(x)\in N$ but not both. 
        Thus, either $x\in B$ or $\sigma(x)\in B$ but not both.
        It follows $\phi(x)$ does not define $B$ in $\mathcal{N}$, so by elementary equivalence, $\phi(x)$ cannot define $2\mathbb{Z}$ in $\mathcal{M}_1$.
    \end{itemize}
\section*{Problem 3}
\subsection*{Part A}
\begin{itemize}
    \item [(a)] Reflexivity: $\forall x E(x,x)$, Symmetry: $\forall x\forall y E(x,y)\rightarrow E(y,x)$, Transitivity: $\forall x\forall y\forall z (E(x,y)\land E(y,z))\rightarrow E(x,z)$
    \item [(b)] For every $n\ge 1$ $\forall x_1\forall x_2\ldots\forall x_n\exists y\bigwedge_{i=1}^{n}\lnot E(x_i,y)$
    \item [(c)] Existence of a class of size $n$: For every $n>0$ $\exists x_1\exists x_2\ldots\exists x_n(\underset{1\le i<j\le n}{\bigwedge}(E(x_i,x_j)\land \forall y E(y,x_1))\rightarrow(y=x_1\lor y=x_2\lor\ldots\lor y=x_n))$\\
    Uniqueness of a class of size $n$: For every $n>0$ $\forall x_1\exists x_2\ldots\exists x_n\forall y_1\exists y_2\ldots \exists y_n(\underset{1\le i<j\le n}{\bigwedge}E(x_i,x_j)\land\underset{1\le i<j\le n}{\bigwedge}E(y_i,y_j)\rightarrow\underset{1\le i\le n}{\bigvee}x_i=y_i)$
\end{itemize}
\subsection*{Part B}
\begin{itemize}
    \item [(a)] Let $\mathcal{M}$ be a model with the following structure:
    \begin{itemize}
        \item Exactly $1$ equivalence class of size $1$
        \item Exactly $1$ equivalence class of size $2$
        \item Exactly $1$ equivalence class of size $3$
        \item And so on...
    \end{itemize}
    Let $\mathcal{N}$ be a model with the following structure:
    \begin{itemize}
        \item Exactly $1$ equivalence class of size $1$
        \item Exactly $1$ equivalence class of size $2$
        \item Exactly $1$ equivalence class of size $3$
        \item And so on...
        \item Additionally $\mathcal{N}$ contains an infinite equivalence class.
    \end{itemize}
    Both models satify all the axioms in the theory $T$. However, $\mathcal{M}$ and $\mathcal{N}$ are not isomorphic because isomorphism preserve the size and structure of equivalence classes.
    \item [(b)] For each model $\mathcal{M}_n$ we have the following structure:
    \begin{itemize}
        \item Exactly $1$ equivalence class of size $1$
        \item Exactly $1$ equivalence class of size $2$
        \item Exactly $1$ equivalence class of size $3$
        \item And so on...
        \item Additionally $\mathcal{M}_n$ contains an equivalence class of size $\alpha+n$ where $\alpha$ is the smallest infinite cardinality.
    \end{itemize}
    Each of the models satify the axioms of $T$, but none of the $\{\mathcal{M}_n:n\in\mathbb{N}\}$ can be isomorphic to each other because the infinite equivalence classes cannot be mapped to one another because they have different cardinalities.
\end{itemize}
\subsection*{Part C}
\begin{itemize}
    \item [(a)]\begin{itemize}
        \item [i.] Reflexivity: $\forall x E(x,x)$, Symmetry: $\forall x\forall y E(x,y)\rightarrow E(y,x)$, Transitivity: $\forall x\forall y\forall z (E(x,y)\land E(y,z))\rightarrow E(x,z)$
        \item [ii.] For every $n\ge 1$ $\forall x_1\forall x_2\ldots\forall x_n\exists y\bigwedge_{i=1}^{n}\lnot E(x_i,y)$
        \item [iii.] For every $n\ge 1$ $\forall x_1\ldots\forall x_n(\underset{1\le i\le j\le n}{\bigwedge}E(x_i,x_j))\rightarrow\exists y(\underset{1\le i\le n}{\bigwedge}E(y,x_i)\land \underset{1\le i\le n}{\bigwedge}y\neq x_i)$
    \end{itemize}
    \item [(b)] Let $\mathcal{M}$ and $\mathcal{N}$ be countable models of $T_1$. 
    We show that we can construct an isomorphism $\sigma$ between $\mathcal{M}$ and $\mathcal{N}$.\\
    We do so by building the isomorphism step by step checking that the structure is preserved.
    Fix some enumeration $\{m_1,m_2\ldots\}$. 
    For each $m_k\in M$, we can find a corresponding $\sigma(m_k)=n_k\in N$ s.t for $1\le i,j\le k$ $(m_i,m_j)\in E^M$ iff $(\sigma(m_i),\sigma(m_j))\in E^N$. 
    We can guarantee that we can find such an $n_k$ for each $k\in\mathbb{N}$ because each model has infinitely many equivalence classes which each contains infinitely many elements. 
    This guarantees our isomorphism is injective.
    We also want to guarantee that our isomorphism is surjective. 
    For any $k\in \mathbb{N}$ consider an $n\in N$ that has not already been assigned. 
    Because $M$ is countably infinite that contains infinitely many equivalence classes which each contains infinitely many elements, we are guaranteed to be able to find an unassigned $m$ s.t for $1\le i\le k$ $(m_i,m)\in E^M$ iff $(\sigma(m_i),n)\in E^N$.
    Proceding in this manner, we can construct an isomorphism between $M$ and $N$.
    \item [(c)] A theory is complete if, for every sentence $\varphi$ in the language $L$, either $T_1\vdash\varphi$ or $T_1\vdash\lnot\varphi$.
    However, we showed that any two countable models of $T_1$ are isomorphic. It follows any two countable models of $T_1$ are elementary equivalent. 
    This implies that any two models must satisfy the same sentences in $L$. 
    In other words, we can't have two different models of $T_1$ where one model satisfies $\varphi$ and the other satifies $\lnot\varphi$.
    Therefore, if every model of $T_1$ satifies a sentence $\varphi$, then $T_1\models\varphi$ and $T_1\not\models\lnot\varphi$. Thus, $T_1\vdash\varphi$ and $T_1\not\vdash\lnot\varphi$ by soundedness completeness.
\end{itemize}
\end{document}